\chapter{Background}
\label{ch:background}

This chapter gives a brief introduction to the subject of anomaly detection. The framework of tasks and problems used throughout the paper is presented and justified.

\section{Anomaly detection}
\label{sect:adb}

In essence, anomaly detection is the task of automatically detecting items (\emph{anomalies}) in datasets that in some sense do not fit in with the rest of those datasets (i.e.\ are \emph{anomalous} with regard to the rest of the data). The nature of both the datasets and anomalies are dependent on the specific application in which anomaly detection is applied, and vary drastically between application domains. As an illustration of this, consider the two datasets shown in Figures~\ref{fig:example1} and~\ref{fig:example1}. While these are similar in the sense that they both involve sequences, they differ in the type of data points (real-valued vs.\ symbolic), the structure of the dataset (one long sequence vs.\ several sequences), as well as the nature of the anomalies (a subseqeuence vs.\ one sequence out of many). Several surveys~\cite{hodge}~\cite{bakar}~\cite{chandola}~\cite{agyemang} and books~\cite{barnett}~\cite{hawkins}~\cite{leroy} have been published which treat various anomaly detection applications in greater depth.

\begin{figure}[htb]
    \centering
    \includegraphics[width=\textwidth]{resources/anomaly_example}
    \caption{\small Real-valued sequence with an anomaly at the center.}
    \vspace{-0pt}
\label{fig:example2}
\end{figure}

Like many other concepts in machine learning and data science, the term `anomaly detection' does not refer to any single well-defined problem. Rather, it is an umbrella term encompassing a collection of loosely related techniques and problems. Anomaly detection problems are encountered in nearly every domain in business and science in which data is collected for analysis. Naturally, this leads to a great diversity in the applications and implications of anomaly detection techniques. Due to this wide scope, anomaly detection is continuously being applied to new domains despite having been researched for decades.

\begin{figure}[htb]
    \centering
    \begin{tabular}{| l | l l l l l l l l |}
        \hline
        $\mathbf{S_1}$ & login & passwd & mail & ssh & \dots & mail & web & logout \\ \hline
        $\mathbf{S_2}$ & login & passwd & mail & web & \dots & web & web & logout \\ \hline
        $\mathbf{S_3}$ & login & passwd & mail & ssh & \dots & web & web & logout \\ \hline
        $\mathbf{S_4}$ & login & passwd & web & mail & \dots & web & mail & logout \\ \hline
        $\mathbf{S_5}$ & login & passwd & login & passwd & login & passwd & \dots & logout \\\hline
    \end{tabular}
    \caption{Several sequences of user commands. The bottom sequence is anomalous compared to the others.}
\label{fig:example1}
\end{figure}

In other words, anomaly detection as a subject encompasses a diverse set of problems, methods, and applications. Different anomaly detection problems and methods often have few similarities, and no unifying theory exists. Indeed, the eventual discovery of such a theory seems highly unlikely, considering the subjectivity inherent to most anomaly detection problems. Even the term `anomaly detection' itself has evaded any widely accepted definition~\cite{hodge} in spite of multiple attempts.

Despite this diversity, anomaly detection problems from different domains often share some structure, and studying anomaly detection as a subject can be useful as a means of understanding and exploiting such common structure. Anomaly detection methods are vital analysis tools in a wide variety of domains, and the set of scientific and commercial domains which could benefit from improved anomaly detection methods is huge. Indeed, due to increasing data volumes, exhaustive manual analysis is (or will soon be) prohibitively expensive in many domains, rendering effective automated anomaly detection critical to future development.

\section{On Anomaly Detection Research}

Most anomaly detection research work consists of either taking existing methods and applying them to new applications (i.e.\ on new types of data), or investigating new methods for previously studied applications. In order to handle the increasing need for effective anomaly detection in many areas of business and science it is vital that these activities can be performed in a highly automated and straight-forward manner. However, there are a few issues with the current state of the subject, which make anomaly detection research needlessly complicated.

Firstly, comparing different anomaly detection methods found in the literature is difficult, since even though it might not appear so at first glance, papers on anomaly detection often target subtly different problems. For instance, TODO. This renders direct comparisons problematic and makes it hard to assess which methods are appropriate to use in new applications. A systematic way of comparing anomaly detection methods would be helpful in mitigating this problem.

The second problem is that there is often a lack of reproducibility of produced results. Due in part to the subjective nature of the subject, and in part to a historical lack of freely available datasets, new methods are often not adequately compared to previous methods. Furthermore, the performance of many anomaly detection methods is often highly dependent on parameter choices, and only the results for the best parameter values (which might be difficult to find) are often presented~\cite{keogh5}. Finally, source code is often unavailable, which makes verification a tedious process. These issues, when taken together, make it hard to reproduce results, which in turn makes anomaly detection research needlessly difficult.

This work attempts to simplify anomaly detection research by addressing the above issues. First, a general framework for systematically comparing anomaly detection problem formulations is presented, the purpose of which is to help highlight similarities and differences between problems, and thereby simplify the application of existing methods to new domains by mitigating the first problem above.

This framework is then used to formalise the subject of anomaly detection in the domain of real-valued time series, and to reformulate the activity of finding appropriate methods for specific datasets in this domain as an optimization problem over the set of possible algorithms. It is then shown that solutions to this optimization problem can be algorithmically approximated for a large class of algorithms (including most previously published methods).

Finally, a software implementation of this optimization problem is presented, along with some preliminary performance results. This mitigates the second problem above for anomaly detection in real-valued time series, by providing an environment in which previous methods can be easily replicated and compared on arbitrary datasets.

One distinction that is important to make is between problems and methods. Informally, the process of finding an appropriate anomaly detection method for some application consists of two steps. First, an appropriate problem formulation must be found, which accurately captures intuitive notions of what constitutes an anomaly in the specific application. Next, a method of solving or finding approximate solutions to this problem must be constructed.

Due to the subjective nature of anomaly detection, radically different problem formulations might be appropriate for applications that are superficially very similar. Furthermore, there is often no obvious connection between the intuitive notion of what constitutes an anomaly in some application and the problem formulations which most accurately capture that notion, so prospective problem formulations must themselves be empirically evaluated.

This means that unless specific information is available on what problems are appropriate for a given application, finding the correct problem formulations should take priority over formulating methods. Finding efficient methods should be done only after it has been shown that the problem the methods are solving is relevant to the application. In the literature, methods are often emphasised at the expense of problems. Since the goal of this project was to find methods of automating the research process for arbitrary applications, there is instead a heavy emphasis on finding problems in this report. 

\section{Problem formulation}
\label{sect:problem_formulation}
As a first step towards automating anomaly detection research, the problem of finding anomaly detection methods for some application must be formalised. In this section, the first step of the anomaly detection research process---finding an appropriate problem---is formulated as an optimization problem. The remainder of this report deals with this problem.

Informally, the idea is to search the set of possible problem formulations for the specific problem formulation with solutions (i.e.\ anomaly scores) that come the closest to the solutions which would be produced by a domain expert.

To begin with, the sets of valid problem inputs (datasets) and outputs (solutions) must be defined. Here, we simply assume that some set $D$ has been defined consisting of all the valid datasets for the current application (such as the set of all finite collections of images of some specific size and bit depth), and that some set $S$ has been defined consisting of all valid solutions (such as the indices of the two most anomalous images in a set).

TODO: replace $S$ with $\forall d \in D: S_d$?

To begin with, a formal description of this problem requires that a set of all possible problem formulations is constructed. Here, it is simply assume that this set has been defined, and that it consists of the set of valid formulae in some logic sufficiently expressive to capture all relevant problem formulations. Let us call this set $P$.

Next, an objective function must be formulated, which associates with each problem formulation $p \in P$ how well $p$ captures the anomalies of our hypothetical domain expert. Since how the objective function is computed is not relevant to the optimization problem formulation, it is helpful to introduce an oracle $O(p, i): P \times D \rightarrow S$, which takes a problem $p \in P$ and an input dataset $i \in D$, and computes the associated solution $s_{p,i}$. The success of $p$ in capturing the anomalies in $i$ can then be stated as $\epsilon(i, O(p, i))$, where $\epsilon: D \times S \rightarrow \mathbb{R}^+$ is an error function, consistent with the assessments of our hypothetical domain expert, that assigns a value to each $s \in S$ according to how well it captures the anomalies in $i$.

Finally, since the goal is to minimise the expected error for datasets sampled from the given application, a random variable $I$ over $D$ must be introduced that models the probability of generating any given $d \in D$. A suitable objective function would then be $\mathbb{E}_I [\epsilon(i, O(p, i))]$.

The goal then, is to find 
\[
    p_{opt} = \argmin_{p \in P}\mathbb{E}_I [ \epsilon(i, O(p, i)) ].
\]

Here, $p_{opt}$ corresponds to the best possible problem formulation for the specific $\epsilon$, and $O'(i) = O(p_{opt}, i)$ to the algorithm which computes the answer. Of course, computing either of these is impossible, for a few reasons. The biggest theoretical hurdle is that, for any logic $P^*$ sufficiently complicated to encompass non-trivial problems, the problem of producing a solution $S$ for an arbitrary problem definition $P$ with input data $I$ is uncomputable (TODO: verify), so the oracle $O$ described above can not exist. For the optimization problem to be of any real-world use, $P$ must be restricted to a set of problems that can be solved by some computable oracle $O$.

Another problem is that it is generally not possible to directly formulate either $X$ or $\epsilon$. To appropriately select $X$ would require a steady stream of incoming datasets, something that is typically not available for most applications. Similarly, since $\epsilon$ essentially represents an idealised, objective domain expert's opinion of what parts of $X$ constitute anomalies, it can not be computed without (implicit) knowledge of $P^o$. To get around these issues, $X$ and $\epsilon$ must be replaced with a set of training data $T \subset D$. Unless specific knowledge of the domain is available from which $\epsilon$ can be reconstructed, this data must also be \emph{labeled}, i.e.\ with each $t_i \in T$ must be associated a $s_i \in S$.

The rest of this report builds upon this problem formulation. In chapter~\ref{ch:??}, an algorithm is constructed which can solve a large class anomaly detection problems for time series. This algorithm is then taken to be a partial representation $O^*$ of $O$. Correspondingly $P$ is replaced by the set $P^* \subset P$ of problems which can be solved by $O^*$. Finally, $\epsilon$ can be approximated by $\epsilon^*(s_a, s_b) = \delta(s_a, s_b)$ for some suitable distance measure $\delta$. This enables the computation of the following estimate of $p_{opt}$:
\[
    p^*_{opt} = \argmin_{p \in P^*} \sum_{(i_j, s_j) \in T} \epsilon^*(s_j, O(p, i_j)).
\]

With this approach, finding the most accurate anomaly detection algorithm (supported by $O^*$) for $T$ becomes an optimization problem over $P^*$, where the objective function is $\sum_{(i_j, s_j) \in T} \epsilon^*(s_j, O(p, i_j))$.

The main purpose of the framework presented in the next chapter is to simplify the formulation of a suitable restricted oracle $O^*$, by providing a means of systematically constraining $P^*$. This approach is taken in Chapter~\ref{ch:??} to construct a oracle that can solve most of the time series anomaly detection problems found in the literature, while remaining tractable.
