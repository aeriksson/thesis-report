\chapter{Background}
\label{ch:background}

This chapter gives a brief introduction to the subject of anomaly detection. The framework of tasks and problems used throughout the paper is presented and justified.

\section{Anomaly detection}
\label{sect:adb}

In essence, anomaly detection is the task of automatically detecting items (\emph{anomalies}) in data sets that in some sense do not fit in with the rest of those data sets (i.e.\ are \emph{anomalous} with regard to the rest of the data). The nature of both the data sets and anomalies are dependent on the specific application in which anomaly detection is applied, and vary drastically between application domains. As an illustration of this, consider the two data sets shown in Figures~\ref{fig:example1} and~\ref{fig:example1}. While these are similar in the sense that they both involve sequences, they differ in the type of data points (real-valued vs.\ symbolic), the structure of the data set (one long sequence vs.\ several sequences), as well as the nature of the anomalies (a subseqeuence vs.\ one sequence out of many). Several surveys~\cite{hodge}~\cite{bakar}~\cite{chandola}~\cite{agyemang} and books~\cite{barnett}~\cite{hawkins}~\cite{leroy} have been published which treat various anomaly detection applications in greater depth.

\begin{figure}[htb]
    \centering
    \includegraphics[width=\textwidth]{resources/anomaly_example}
    \caption{\small Real-valued sequence with an anomaly at the center.}
    \vspace{-0pt}
\label{fig:example2}
\end{figure}

Like many other concepts in machine learning and data science, the term `anomaly detection' does not refer to any single well-defined problem. Rather, it is an umbrella term encompassing a collection of loosely related techniques and problems. Anomaly detection problems are encountered in nearly every domain in business and science in which data is collected for analysis. Naturally, this leads to a great diversity in the applications and implications of anomaly detection techniques. Due to this wide scope, anomaly detection is continuously being applied to new domains despite having been researched for decades.

\begin{figure}[htb]
    \centering
    \begin{tabular}{| l | l l l l l l l l |}
        \hline
        $\mathbf{S_1}$ & login & passwd & mail & ssh & \dots & mail & web & logout \\ \hline
        $\mathbf{S_2}$ & login & passwd & mail & web & \dots & web & web & logout \\ \hline
        $\mathbf{S_3}$ & login & passwd & mail & ssh & \dots & web & web & logout \\ \hline
        $\mathbf{S_4}$ & login & passwd & web & mail & \dots & web & mail & logout \\ \hline
        $\mathbf{S_5}$ & login & passwd & login & passwd & login & passwd & \dots & logout \\\hline
    \end{tabular}
    \caption{Several sequences of user commands. The bottom sequence is anomalous compared to the others.}
\label{fig:example1}
\end{figure}

In other words, anomaly detection as a subject encompasses a diverse set of problems, methods, and applications. Different anomaly detection problems and methods often have few similarities, and no unifying theory exists. Indeed, the eventual discovery of such a theory seems highly unlikely, considering the subjectivity inherent to most anomaly detection problems. Even the term `anomaly detection' itself has evaded any widely accepted definition~\cite{hodge} in spite of multiple attempts.

Despite this diversity, anomaly detection problems from different domains often share some structure, and studying anomaly detection as a subject can be useful as a means of understanding and exploiting such common structure. Anomaly detection methods are vital analysis tools in a wide variety of domains, and the set of scientific and commercial domains which could benefit from improved anomaly detection methods is huge. Indeed, due to increasing data volumes, exhaustive manual analysis is (or will soon be) prohibitively expensive in many domains, rendering effective automated anomaly detection critical to future development.

\section{On Anomaly Detection Research}

Most anomaly detection research work consists of either taking existing methods and applying them to new applications (i.e.\ on new types of data), or investigating new methods for previously studied applications. In order to handle the increasing need for effective anomaly detection in many areas of business and science it is vital that these activities can be performed in a highly automated and straight-forward manner. However, there are a few issues with the current state of the subject, which make anomaly detection research needlessly complicated.

Firstly, comparing different anomaly detection methods found in the literature is difficult, since even though it might not appear so at first glance, papers on anomaly detection often target subtly different problems. For instance, TODO. This renders direct comparisons problematic and makes it hard to assess which methods are appropriate to use in new applications. A systematic way of comparing anomaly detection methods would be helpful in mitigating this problem.

The second problem is that there is often a lack of reproducibility of produced results. Due in part to the subjective nature of the subject, and in part to a historical lack of freely available datasets, new methods are often not adequately compared to previous methods. Furthermore, the performance of many anomaly detection methods is often highly dependent on parameter choices, and only the results for the best parameter values (which might be difficult to find) are often presented~\cite{keogh5}. Finally, source code is often unavailable, which makes verification a tedious process. These issues, when taken together, make it hard to reproduce results, which in turn makes anomaly detection research needlessly difficult.

This work attempts to simplify anomaly detection research by addressing the above issues. First, a general framework for systematically comparing anomaly detection problem formulations is presented, the purpose of which is to help highlight similarities and differences between problems, and thereby simplify the application of existing methods to new domains by mitigating the first problem above.

This framework is then used to formalize the subject of anomaly detection in the domain of real-valued time series, and to reformulate the activity of finding appropriate methods for specific data sets in this domain as an optimization problem over the set of possible algorithms. It is then shown that solutions to this optimization problem can be algorithmically approximated for a large class of algorithms (including most previously published methods).

Finally, a software implementation of this optimization problem is presented, along with some preliminary performance results. This mitigates the second problem above for anomaly detection in real-valued time series, by providing an environment in which previous methods can be easily replicated and compared on arbitrary datasets.
