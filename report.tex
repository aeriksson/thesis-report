\documentclass[a4paper,twoside]{report}
\usepackage[utf8]{inputenc}
\usepackage[T1]{fontenc}
\usepackage[english]{babel}
\usepackage{amsmath}
\usepackage{amssymb}
\usepackage{wrapfig}
\usepackage{algpseudocode}
\usepackage[a4paper,inner=2.5cm,outer=2.5cm,top=2.5cm,bottom=2.5cm,pdftex]{geometry}
\usepackage[pdftex]{graphicx}
\usepackage{epsfig}
\usepackage{epstopdf}
\usepackage{psfrag}
\usepackage{fancyhdr}
\usepackage{natbib}
\usepackage{babelbib}
\usepackage[hidelinks]{hyperref}
\usepackage{caption}
\usepackage{subcaption}
\usepackage{placeins}
\usepackage{float}
\usepackage{graphicx}
\usepackage{color,soul}

\date{}

%Section number in equation numbering
\numberwithin{equation}{section}


% Line spacing
\makeatletter
\let\std@footnotetext\@footnotetext
\usepackage{setspace}
\setstretch{1.4}
\let\@footnotetext\std@footnotetext
\makeatother


%Section and subsection styling
\makeatletter
\renewcommand{\chapter}{%
\@startsection
        {chapter}{1}{-20pt}{-4ex \@plus -1ex \@minus -.2ex}%
        {3.4ex \@plus.2ex}{\centering\Huge\scshape}%
}

\renewcommand{\section}{%
\@startsection
	{section}{1}{-20pt}{-4ex \@plus -1ex \@minus -.2ex}%
	{2.3ex \@plus.2ex}{\centering\Large\scshape}%
}
\renewcommand{\subsection}{%
\@startsection
	{subsection}{1}{-20pt}{-2.5ex \@plus -1ex \@minus -.2ex}%
	{1.8ex \@plus.2ex}{\centering\large\scshape}%
}
\makeatother

%No section numbering. References become empty.
%\renewcommand{\thesection}{}

%Margins
%\setlength{\topmargin}{0.4in}
%\setlength{\textheight}{9in}
%\setlength{\oddsidemargin}{.125in}
%\setlength{\textwidth}{6.25in}
\setlength{\headheight}{15pt}

%Settings for references
\bibpunct{[}{]}{;}{n}{,}{,}

%Theorems, Lemmas, Definitionos and Formulations
\newtheorem{definition}{Definition}[chapter]
\newtheorem{task}{Task}[chapter]
\newtheorem{problem}{Problem}[chapter]

\newcommand{\vect}[1]{\boldsymbol{#1}}
\newcommand{\argmin}{\operatornamewithlimits{argmin}}
\newcommand{\argmax}{\operatornamewithlimits{argmax}}

\newcommand{\HRule}{\rule{\linewidth}{0.5mm}}

\setcounter{tocdepth}{1}

%Begin Document
\begin{document}

\title{A Framework for Anomaly Detection with Applications to Sequences}
\author{André Eriksson}
\date{May 2014}
\blurb{Master's Thesis at NADA\\Supervisor: Hedvig Kjellström\\Examiner: TODO}
\trita{TRITA xxx yyyy-nn}


%Headers and footers
\pagestyle{fancy}
\fancyhead{}
\fancyhead[RE,LO]{\leftmark}
%\newcommand{\hdrsection}{\markright}
\renewcommand{\headrulewidth}{0pt}
%\fancyhead[L]{}
%\fancyhead[R]{\thesection \hdrsection}

\setcounter{page}{0}
%\thispagestyle{empty}

\begin{abstract}

Anomaly detection is an important issue in data mining and analysis, with applications in almost every area in science, technology and business that involves data collection. The development of general anomaly detection techniques can therefore have a large impact on data analysis across many domains. In spite of this, little work has been done to consolidate the different approaches to the subject.

In this report, this deficiency is addressed in the target domain of temporal machine-generated data. To this end, new theory for comparing and reasoning about anomaly detection tasks and methods is introduced, which facilitates a problem-oriented rather than a method-oriented approach to the subject. Using this theory as a basis, the possible approaches to anomaly detection in the target domain are discussed, and a set of interesting anomaly detection tasks is highlighted.

One of these tasks is selected for further study: \emph{the detection of subsequences that are anomalous with regards to their context within long univariate real-valued sequences}. A framework for relating methods derived from this task is developed, and is used to derive new methods and an algorithm for solving a large class of derived problems. Finally, a software implementation of this framework along with a set of evaluation utilities is discussed and demonstrated.
 
\end{abstract}

\tableofcontents
\clearpage

\setlength{\parindent}{0pt}
\setlength{\parskip}{2ex plus 0.5ex minus 0.2ex}

\chapter{Introduction}

\begin{figure}[htb]
    \vspace{-10pt}
    \begin{center}
        \includegraphics[trim= 30mm 10mm 30mm 10mm, clip, width=\textwidth]{resources/citations}
    \end{center}
    \vspace{-20pt}
    \caption{\small Approximate number of papers (by year) published between $1980$ and $2011$ containing the terms ``anomaly detection'', ``outlier detection'' and ``novelty detection''. All three terms exhibit strong upward trends in recent years. Source: Google Scholar.}
    \vspace{-0pt}
\label{fig:citations}
\end{figure}

This report is the result of a master's thesis project at the KTH Royal Institute of Technology, performed partly in conjunction with an internship at Splunk Inc.\@, based in San Francisco, California, USA\@. The goal of the project was to develop efficient and general methods of anomaly detection suitable for real-valued time series data. The main contributions of this thesis are:
\begin{enumerate}
    \item A general framework for comparing and relating anomaly detection problems and methods.
    \item An application of this framework to anomaly detection in real-valued time series.
    \item Software for automatically discovering optimal anomaly detection methods for real-valued time series datasets.
\end{enumerate}

Splunk is essentially a database and tool for storing and analyzing very large sets of sequential machine-generated data (i.e.\ machine-generated data consisting of discrete events with associated time stamps). The term \emph{machine-generated data} refers to any data consisting of discrete events that have been created automatically from a computer process, application, or other machine without the intervention of a human. Common types of machine-generated data include computer, network, or other equipment logs; environmental or other types of sensor readings; or other miscellaneous data, such as location information~\cite{machine_data}. Splunk is designed for this type of data, especially data sets where each event has an associated time stamp.

In recent years, anomaly detection has become increasingly important in a variety of domains in business, science and technology. Roughly defined as the automated detection within data sets of elements that are somehow abnormal, anomaly detection encompasses a broad set of techniques and problems. In part due to the emergence of new application domains, and in part due to the evolving nature of many traditional domains, new applications of and approaches to anomaly detection and related subjects are being developed at an increasing rate, as indicated in Figure~\ref{fig:citations}.

Since anomaly detection is an important and common problem in the domains in which Splunk is used, it was determined that Splunk could benefit for efficient, general anomaly detection methods. Furthermore, since continuous time series are easy to form from sequential machine-generated data, and readily amenable to analysis, these were selected as a focus.

In Chapter~\ref{ch:background}, various background information pertinent to the rest of the report is presented. Specifically, the subject of anomaly detection is presented in more depth, along with some background on the issues which are relevant to the rest of the report.

Due to a lack of previous research on general (i.e.\ non-application specific) anomaly detection methods for continuous time series, new theory was required. As part of the project, a framework was developed for interrelating about anomaly detection problems. This framework can be used to consolidate different approaches to anomaly detection and to simplify reasoning about and comparing anomaly detection problems and methods systematically. Chapter~\ref{ch:?} introduces this framework.

Next, the framework was applied to anomaly detection in real-valued time series. As part of this, a general algorithm for solving a large class of time series anomaly detection problems was introduced. This algorithm can then be used together with an optimization strategy to automate the process of finding methods suitable for new datasets.

A software implementation of this algorithm was then developed. Called \texttt{ad-eval}, the implementation was designed as a minimalistic Python toolkit. It has been released under an open source license at \url{http://github.com/aeriksson/ad-eval}, in order to facilitate reproducibility in anomaly detection research. Both the application of the framework to time series, and \texttt{ad-eval} are treated in Chapter~\ref{ch:?}.

Finally, a simple evaluation of a few implemented anomaly detection methods was performed using \texttt{ad-eval}, in order to demonstrate its use. The source code used in this evaluation was designed to be highly modular and was made available as part of the \texttt{ad-eval} source code repository, to guarantee the reproducibility of the results and to facilitate more thorough evaluations once adequate data becomes available. The results are presented in Chapter~\ref{ch:results}.

The report is concluded in Chapter~\ref{ch:discussion} with a summary of the project and a few possible directions for future work.

\chapter{Background}
\label{ch:background}

This chapter gives a brief introduction to the subject of anomaly detection. The framework of tasks and problems used throughout the paper is presented and justified.

\section{Anomaly detection}
\label{sect:adb}

In essence, anomaly detection is the task of automatically detecting items (\emph{anomalies}) in data sets that in some sense do not fit in with the rest of those data sets (i.e.\ are \emph{anomalous} with regard to the rest of the data). The nature of both the data sets and anomalies are dependent on the specific application in which anomaly detection is applied, and vary drastically between application domains. As an illustration of this, consider the two data sets shown in Figures~\ref{fig:example1} and~\ref{fig:example1}. While these are similar in the sense that they both involve sequences, they differ in the type of data points (real-valued vs.\ symbolic), the structure of the data set (one long sequence vs.\ several sequences), as well as the nature of the anomalies (a subseqeuence vs.\ one sequence out of many). Several surveys~\cite{hodge}~\cite{bakar}~\cite{chandola}~\cite{agyemang} and books~\cite{barnett}~\cite{hawkins}~\cite{leroy} have been published which treat various anomaly detection applications in greater depth.

\begin{figure}[htb]
    \centering
    \includegraphics[width=\textwidth]{resources/anomaly_example}
    \caption{\small Long real-valued sequence with an anomaly at the center.}
    \vspace{-0pt}
\label{fig:example2}
\end{figure}

Like many other concepts in machine learning and data science, the term `anomaly detection' does not refer to any single well-defined problem. Rather, it is an umbrella term encompassing a collection of loosely related techniques and problems. Anomaly detection problems are encountered in nearly every domain in business and science in which data is collected for analysis. Naturally, this leads to a great diversity in the applications and implications of anomaly detection techniques. Due to this wide scope, anomaly detection is continuously being applied to new domains despite having been researched for decades.

\begin{figure}[htb]
    \centering
    \begin{tabular}{| l | l l l l l l l l |}
        \hline
        $\mathbf{S_1}$ & login & passwd & mail & ssh & \dots & mail & web & logout \\ \hline
        $\mathbf{S_2}$ & login & passwd & mail & web & \dots & web & web & logout \\ \hline
        $\mathbf{S_3}$ & login & passwd & mail & ssh & \dots & web & web & logout \\ \hline
        $\mathbf{S_4}$ & login & passwd & web & mail & \dots & web & mail & logout \\ \hline
        $\mathbf{S_5}$ & login & passwd & login & passwd & login & passwd & \dots & logout \\\hline
    \end{tabular}
    \caption{Several sequences of user commands. The bottom sequence is anomalous compared to the others.}
\label{fig:example1}
\end{figure}

In other words, anomaly detection as a subject encompasses a diverse set of problems, methods, and applications. Different anomaly detection problems and methods often have few similarities, and no unifying theory exists. Indeed, the eventual discovery of such a theory seems highly unlikely, considering the subjectivity inherent to most anomaly detection problems. Even the term `anomaly detection' itself has evaded any widely accepted definition~\cite{hodge} in spite of multiple attempts.

Despite this diversity, anomaly detection problems from different domains often share some structure, and studying anomaly detection as a subject can be useful as a means of understanding and exploiting such common structure. Anomaly detection methods are vital analysis tools in a wide variety of domains, and the set of scientific and commercial domains which could benefit from improved anomaly detection methods is huge. Indeed, due to increasing data volumes, exhaustive manual analysis is (or will soon be) prohibitively expensive in many domains, rendering effective automated anomaly detection critical to future development.

\section{On Anomaly Detection Research}

Most anomaly detection research work consists of either taking existing methods and applying them to new applications (i.e.\ on new types of data), or investigating new methods for previously studied applications. In order to handle the increasing need for effective anomaly detection in many areas of business and science it is vital that these activities can be performed in a highly automated and straight-forward manner. However, there are a few issues with the current state of the subject, which make anomaly detection research needlessly complicated.

Firstly, comparing different anomaly detection methods found in the literature is difficult, since even though it might not appear so at first glance, papers on anomaly detection often target subtly different problems. For instance, TODO. This renders direct comparisons problematic and makes it hard to assess which methods are appropriate to use in new applications. A systematic way of comparing anomaly detection methods would be helpful in mitigating this problem.

The second problem is that there is often a lack of reproducibility of produced results. Due in part to the subjective nature of the subject, and in part to a historical lack of freely available datasets, new methods are often not adequately compared to previous methods. Furthermore, the performance of many anomaly detection methods is often highly dependent on parameter choices, and only the results for the best parameter values (which might be difficult to find) are often presented~\cite{keogh5}. Finally, source code is often unavailable, which makes verification a tedious process. These issues, when taken together, make it hard to reproduce results, which in turn makes anomaly detection research needlessly difficult.

This work attempts to simplify anomaly detection research by addressing the above issues. First, a general framework for systematically comparing anomaly detection problem formulations is presented, the purpose of which is to help highlight similarities and differences between problems, and thereby simplify the application of existing methods to new domains by mitigating the first problem above.

This framework is then used to formalize the subject of anomaly detection in the domain of real-valued time series, and to reformulate the activity of finding appropriate methods for specific data sets in this domain as an optimization problem over the set of possible algorithms. It is then shown that solutions to this optimization problem can be algorithmically approximated for a large class of algorithms (including most previously published methods).

Finally, a software implementation of this optimization problem is presented, along with some preliminary performance results. This mitigates the second problem above for anomaly detection in real-valued time series, by providing an environment in which previous methods can be easily replicated and compared on arbitrary datasets.

\section{Tasks and problems framework}
\label{sect:tasks_problems}

We now present the tasks and problems framework, which is used throughout this report to reason about anomaly detection problems and discuss tasks pertinent to the target domain. A major feature of this framework is that it provides a perspective where methods and optimizations are de-emphasized in favour of problem formulations. Most of the literature is method-centric, tending to shift the focus away from nuances of the anomaly detection problems these methods address and towards small and often insignificant details, thereby inhibiting the development of a high-level perspective of the subject.

In order to facilitate the comparison of different approaches, we introduce the concepts of tasks and problems. In this context, a \emph{problem} means an exact, unambiguous problem specification with a well-defined answer. In contrast, a \emph{task} is a partial specification; it leaves out one or more factors necessary to formulate a problem. Problems can be regarded as derived from one or more tasks through the specification of additional details. Similarly, tasks can themselves be seen as derived from other, more general tasks. Tasks with a high degree of generality will be referred to as \emph{high level tasks}. Finally, \emph{methods} are defined as specific algorithms for obtaining the answer to some problem.

Due to the inexact nature of anomaly detection, it is usually not clear how to precisely specify problems that can accurately capture specific types of anomalies in specific data sets, so the problem formulations themselves must be evaluated empirically. Trying to find optimal methods before a specific problem formulation has been settled upon constitutes premature optimization and is consequently inadvisable.

Since tasks and problems are only derived from other tasks through the specification of additional details, they form a hierarchy of derivations. This hierarchy can be envisioned as a directed acyclic graph, where problems and tasks constitute sinks and non-sinks respectively. If one task could be found from which the entire graph of anomaly detection tasks and problems can be reached, then this task could be used as a definition for anomaly detection (i.e.\ it would be general enough to cover all methods and problems). We propose that the following task be used for this purpose:

\begin{task}[Anomaly detection]
\label{task:anomaly_detection}
  Given a data set\footnote{Throughout this paper, a `data set' is taken to mean a (countable) set in the mathematical sense, where each item is associated with a unique index (so that multiple items can take on the same value). Tasks to which the structure of data is not relevant are formulated using data sets even though they might apply to sequences or other types of data as well.} $D$ of data, find subsets $s \subseteq D$ that are anomalous.
\end{task}

Throughout this report, this task is used as a basis from which all other tasks and problems are derived. To facilitate the derivation of new tasks and problems, it is necessary to emphasize exactly which \emph{factors} must specified in order to derive a problem from Task~\ref{task:anomaly_detection}. Enumeration of these factors, and the possible choices for each, can provide insight into the structure of the anomaly detection hierarchy (and, consequently, into the subject itself). We note that in order to derive a problem from Task~\ref{task:anomaly_detection}, at least the following five factors must be specified:
\begin{description}
  \item[Data format] The structure of the data set on which the analysis is performed.
  \item[Reference data] The data set on which the anomaly classification should be based.
  \item[Output format] What data should be produced by methods.
  \item[Anomaly measure] The heuristic used to assess how anomalous items are.
  \item[Anomaly type] Which structural properties of the data should be considered.
\end{description}
These factors, which are emphasized throughout the report, will be referred to as \emph{principal factors}. While individual problems might specify factors (such as restrictions) other than the principal factors, \emph{any} anomaly detection problem must be derived from at least five high-level tasks, each of which corresponds to a principal factor choice. This makes the principal factors uniquely interesting. The bulk of Chapter~\ref{ch:tasks} is dedicated to the examination of these factors, the discussion of possible choices for each, and the formulation of corresponding tasks.

One major advantage of this framework is that different problems can be related through the tasks from which they inherit. Studying the `most specific common task' of two problems can be useful in estimating differences and similarities between the two problems. Furthermore, as we will see, high level tasks can often be reduced to one another. Thus, the framework can be utilized to recognize when different problems are similar or can be related through reductions. For these reasons, we believe that a theory designed to relate and contrast the tasks induced by the principal factor choices can be useful in advancing the subject of anomaly detection.

Due to the inexact and subjective nature of many anomaly detection applications---the notion of an `anomaly' is typically vague and can not be given a precise definition---it is often not possible to fully specify all factors necessary to define problems. Anomaly detection often deals with the detection of unforeseen phenomena; in this case, the nature of interesting anomalies can not be known until they are detected. Therefore, tasks are often more relevant than problems when developing approaches to new domains.

The framework can profitably be applied to perform systematic analyses of anomaly detection in new domains, or to develop new anomaly detection methods. Based on the discussion above, we suggest that anomaly detection methods for new domains are best performed through the following four-stage process:
\begin{enumerate}
  \item Consider what information is known about the domain, the type of anomalies that are interesting, et cetera. Derive the most specific possible task based on this information.
  \item Based on the specified task, derive as many problems as possible by experimenting with different choices of factors.
  \item Evaluate these problems with regard to the target domain. Select one or a few problems that seem to accurately reflect the demands of the domain.
  \item Derive and implement efficient methods for solving the selected problem (or problems).
\end{enumerate}

This process was used in the project to discuss and compare approaches to anomaly detection on machine-generated data in a systematic way. This report addresses the use of the three first steps in this project; Chapter~\ref{ch:problems} treats the first step; Chapter~\ref{ch:methods}, the second step; and Chapter~\ref{ch:results} briefly covers the third step.

\chapter{A Framework for Anomaly Detection}
\label{ch:framework}

In this chapter, the principal factors outlined in Section~\ref{sect:tasks_problems} are presented in greater depth, in order to better understand of the ramifications of the potential choices.

Sections~\ref{sect:data_format} through~\ref{sect:output_format} cover individual principal factors. For each principal factor, tasks corresponding to the different choices are presented to as large an extent as possible.

A few tasks that are typically considered related to, but not necessarily part of, anomaly detection are also presented (in Section~\ref{sect:related_tasks}).

\section{Tasks and problems framework}
\label{sect:tasks_problems}

We now present the tasks and problems framework, which is used throughout this report to reason about anomaly detection problems and discuss tasks pertinent to the target domain. A major feature of this framework is that it provides a perspective where methods and optimizations are de-emphasized in favour of problem formulations. Most of the literature is method-centric, tending to shift the focus away from nuances of the anomaly detection problems these methods address and towards small and often insignificant details, thereby inhibiting the development of a high-level perspective of the subject.

In order to facilitate the comparison of different approaches, we introduce the concepts of tasks and problems. In this context, a \emph{problem} means an exact, unambiguous problem specification with a well-defined answer. In contrast, a \emph{task} is a partial specification; it leaves out one or more factors necessary to formulate a problem. Problems can be regarded as derived from one or more tasks through the specification of additional details. Similarly, tasks can themselves be seen as derived from other, more general tasks. Tasks with a high degree of generality will be referred to as \emph{high level tasks}. Finally, \emph{methods} are defined as specific algorithms for obtaining the answer to some problem.

Due to the inexact nature of anomaly detection, it is usually not clear how to precisely specify problems that can accurately capture specific types of anomalies in specific data sets, so the problem formulations themselves must be evaluated empirically. Trying to find optimal methods before a specific problem formulation has been settled upon constitutes premature optimization and is consequently inadvisable.

Since tasks and problems are only derived from other tasks through the specification of additional details, they form a hierarchy of derivations. This hierarchy can be envisioned as a directed acyclic graph, where problems and tasks constitute sinks and non-sinks respectively. If one task could be found from which the entire graph of anomaly detection tasks and problems can be reached, then this task could be used as a definition for anomaly detection (i.e.\ it would be general enough to cover all methods and problems). We propose that the following task be used for this purpose:

\begin{task}[Anomaly detection]
\label{task:anomaly_detection}
  Given a data set\footnote{Throughout this paper, a `data set' is taken to mean a (countable) set in the mathematical sense, where each item is associated with a unique index (so that multiple items can take on the same value). Tasks to which the structure of data is not relevant are formulated using data sets even though they might apply to sequences or other types of data as well.} $D$ of data, find subsets $s \subseteq D$ that are anomalous.
\end{task}

Throughout this report, this task is used as a basis from which all other tasks and problems are derived. To facilitate the derivation of new tasks and problems, it is necessary to emphasize exactly which \emph{factors} must specified in order to derive a problem from Task~\ref{task:anomaly_detection}. Enumeration of these factors, and the possible choices for each, can provide insight into the structure of the anomaly detection hierarchy (and, consequently, into the subject itself). We note that in order to derive a problem from Task~\ref{task:anomaly_detection}, at least the following five factors must be specified:
\begin{description}
  \item[Data format] The structure of the data set on which the analysis is performed.
  \item[Reference data] The data set on which the anomaly classification should be based.
  \item[Output format] What data should be produced by methods.
  \item[Anomaly measure] The heuristic used to assess how anomalous items are.
  \item[Anomaly type] Which structural properties of the data should be considered.
\end{description}
These factors, which are emphasized throughout the report, will be referred to as \emph{principal factors}. While individual problems might specify factors (such as restrictions) other than the principal factors, \emph{any} anomaly detection problem must be derived from at least five high-level tasks, each of which corresponds to a principal factor choice. This makes the principal factors uniquely interesting. The bulk of Chapter~\ref{ch:framework} is dedicated to the examination of these factors, the discussion of possible choices for each, and the formulation of corresponding tasks.

One major advantage of this framework is that different problems can be related through the tasks from which they inherit. Studying the `most specific common task' of two problems can be useful in estimating differences and similarities between the two problems. Furthermore, as we will see, high level tasks can often be reduced to one another. Thus, the framework can be utilized to recognize when different problems are similar or can be related through reductions. For these reasons, we believe that a theory designed to relate and contrast the tasks induced by the principal factor choices can be useful in advancing the subject of anomaly detection.

Due to the inexact and subjective nature of many anomaly detection applications---the notion of an `anomaly' is typically vague and can not be given a precise definition---it is often not possible to fully specify all factors necessary to define problems. Anomaly detection often deals with the detection of unforeseen phenomena; in this case, the nature of interesting anomalies can not be known until they are detected. Therefore, tasks are often more relevant than problems when developing approaches to new domains.

The framework can profitably be applied to perform systematic analyses of anomaly detection in new domains, or to develop new anomaly detection methods. Based on the discussion above, we suggest that anomaly detection methods for new domains are best performed through the following four-stage process:
\begin{enumerate}
  \item Consider what information is known about the domain, the type of anomalies that are interesting, et cetera. Derive the most specific possible task based on this information.
  \item Based on the specified task, derive as many problems as possible by experimenting with different choices of factors.
  \item Evaluate these problems with regard to the target domain. Select one or a few problems that seem to accurately reflect the demands of the domain.
  \item Derive and implement efficient methods for solving the selected problem (or problems).
\end{enumerate}

This process was used in the project to discuss and compare approaches to anomaly detection on machine-generated data in a systematic way. This report addresses the use of the three first steps in this project; Chapter~\ref{ch:problems} treats the first step; Chapter~\ref{ch:methods}, the second step; and Chapter~\ref{ch:results} briefly covers the third step.


\section{Data format}
\label{sect:data_format}

Obviously, the format of the data sets on which anomaly detection is performed varies drastically between applications, and it is not possible to exhaustively cover this factor. Nevertheless, data sets from different applications often share some fundamental characteristics, which can affect the analysis and the selection of other factors. A few such characteristics are now discussed.

To begin with, assuming that input is given as a data set $D = (d_1, d_2, \dots, d_n)$, where the $d_i$ belong to some set $X$, the structure of $X$ is vital to the choice of the other principal factors. A distinction is typically made between \emph{categorical}, \emph{discrete}, and \emph{real-valued} data sets based on the properties of $X$. We now formulate these as tasks, starting with categorical data sets:

\begin{task}[Anomaly detection for categorical data]
  Detect anomalies in a data set $D$, where $\forall d \in D: d \in A$ for some finite alphabet $A = \{a_1, a_2, \dots, a_k\}$.
\end{task}

Categorical (or \emph{symbolic}) data arises in many contexts and is comparatively easy to handle. In many cases, methods for handling categorical data have been fully developed.

If the underlying set is infinite but countable, it is referred to as discrete:

\begin{task}[Anomaly detection for discrete data]
  Detect anomalies in a data set $D$, where $\forall d \in D: d \in B$ for some $k$ and some infinite countable set $B = \{b_1, b_2, \dots\}$.
\end{task}

Usually, the underlying set is either $\mathbb{Z}$ or $\mathbb{N}$ in this task.

Finally, the underlying set might be uncountable. To the author's knowledge, the only uncountable sets encountered in anomaly detection applications is the real numbers. For this reason, other uncountable sets are ignored in the following task.

\begin{task}[Anomaly detection for real-valued data]
  Detect anomalies in a data set $D$, where $\forall d \in D: d \in \mathbb{R}^n$.
\end{task}

Many anomaly measures used for discrete or categorical data (such as information-theoretic measures or certain probabilistic models) are not valid for real-valued data. In such cases, a suitable discretization of the data might be useful. Real-valued data sets encountered in applications are often samples of processes that are assumed to be continuous. When the ordering of the samples is reflected in the data set, the data set is referred to as \emph{continuous}.

Of course, data sets consisting of multiple types of data are frequently encountered in applications. Such data sets are usually referred to as \emph{mixed}:
\begin{task}[Anomaly detection for mixed data]
  Detect anomalies in a data set $D$, where $\forall (a, b, c) \in D: a \in A, b \in B, c \in C$, where $A = \{a_1, a_2, \dots, a_l\}$ is categorical, $B = \{b_1, b_2, \dots\}$ is discrete and $C = \mathbb{R}^n$ is continuous.\footnote{Of course, any of $A, B$ or $C$ might be empty.}
\end{task}

Few methods have been proposed for dealing with mixed data. Typically, such data is split into individual series that are analyzed individually.

Another characteristic that has a large impact on the analysis is the dimensionality of the data points. A distinction is typically made between \emph{univariate} and \emph{multivariate} data. We address this with the following two tasks:
\begin{task}[Univariate anomaly detection]
  Detect anomalies in a data set $D$, where the $d \in D$ are scalars.
\end{task}
\begin{task}[Multivariate anomaly detection]
  Detect anomalies in a data set $D$, where the $d \in D$ are vectors (of length greater than one).
\end{task}
While the distinction between uni- and multivariate data might seem unnecessary, it proves important in applications. Most machine learning methods take significantly longer to learn (both in terms of time and convergence) as the dimensions of the data increase. Furthermore, some are not applicable to multivariate data at all.

Finally, we consider the presence of additional structure in the data set. Utilizing this characteristic, when it is present, is usually vital to successful analysis. For instance, many data sets encountered in applications have a natural ordering, which typically affects the intuitive notion of what should constitute an anomaly. As we will see in Sections~\ref{sect:anomaly_types} and~\ref{sect:anomaly_measures}, it is common to base the choice of other factors---such as the anomaly type or anomaly measure---on the presence of such additional structure.

Finally, it should be noted that the data is often filtered or otherwise modified to affect how it is presented to the anomaly detection method. The issue of appropriate presentations for temporal machine-generated data is considered in depth in Chapter~\ref{ch:transformations}.

\clearpage

\section{Reference data}
\label{sect:reference_data}

\begin{wrapfigure}{r}{0.5\textwidth}
    \vspace{-25pt}
    \begin{center}
        \leavevmode
        \includegraphics[width=0.5\textwidth]{resources/supervision}
    \end{center}
    \caption{{\small Euler diagram of the available reference data for the four types of supervision.}}
\label{fig:supervision}
    \vspace{-40pt}
\end{wrapfigure}

As is customary in most areas of machine learning, anomaly detection problems are classified as either \emph{supervised}, \emph{semi-supervised} or \emph{unsupervised} based on the availability of \emph{reference} (or \emph{training}) data. In contrast to the \emph{evaluation data}, which is the data in which anomalies are to be found, the reference data acts as a baseline, defining what constitutes normal and/or anomalous. The three classes of reference data are now discussed and presented as tasks, starting with supervised anomaly detection:

\begin{task}[Supervised anomaly detection]
\label{task:supervised}\ \\
    Given reference sets $N$ (containing normal reference data) and $A$ (containing anomalous reference data), find anomalies (with regard to $N$ and $A$) in a data set $D$.
\end{task}
In essence, supervised anomaly detection constitutes a traditional supervised classification problem. As such, it can be handled by any two-class classifier, such as regular support vector machines. However, characteristics shared by most anomaly detection applications make supervised approaches unsuitable.

First, anomalous reference data is almost always relatively scarce, potentially leading to skewed classes (described in~\cite{phua} and~\cite{joshi}). Furthermore, supervised anomaly detection methods are by definition unable to detect types on anomalies that are not represented in $A$, and so can not be used to find \emph{novel} anomalies. This is problematic as it is often not feasible to construct an $A$ containing all possible anomalies.

Semi-supervised anomaly detection, on the other hand, assumes the availability of either only $A$ or only $N$. Divided according to which of the two reference classes is available, the following two tasks can be specified:
\begin{task}[Semi-supervised anomaly detection with normal reference data]
  Given a reference set $N$ containing normal reference data, find anomalies (with regard to $N$) in a data set $D$.
\end{task}
\begin{task}[Semi-supervised anomaly detection with anomalous reference data]
\label{task:semisupervised_anomaly_detection}
  Given a reference set $A$ containing anomalous reference data, find anomalies (with regard to $A$) in a data set $D$.
\end{task}

While the latter task has been discussed (for instance in~\cite{dasgupta}), the vast majority of methods focus on the former. Considering the difficulties involved in obtaining anomalous reference data, as mentioned above, this should not be surprising.

Semi-supervised methods are often used more frequently than supervised methods due to the relative ease with which they can be configured.

Finally, unsupervised anomaly detection requires no reference data set, and can be defined by the following task:
\begin{task}[Unsupervised anomaly detection]
\label{task:unsupervised_anomaly_detection}
  Find subsets of some data set $D$ that are anomalous with regard to the rest of $D$.
\end{task}

For unsupervised anomaly detection, $D$ itself is referred to as the reference data.

Since reference data is not always available, unsupervised methods are typically considered to be of wider applicability than both supervised and semi-supervised methods~\cite{chandola}. However, unsupervised methods are unsuitable for certain tasks. Since reference data can not be manually specified, it is more difficult to sift out uncommon but uninteresting items in unsupervised anomaly detection than in semi-supervised anomaly detection. Furthermore, unsupervised methods will not detect anomalies that are common but unexpected (although such items are arguably not anomalies by definition).

It is useful to note that unsupervised anomaly detection tasks can often be reduced to semi-supervised anomaly detection tasks by setting $A = D$ and modifying the underlying anomaly measure such that all elements $a_i \in A$ are judged as dissimilar to themselves.

Of course, it is sometimes not feasible or desirable to compare items with the entirety of the reference data set. This is mainly the case when the data set supports additional structure, such an ordering or metric, which gives rise to a natural concept of locality within the data set. As a concrete example of such an application, consider unsupervised anomaly detection in a long sequence: often how an item compares to those items `closest' to it (in the ordering) is much more relevant to whether or not that item should be considered an anomaly than how it compares to the rest of the sequence.

In such cases, it is reasonable to associate with each individual data item a subset of the reference data, and let this subset constitute the reference data for that item. Throughout this report, such subsets will be referred to as the \emph{contexts} of the individual data items. Formally, this can be represented through a \emph{context function} that maps each subset $D'$ of the data set (or at least those subsets of interest to the analysis) to some set $C(D')$ such that $D' \cap C(D') = \emptyset$. The requirement that $D' \cap C(D') = \emptyset$ is necessary to keep elements from lowering their own anomaly scores when performing unsupervised anomaly detection. For the purposes of this report, contexts are mainly interesting for unsupervised anomaly detection, and can henceforth be assumed to refer to subsets of the evaluation data unless explicitly stated.

\begin{figure}[thb]
    \vspace{-4pt}
    \begin{center}
        \leavevmode
        \includegraphics[width=1\textwidth]{resources/contextz}
    \end{center}
    \vspace{-15pt}
    \caption{{\small Schematic view of a data set illustrating a few contexts. In each panel, the black dots represent selected items, the dark grey dots represent items in the context of the selected items, and the light grey dots indicate items not in the context of the selected items. The left panel shows the trivial context---all items are part of the context. The middle panel shows a local context of a single item. The right panel shows a local context of a subset of the data set.}}
\label{fig:contexts}
    \vspace{-5pt}
\end{figure}

Consider again the example of a long sequence. Writing this sequence as $S = (s_1, s_2, \dots, s_n)$, a reasonable context function defined for individual points could be the following:
\[
    C(s_i) = \{s_{i-w}, s_{i - w + 1}, \dots, s_{i - 1}, s_{i + 1}, s_{i + 2}, \dots, s_{i + w}\}.
\]
When using this context, which is referred to as the \emph{symmetric local context}, the local characteristics of the sequence around $s_i$ are taken into account, while the rest of the sequence is ignored.

Context functions $C(d)$ defined on individual elements $d \in D$ (such as the one above) can be naturally extended to subsets $D' \subseteq D$ of the data by defining
\[
    C(D') = \bigcup_{d \in D'} C(d) \setminus D'.
\]
The context functions encountered in this report are all on this form. For this reason, there is little need to make a distinction between contexts defined for single elements and contexts defined for subsets.

Note that contexts can be seen as a generalization of the concept of reference data. For instance, the \emph{trivial context}, given by, for $d \in D$; $C(d) = D \setminus d$, corresponds to unsupervised anomaly detection. It is obtained when the scope of any local context grows large enough.

Figure~\ref{fig:contexts} shows a schematic view of a data set, along with three contexts. The leftmost panel illustrates the trivial context of a single element, the middle panel illustrates a local context of a single element, and the rightmost panel illustrates the natural extension of this local context to subsets.

\section{Anomaly types}
\label{sect:anomaly_types}

\begin{figure}[htb]
    \begin{center}
        \includegraphics[width=1\textwidth]{resources/filters}
    \end{center}
    \caption{{\small Schematic illustration of filters and contexts acting on an evaluation sequence $S = (s_1, s_2, \dots, s_{40})$. The top panel shows the evaluation set $E = \mathcal{F}(S) = \{e_1, e_2, \dots, e_{19}\}$ extracted by a sliding window filter with width $4$ and step $2$. The bottom panel shows the local symmetric context of $e_{10}$ with width $w = 12$: $C(e_{10}) = \{c_1, c_2, \dots, c_{24}\}$, as well as the reference data set $R_{e_{10}} = \mathcal{F}_R(e_{10}) = \{r_1, r_2, \dots, r_{10}\}$ extracted by an analogous sliding window reference filter.}}
\label{fig:filters}
\end{figure}

An important aspect of any problem is which subsets of the data set $D$ to consider when looking for anomalies; i.e.\ which subsets of $D$ should constitute the \emph{evaluation set} $E$. If all subsets are considered, the size $|E|$ of $E$ is $2^{|D|}$. This number is obviously too large to handle effectively, and the evaluation set must somehow be limited.

Fortunately, only a small fraction of all possible subsets is typically of interest in any given application. Precisely which subsets are interesting depends on the structure of $D = \{d_1, d_2, \dots, d_n\}$. If $D$ lacks additional structure (such as an ordering or metric) inducing a concept of locality, then it is reasonable to consider only the singleton sets, i.e. $E = \{\{d_i\} | d_i \in D\}$. When such additional structure exists (and is pertinent to the analysis), it is reasonable to let $E$ consist of subsets in which all elements are `close' (with regards to this additional structure).

As an example, consider a sequence $S = (s_1, s_2, \dots, s_n)$. As mentioned in the previous section, a locality concept is naturally induced by the sequence ordering, and it reasonable to let $E$ consist of contiguous subsequences of $S$:
\[
    E = \{(s_{a_1}, s_{a_1 + 1}, \dots , s_{b_1}) , (s_{a_2}, s_{a_2 +1}, \dots, s_{b_2}), \dots, (s_{a_k}, s_{a_k+1}, \dots, s_{b_k})\}.
\]
For such $E$, it is the case that $|E| \in O(|D|^2)$. Furthermore, it is often the case that not all contiguous subsequences must be evaluated---for instance it may suffice to treat only subsequences of some specific length, leading to $|E| \in O(|D|)$. Finally, if the ordering is not relevant to the analysis, then $E$ should be the singleton sets of $S$, and $|E| = |D|$. Thus, placing reasonable restrictions (based on the structure of the data set) on $E$ can render the analysis much more manageable.

To facilitate the construction of $E$, the concept of $\emph{filters}$ can be used. An \emph{evaluation filter} is a function $\mathcal{F}_E(D): D \mapsto E \subset 2^D$ that constructs the evaluation set. One evaluation filter for sequences used extensively in this report is the \emph{sliding window filter}:
\[
    \mathcal{F}_E(S) = \{(s_1, s_2, \dots, s_w), (s_{s+1}, s_{s+2}, \dots, s_{s+w}), \dots, (s_{n-w}, s_{n-w+1}, \dots, c_{n})\}\footnote{It is here assumed that $ (s+w) | n$. If this is not the case, the last element extracted might be a bit different.}
\]
with width $w$ and step $s$. This filter is the most reasonable choice for sequences when all items in $E$ must be of the same length (as is typically the case).

Of course, it is often reasonable to, in addition to the evaluation set $E$, also construct a reference set, with regards to which to compare the elements in $E$. Naturally, with each element $e_i \in E$ should be associated one such set $R_{e_i}$, consisting of subsets of the context $C(e_i)$. Analogously with evaluation filters, \emph{reference filters} can be introduced, which simplify the construction of such $R_{e_i}$. Since the context is a set of sets, these should have the form $\mathcal{F}(e_i): C(e_i) \mapsto R_{e_i} \subset 2^D$. As an example of a reference filter, consider the sliding window reference filter for sequences with length $w$ and step $s$:
\[
    \mathcal{F}_R(e_i) = \bigcup_{(c_1, c_2, \dots, c_n) \in C(e_i)}\{(c_1, c_2, \dots, c_w), (c_{s+1}, c_{s+2}, \dots, c_{s+w}), \dots, (c_{n-w}, c_{n-w+1}, \dots, c_{n})\}.
\]

A schematic illustration of the operation on a sequence of sliding window filters and a local context is shown in Figure~\ref{fig:filters}. Here, an evaluation set consisting of $19$ subsequences of a sequence of length $40$ is constructed. With each element $e_i \in E$ is associated a reference set $R_{e_i}$ (as is seen in the figure, $R_{e_{10}} = 10$). An anomaly detection algorithm could compare each of the $e_i$ to the corresponding $R_{e_i}$ in turn in order to detect contextual collective anomalies (defined below).

\begin{figure}[htb]
    \begin{center}
        \includegraphics[width=1\textwidth]{resources/types_of_anomalies}
    \end{center}
    \caption{{\small Different types of anomalies in a real-valued continuous sequence. In the middle of each series is an aberration---shaded black---corresponding to a specific type of anomaly. Appropriate contexts for these anomalies are shaded dark grey, while items not part of the contexts are shaded light grey. The top panel contains a point anomaly---a point anomalous with regard to all other points in the series. The second panel contains a contextual anomaly---a point anomalous with regard to its context (in this case, the few points preceding and succeeding it), but not necessarily to the entire series. The third panel contains a collective anomaly---a subsequence anomalous with regard to the rest of the time series. The fourth contains a contextual collective anomaly---a subsequence anomalous with regard to its context.}}
\label{fig:anomaly_types}
\end{figure}

To simplify the discussion of contexts and evaluation/reference sets, it is helpful to introduce a few different \emph{anomaly types}, based on which appropriate choices of evaluation and reference filters can be inferred. For the purposes of the discussion in this report, an introduction of four anomaly types will suffice. In order of increasing generality, these are \emph{point anomalies}, \emph{contextual point anomalies}, \emph{collective anomalies}, and \emph{contextual collective anomalies}. An illustration of these anomaly types in the context of real-valued sequences is shown in Figure~\ref{fig:anomaly_types}.

\emph{Point anomalies} are arguably the simplest out of these anomaly types. They correspond to single points in the data set (i.e.\ $E$ consists of the singleton sets of $D$) that are considered anomalous with regard to the entire reference set (i.e.\ a trivial context is appropriate). Finding them can be formulated as the following task:

\begin{task}[Point anomaly detection]
\label{task:point}
  Given a data set $D$, detect individual $d_i \in D$ that are anomalous with regard to the entire reference set.
\end{task}

Point anomalies are often referred to as \emph{outliers} and arise in many domains~\cite{eskin}. Their detection is often relatively straightforward. Statistical methods have been shown to be well suited for handling point anomalies, and are usually sufficient.

Of course, point anomalies are not often not sufficient to describe all anomalies when $D$ admits a concept of locality. In this case, \emph{contextual point anomalies} can capture a more general class of anomalies. Contextual anomalies are individual items that are anomalous with regards to their context (as given by some non-trivial context function); i.e.\ while they might seem normal when compared with all elements in the reference data, they are anomalous when compared to the other items in their context. Detecting contextual point anomalies can be formulated as the following task:

\begin{task}[Contextual anomaly detection]
\label{task:contextual}
  Given a set $D$ and a non-trivial context function $C(d)$, detect elements $d \in D$ anomalous with regards to $C(d)$.
\end{task}

Furthermore, detecting individual anomalous points $d \in D$ might not always suffice (specifically, when the anomalies are continuous), and \emph{collective anomalies} might be required to capture relevant phenomena. These correspond to contiguous sets of non-anomalous points that, when taken as a whole, are anomalous with regards to the entire reference set (i.e.\ a trivial context is appropriate). The task of detecting such anomalies can be formulated using filters:

\begin{task}[Collective anomaly detection]
\label{task:collective}
  Given a set $D$ and a (non-singleton) filter $\mathcal{F}$, detect point anomalies in the evaluation set $\mathcal{F}(D)$.
\end{task}

Finally, \emph{contextual collective anomalies} are the most general class of anomalies, and correspond to contiguous sets of non-anomalous points that are anomalous with regard to a specific context but not to the entire reference set.

\begin{task}[Contextual collective anomaly detection]
\label{task:contextual_power}
  Given a data set $D$, a (non-singular) filter $\mathcal{F}$ and a non-trivial context function $C$, detect elements $e_i$ in the evaluation set $\mathcal{F}(D)$ that are anomalous with regards to the context $C(e_i)$.
\end{task}

An illustration of the above concepts in real-valued sequences is shown in Figure~\ref{fig:anomaly_types}. Assuming that unsupervised anomaly detection is used, Task~\ref{task:point} amounts to disregarding the information provided by the ordering and detecting only `rare' items. While the task can capture the aberration in the first sequence in Figure~\ref{fig:anomaly_types}, none of the aberrations in the other sequences would be considered point anomalies.

While the value at the aberrant point at the center of the second sequence occurs elsewhere in that sequence, it is anomalous with regards to its local context, and as such, should be considered a contextual point anomaly and can be captured by Task~\ref{task:contextual}.

Since the third time series is continuous, the aberration present at its center can not be captured by Tasks~\ref{task:point} Or~\ref{task:contextual}. It is, however, a collective anomaly, and can be accurately captured by Task~\ref{task:collective}.

Finally, neither of the Tasks~\ref{task:point},~\ref{task:contextual} or~\ref{task:collective} captures the aberration in the fourth sequence, as it is both continuous and occurs elsewhere in the sequence. However, with an appropriate choice of (local) context, it can be deemed a contextual collective anomaly, and can be captured by Task~\ref{task:contextual_power}.

It should be noted that while all of the above tasks are special cases of contextual collective anomaly detection, it is often possible to reduce each of the tasks to detection of point anomalies, as well. Task~\ref{task:collective} reduces to detection of point anomalies by its definition (i.e.\ it corresponds to detecting point anomalies in a higher-dimensional space), while data normalization can sometimes be utilized to perform this reduction for contextual anomaly detection (see~\cite{meckesheimer} for instance).

\section{Anomaly measures}
\label{sect:anomaly_measures}

Arguably the most significant aspect of an anomaly detection method is what heuristic is used to decide if items are anomalous or not. This choice defines (often in unpredictable ways) what types of features will be considered anomalous, so it is vital to make an appropriate choice.

Many heuristics have been proposed in the literature, with varying degrees of justification and success. No exhaustive presentation of these is given here; instead a selection of some of the more common approaches is presented. We begin by describing statistical and information theoretic approaches---two areas which provide theoretic justifications for what should be considered anomalous. We then present a few other approaches taken from traditional machine learning.

Statistical measures usually address the following task:
\begin{task}[Statistical anomaly detection]
\label{task:statistical}
  Given a data set $D$, find items $d \in D$ that are unlikely to have been generated by the same distribution as the rest of $D$ (or some reference set $R$).
\end{task}
In general, statistical measures work by estimating a statistical distribution underlying the data, and then labeling data points based on how likely they are to have been generated by this distribution. They have been applied to a wide range of domains, often with good results. Several books and surveys have been published on statistical anomaly detection, including~\cite{barnett},~\cite{bakar},~\cite{leroy} and~\cite{hawkins}.

Statistical measures are usually classified based on whether the distribution is known in advance (with unknown parameters) or not. We formulate these two cases as the following two tasks:
\begin{task}[Parametric statistical anomaly detection]
  Given a data set $D$, generated from the distribution $\mathcal{D}(\theta)$ with the unknown parameter $\theta$, estimate $\theta$ based on $D$ (or some reference data set $R$), and find items $d \in D$ that are unlikely to have been generated by $\mathcal{D}(\theta)$.
\end{task}
\begin{task}[Non-parametric statistical anomaly detection]
  Given a data set $D$, estimate the distribution $\mathcal{D}$ of $D$ (or some reference data set) from a set of basis functions and find items $d \in D$ that are unlikely to have been generated by $\mathcal{D}$.
\end{task}
While non-parametric approaches are more widely applicable (the distribution of data is usually not known), the extra information provided to parametric methods mean that they converge faster and are more accurate. Both these statistical tasks suffer from scalability issues. Furthermore, it can be difficult to modify statistical methods to take context into account.

A relatively novel and interesting approach to anomaly measures is \emph{information theoretic measures}. Mainly used for symbolic data sets, these measures judge similarity by estimating how much information is shared between items or subsets of items (i.e.\ by computing measures of shared information between elements). Like statistics, information theory can be used to give a theoretical justification for what is considered anomalous.

Several different measures of shared information have been implemented, such as the compressive-based dissimilarity measure (CDM)~\cite{keogh2} and (relative) conditional entropy~\cite{xiang}. While information theoretic approaches are mainly useful for discrete data, they have shown promise for describing anomalies in continuous data when combined with a discretization transformation~\cite{keogh2}. Information theoretical anomaly detection can be summarized as the following task:

\begin{task}[Information theoretic anomaly detection]
  Given a data set $D$, find items in $D$ that share little information with the rest of $D$ (or some reference data set).
\end{task}

Anomaly measures inspired by traditional machine learning are also common and have been extensively researched in various contexts. We here present two tasks, using classifier-based and point-based measures, which together cover a majority of the methods proposed in the literature. We begin by introducing classifier-based measures:

\begin{task}[Classifier-based anomaly detection]
  Given a data set $D$, train a two-class classifier and use this classifier to determine whether or not the elements in $D$ are anomalous.
\end{task}

While classifiers are commonly used with Task~\ref{task:anomalous_set}, weighing schemes can be utilized to produce other forms of output. Depending on the type of training data, different types of classifiers must be used. Semi-supervised problems require one-class classifiers, while unsupervised methods require unsupervised classifiers. Furthermore, classifiers that do not support iterative addition and removal of elements can not be used for unsupervised anomaly detection.

Another common approach is distance-based heuristics:
\begin{task}[Distance-based anomaly detection]
  Given a data set $D \subset X$ for some set $X$ and a distance measure $\delta(d_i, d_j): X \times X \rightarrow \mathbb{R}$, determine if the $d_i \in D$ are anomalous or not based on $\delta(d_i, d_j)$ for $d_j \in D, i \neq j$.
\end{task}
Obviously, such measures are not always applicable, since they require $X$ to allow some form of useful distance function. Derived problem formulations often base their anomaly measures on the local point density of or the distances to the few nearest points in the training set around evaluated points. Such methods include \emph{k-nearest neighbors} (kNN) and, more generally, \emph{variable kernel density estimation}.

This task is very flexible, since any distance measure be used, and the distance values so obtained can be utilized in many ways. It is also well suited for both semi-supervised and unsupervised anomaly detection.

In some situations, it is useful to build a predictive model:
\begin{task}[Detecting anomalies based on a predictive model]
\label{task:predictive_model}
  Given a sequence $S$, build a model that predicts the next state based on the previous states. Determine if each $s_i \in S$ is an anomaly or not based on how much it diverges from the value predicted by the model.
\end{task}
Models that have been studied include Markov chains, hidden Markov models, autoregressive models, as well as several others.

While the above tasks cover a majority of methods in the literature, several heuristics not covered by them have been proposed. Furthermore, there is often overlap between these tasks. For instance, most predictive models used in Task~\ref{task:predictive_model} are based on statistical measures, and problems using such models could just as well be said to target Task~\ref{task:statistical}

\section{Output format}
\label{sect:output_format}

Finally, the format of output generated by methods must be specified. The choice of this factor depends on the target presentation format and performance requirements. We here briefly present three common output formats.

\begin{task}[Presenting anomaly scores]
\label{task:anomaly_scores}
  Given a data set $D$, associate with each $d_i \in D$ a score $a_i \in \mathbb{R}$ based on how anomalous it is.
\end{task}
This is the most general of the commonly used output formats. While the anomaly scores for non-anomalies are usually not interesting, this task is still useful, especially for visualization purposes.

In many applications, relative anomaly rankings are not essential, and a list of anomalous elements might suffice. In such cases, the following task can be useful:
\begin{task}[Producing a set of anomalous elements]
\label{task:anomalous_set}
  Given a data set $D$, construct a set $D' \subset D$ containing the anomalous elements of $D$.
\end{task}

One task that has received a lot of attention in recent years is \emph{discord detection}:
\begin{task}[Discord detection]
\label{task:discord}
  Given a data set $D$, detect and present the $k$ most anomalous items $d \in D$.
\end{task}
First introduced in~\cite{keogh1}, discord detection is especially interesting, since discords are less computationally intensive to produce than other forms of output. Despite its limitation of only producing a few anomalous elements, the task usually provides sufficient output, since analysts are often only interested in finding the few most anomalous items. Discord detection has received a lot of attention in recent years (see~\cite{keogh1},~\cite{bu},~\cite{yankov},~\cite{fu},~\cite{lin}).

Note that Tasks~\ref{task:anomalous_set} and~\ref{task:discord} can be reduced to Task~\ref{task:anomaly_scores} by selecting only items with scores above some threshold $t \in \mathbb{R}$ (assuming unique anomaly scores).

\section{Related high-level tasks}
\label{sect:related_tasks}

Anomaly detection is closely related to several other problems in data mining, machine learning and signal processing. We here briefly present a few high-level tasks, analogous to Task~\ref{task:anomaly_detection}, for a few such problems.

There are multiple related problems in signal processing that involve the detection and removal of anomalies in data. Typically referred to as \emph{noise removal}, \emph{data cleaning}~\cite{meckesheimer}, or \emph{signal recovery}, these problems are summarized in the following task:
\begin{task}[Signal detection]
  Given a data set $D$ consisting of a relevant signal with added noise, remove all anomalies caused by the noise.
\end{task}

\emph{Novelty detection}~\cite{chandola} refers to the detection of novel, or previously unseen, items or subsequences in a sequence~\footnote{It should be noted that the term `novelty detection' is occasionally used in the literature to refer to semi-supervised anomaly detection.}, as formalized by the following task:
\begin{task}[Novelty detection]
\label{task:novelty_detection}
  Given a sequence $S$, detect subsequences $(s_i, s_{i+1}, \dots, s_j) \subset S$ that are anomalous with regards to earlier subsequences (i.e. $(s_i, s_{i+1}, \dots, s_j) \subset S$ for $k < i$ and $l < j$).
\end{task}
Most of the high-level tasks presented in this chapter can be combined with novelty detection, which is reasonable considering that it can be reduced to Task~\ref{task:contextual_power} through the choice of an appropriate context function. It is especially interesting when used in monitoring applications and in the context of long sequences.

\emph{Change detection} (or \emph{event detection}) is the problem of detecting points at which sequences change in some way.
\begin{task}[Change detection]
      Given a sequence $S$, detect points at which the underlying distribution or model generating the $s_i$ changes.
\end{task}
Change detection is almost exclusively associated with time series, and has been used mainly in the study of climate data~\cite{gopala} and image analysis~\cite{radke}.

Other tasks that are often interesting in the same contexts as anomaly detection include \emph{clustering}~\cite{clustering} and \emph{motif discovery}~\cite{motif}~\cite{motif2}.


\chapter{An application to sequences}
\label{ch:time_series}

Recall that the optimisation problem is stated as
\[
    P^*_{opt} = \argmin_{P \in \mathcal{P}^*} \sum_{T_i \in \mathcal{T}} \delta(s(T_i), O^*(P, T_i)),
\]
where the objects that need to defined are the input and solution formats $[D]$ and $[S]$, the problem set $\mathcal{P}^*$, the test data $\mathcal{T}$, the error function $\epsilon^*$, and the oracle $O$. The framework presented in the previous chapter makes the assumption that all relevant problems $P$ can be decomposed into a set of functions $P = (T_D, F_E, C, F_R, M, \Sigma, T_S)$.

In this chapter, the objects mentioned above are defined for the two most common tasks in anomaly detection in sequences. Furthermore, suitable choices of and restrictions on $T_D$, $F_E$, $C$, $F_R$, $M$, $\Sigma$ and $T_S$ for anomaly detection are discussed in depth.

From here on, a \emph{sequence} will be taken to mean any list ($[X]$ for some set $X$) for which the list order reflects the natural ordering of the elements. A \emph{time series} is defined to be any sequence in $[(\mathbb{R}^+, X)]$, where the elements are ordered such that their first component (the timestamp) is increasing.

In Section~\ref{sect:tasks}, the two anomaly detection tasks we will study are presented. Corresponding oracles are presented.

Section~\ref{sect:prev_research} contains a survey of previous research on anomaly detection in sequences. The components $T_D$, $F_E$, $C$, $F_R$, $M$, $\Sigma$, and $T_S$ are studied individually.

Finally, Section~\ref{sect:implementation} details an implementation of the optimisation problem using the objects from the previous sections.

\section{Tasks}
\label{sect:tasks}

Two main tasks can be distinguished in anomaly detection in sequences: \emph{finding anomalous sequences in a set of sequences}, and \emph{finding anomalous subsequences in a long sequence}~\cite{chandola}. The former task can be seen as the detection of point anomalies in an unstructured set of sequences, while the latter corresponds to finding contextual anomalies in a totally ordered set.

\subsection{Finding anomalous sequences}

The task of finding anomalous sequences in a set of sequences involves taking a list of similar sequences and producing a list of corresponding anomaly scores. The input elements are not related, i.e.\ the input data is unstructured. Thus, the task can be seen as one of detection of point anomalies in a collection of sequences. This task has been the subject of intense research. Thorough reviews are found in~\cite{chandola2} and~\cite{chandola3}.

For an example of this task, see figure \ref{fig:example1}. Here, dataset consists of a set of sequences of user commands extracted from a system log, and task corresponds to detecting individual anomalous sequences in this dataset. While sequences $\mathbf{S_1}$ through $\mathbf{S_4}$ are originate from ordinary user sessions, sequence $\mathbf{S_5}$ could indicate an attack. Accurately detecting such anomalous sequences is an important problem in computer security.

The input data has the format $[D]$, where $D$ is itself a set of sequences, i.e.\ $D = [X]$ for some set $X$. In the example above $X$ is a set of commands, but it could just as well be $\mahtbb{R}$ or any other set. The solution format is $[S]$, where either $S = \mathbb{R}^+$ or $S = \{0, 1\}$ depending on the application requirements.

Since the input data is unstructured, any transformation $T_D$ must produce lists with the same length as it is given. Correspondingly, we can let $S' = S$. This renders $T_S$ redundant, so it can be ignored.

Since the task deals with unstructured data, the components $F_E$, $F_R$, and $\Sigma$ can be ignored. An oracle can then be formulated as:

\begin{algorithmic}
    \Require{Some $X \in [D]$.}
    \State{$X' \gets T_D(X)$}
    \State{$A \gets []$ \Comment{initialize anomaly scores to empty list}}
    \For{$E \in X'$} \Comment{iterate over elements}
        \State{$append(A, M(E, C(X', E)))$ \Comment{compute and store anomaly scores}}
    \EndFor{}
    \State{\Return{$A$ \Comment{aggregate scores to form anomaly vector}}}
\end{algorithmic}

As per the discussion in the previous chapter, $C$ can only be one of two functions, corresponding to unsupervised and semi-supervised anomaly detection, respectively.

\subsection{Finding anomalous subsequences}

The task of finding anomalous subsequences of long sequences corresponds to finding anomalous contiguous sublists of the input data $[D]$. In contrast to the task of finding anomalous sequences, the input data is structured, and the sequence ordering naturally gives rise to concepts of proximity and context. This task is relatively poorly understood, but is highly relevant in many application domains. As a consequence, automated methods can be expected to be very useful for this task. Essentially any monitoring or diagnosis application could benefit from a better understanding of the task.

For examples of sequences to which this task might be applied, see Figures~\ref{fig:example_2} and~\ref{fig:anomaly types}. These are all real-valued sequences which contain anomalous items or subsequences.

As with the previous task, either $S = \mathbb{R}^+$ or $S = \{0, 1\}$ depending on the application. However, it here makes sense to allow $T_D$ to compress the data (i.e.\ return a shorter list than it is given). Correspondingly, a corresponding $T_S$ is required in order to transform the preliminary solution (in $[S']$) to a list of anomaly scores with the same length as the input data.

Since all components must be used for this task, the oracle is identical to the one presented in Section~\ref{sect:oracle}.

\section{Components}
\label{sect:prev_research}

Anomaly detection in sequence is an important and active area of research, and plenty of problems related to anomaly detection in sequences have been studied over the years. In this section, a selection of previously researched problems are presented, arranged according to the framework presented in the previous chapter.

\subsection{The input data format $\mathcal{D}$}

Categorical, discrete, and real-valued sequences have all been extensively studied. Categorical sequences arise naturally in applications such as bioinfomatics~\cite{TODO} and intrusion detection~\cite{TODO}. Discrete sequences are typically encountered when monitoring the frequency of events over time. Finally, real-valued sequences are encountered in any application that involves measuring physical phenomena (such as audio, video and other sensor-based applications).

\subsection{The transformations $T_D$ and $T_S$}

\begin{figure}[htb]
  \begin{center}
    \leavevmode
    \includegraphics[width=\textwidth]{resources/types_of_data}
  \end{center}
  \caption{\small{Illustration of numerosity and dimensionality reduction in a conversion of a real-valued sequence to a symbolic sequence. The top frame shows a real-valued sequence sampled from a random walk. The second frame shows the resulting series after a (piecewise constant) dimensionality reduction has been performed. In the third frame, the series from the second frame has been numerosity-reduced through rounding. The bottom frame shows how a conversion to a symbolic sequence might work; the elements from the third series is mapped to the set $\{a,b,c,d,e,f\}$.}}
\label{fig:types_of_data}
\end{figure}

Transformations are commonly used to facilitate the analysis of sequences, and a large number of different such transformations are found in the literature.

Feature extraction is commonly performed to reduce the dimensionality of sequences, and especially of real-valued ones. Essentially, the task of feature extraction in real-valued sequences corresponds to, given a sequence $s = [s_1, s_2, \dots, s_n]$, finding a collection of basis functions $[\phi_1, \phi_2, \dots, \phi_m]$ where $m < n$ that $s$ can be projected onto, such that $s$ can be recovered with little error. Many different methods for obtaining such bases have been proposed, including the discrete Fourier transform~\cite{faloutsos1}, discrete wavelet transforms~\cite{pong}~\cite{fu}, various piecewise linear and piecewise constant functions~\cite{keogh3}~\cite{geurts}, and singular value decomposition~\cite{keogh3}. An overview of different representations is provided in~\cite{fabian}.

Arguably the simplest of these bases are piecewise constant functions $[\phi_1, \phi_2, \dots, \phi_n]$:
\[
  \phi_i(t) = \left\{
    \begin{array}{l l}
      1 & \quad \text{ if } \tau_i < t < \tau_{i+1} \\
      0 & \quad \text{ otherwise.} \\
    \end{array} \right.
\]
where $(\tau_1, \tau_2, \dots \tau_n)$ is a partition of $[t_1, t_n]$.

Different piecewise constant representations have been proposed, corresponding to different partitions. The simplest of these, corresponding to a partition with constant $\tau_{i+1} - \tau_i$ is proposed in~\cite{keogh4} and~\cite{faloutsos2} and is usually referred to as \emph{piecewise aggregate approximation (PAA)}. As shown in~\cite{keogh5},~\cite{keogh3} and~\cite{faloutsos2}, PAA rivals the more sophisticated representations listed above.

Numerosity reduction is also commonly utilised in analysis of real-valued sequences. One scheme that combines numerosity and dimensionality reduction in order to give real-valued sequences into a categorical representation is \emph{symbolic aggregate approximation} (SAX)~\cite{sax}. This representation has been used to apply categorical anomaly measures to real-valued data with good results. A simplified variant of SAX is demonstraded in figure~\ref{fig:types_of_data}.

In general, real-valued sequences are much easier to deal with than time series. For this reason, irregular time series are commonly transformed to form regular time series, which can be treated as sequences. Formally, such transformations map a sequence in $[(\mathbb{R}^+, X)]$ to a sequence in $[X]$.

The simplest such transformation involves simply dropping the timestamp component of each item. This is useful when the order of items is important, but how far apart they are in time is not. This is often the case when dealing with categorical sequences. An example of such an application is shown in figure~\ref{fig:example1}.

Another common class of transformations involves estimating the (weighted) frequency of events. This is useful in many scenarios, especially in applications involving machine-generated data.

Several methods can be used to generate sequences appropriate for this task from time series, such as histograms, sliding averages, etc. These can be generalised as the following transformation:

Given a time series $[(t_1, x_1), (t_2, x_2), \dots, (t_n, x_n)]$ in $[(\mathbb{R}^+, X)]$, with associated weights $w_i$ and some envelope function $e(s, t): X \times \mathbb{R} \rightarrow X$, as well as a spacing and offset $\Delta, t_0 \in \mathbb{R}^+$, a sequence $[(s_{1}^{'}, \tau_1), (s_{2}^{'}, \tau_2), \dots]$ is constructed where $\tau_i = t_0 + \Delta \cdot i$ and $s_{i}^{'} = \sum_{(s_j, t_j) \in S} s_i w_i e(t_j - \tau_i)$.

The $\tau_i$ can then be discarded and the time series treated as a sequence\footnote{Note that this method requires multiplication and addition to be defined for $X$, and is thus not applicable to most symbolic/categorical data. Also note that $\mathbf{s}'$ is really just a sequence of samples of the convolution $f_S \ast e$ where $f_S = \sum_i \delta(t_i) s_i w_i$.}. Histograms are recovered if $e(s, t) = 1$ when $|t| < \Delta/2$ and $e(x, t) = 0$ otherwise.

How this aggregation is performed has a large and often poorly understood impact on the resulting sequence. As an example, when constructing histograms, the bin width and offset have implications for the speed and accuracy of the analysis. A small bin width leads to both small features and noise being more pronounced, while a large bin width might obscure smaller features. Similarly, the offset can greatly affect the appearance of the histograms, especially if the bin width is large. There is no `optimal' way to select these parameters, and various rules of thumb are typically used~\cite{density_estimation}.

Furthermore, noisy data is often resampled to form regular time series. In this case, any of a number of resampling methods from the digital signal processing literature~\cite{TODO} may be employed.

One commonly used transformation for real-valued data is the Z-normalization transform, which modifies a sequence to exhibit zero empirical mean and unit variance.\footnote{It has been argued that comparing time series is meaningless unless the Z-normalization transform is used~\cite{keogh5}. However, this is doubtful, as the transform masks sequences that are anomalous because they are displaced or scaled relative to other sequences.}

Transformations that transform the data into some alternative domain can also be useful. For example, transformations based on the \emph{discrete Fourier transform} (DFT) and \emph{discrete wavelet transform} (DWT)~\cite{fu} have shown promise. The DFT is parameter-free, while the DWT can be said to be parametrised due to the variety of possible wavelet transforms.

\subsection{The filters $F_E$ and $F_R$}

As was previously mentioned, filters are only interesting in the context of finding anomalous subsequences. Here, the role of the filter is to map a sequence in $[D']$ to a list candidate anomalies (subsequences of the input sequence).

By far the most frequently used filters are \emph{sliding window} filters. These map a sequence $X = [x_1, x_2, \dots, x_n]$ to
\[
    F_E(X) = [[x_1, x_2, \dots, x_w], [x_{k + 1}, x_{k + 2}, \dots, x_{k + w}], \dots, [x_{n - w}, x_{n - w + 1}, \dots, x_n]],
\]
where $w$ and $k$ are arbitrary integers (typically $k \leq w$)\footnote{We here assume that $k | n - w$. Otherwise, the last element above might look a bit different.}.

\subsection{The context function $C$}

The ordering present in sequences naturally give rise to a few interesting contexts, which are now demonstrated for a sequence $s = [s_1, s_2, \dots, s_n]$ and a candidate anomaly $s' [s_i, s_{i + 1}, \dots, s_j]$, where $1 \leq i \leq j \leq n$. It is here assumed that all candidate anomalies are contiguous. As mentioned in the previous chapter, contexts can be used to generalise the concept of training data. Semi-supervised anomaly detection corresponds to the \emph{semi-supervised context} $C(s, s') = T$, where $T$ is some fixed set of training data.

Likewise, traditional unsupervised anomaly detection for subsequences can be formulated using the \emph{trivial context} $C(s, s') = [[s_1, s_2, \dots, s_{i - 1}], [s_{j + 1}, s_{j + 2}, \dots, s_n]]$. This corresponds to finding either point anomalies or collective anomalies in a sequence.

Another interesting context is the \emph{novelty context} $C(s, s') = [[s_1, s_2, \dots, s_{i - 1}]]$. This context captures the task of novelty detection in sequences, which has been researched in~\cite{TODO}.

Finally, a family of \emph{local contexts}
\[
    C(s, s') = [[s_{\max(1, i - a)}, s_{\max(2, i - a + 1)}, \dots, s_{i-1}], [s_{j+1}, s_{j+2}, \ldots s_{min(n, j+b)}]]
\]
may be defined for $a, b \in \mathbb{N}$, in order to handle anomalies such as the one in the last sequence of figure~\ref{fig:anomaly_types}.

\subsection{The anomaly measure $M$}

% A more useful way of restricting this factor is to simply consider which of the anomaly measures studied in the literature are applicable to the target application, and only allow problem formulations which involve these. For this reason, we here settle for a broad overview of commonly studied anomaly measures. We begin by discussing statistical measures, which are especially interesting since they can be theoretically justified.
%
% Statistical measures usually operate under the assumption that $C(e_i)$ has been generated from some underlying distribution or stochastic process, and associates an anomaly score with $e_i$ based on how likely it is to have been generated by the same distribution or process. Typically, statistical measures work by using some standard inference method, coupled with a few assumptions about the dataset, to estimate some simple distribution underlying the $C(e_i)$. Statistical measures have been applied to a wide range of domains, often with good results. Several books and surveys have been published on the subject of anomaly detection using statistical methods~\cite{barnett}~\cite{bakar}~\cite{leroy}~\cite{hawkins}.
%
% Statistical measures are usually classified as either parametric or non-parametric. \emph{Parametric statistical measures} assume that distribution underlying $C(e_i)$ is known, but has unknown parameters (for instance, it might be assumed that the data is $N(\mu, \sigma^2)$, where $\mu$ and $\sigma$ are unknown). \emph{Non-parametric statistical measures}, on the other hand, do not assume that the distribution is known and instead try to estimate the distribution itself by assigning weights to a set of basis functions.
%
% While non-parametric approaches are more widely applicable (the distribution of data is usually not known), the extra information provided to parametric methods mean that they converge faster and are more accurate (as long as the given assumptions are correct). Of course, parametric methods are also less widely applicable, since the underlying distribution is often not known.
%
% For datasets that can be modeled by stochastic processes, \emph{predictive models}, such as Markov chains~\cite{TODO}, hidden Markov models~\cite{TODO}, and autoregressive models~\cite{TODO} are frequently used as anomaly measures. It should be noted that most predictive models presuppose an ordering and a one-sided context.
%
% Due to the relatively high computational cost of density estimation, statistical methods are mainly used to find point anomalies. Since contextual anomalies require different training sets for each $e_i \in E$, detecting contextual anomalies requires $|E|$ density estimations (unless some clever optimisation is employed), which is typically prohibitively expensive. Since most density estimation methods scale poorly with increasing dimensionality, collective anomalies can also be prohibitively expensive to detect using statistical methods.
%
% A relatively novel and interesting class of anomaly measures is \emph{information theoretic measures}. Mainly used for symbolic datasets, these measures judge similarity by estimating how much information is shared between items or subsets of items (i.e.\ by computing measures of shared information between elements). Like statistics, information theory can be given a convenient theoretical justification.
%
% Several different measures of shared information have been suggested, such as the compressive-based dissimilarity measure (CDM)~\cite{keogh2} and (relative) conditional entropy~\cite{xiang}. While information theoretic approaches are mainly useful for symbolic data, they have shown promise for describing anomalies in continuous data when combined with a discretization and numerosity reduction~\cite{keogh2}.
%
% Anomaly measures inspired by traditional machine learning methods are also common and have been extensively researched in various contexts. For instance, classifier-based methods such as support vector machines are commonly used (TODO: add citation here). While classifiers only produce as many distinct outputs as there are classes, ensembles or weighting schemes can be utilized to produce finer grained output. Like statistical anomaly measures, classifier-based anomaly measures are relatively expensive to train, so they are typically not suitable for non-trivial contexts.
%
% Distance-based anomaly measures are also commonly used. These assign anomaly scores to elements by means of some local point density estimate. Examples include k-nearest neighbors (TODO: cite) and local outlier factor (TODO: cite). Distance-based typically measures scale well with increasing dimensionality, and are appropriate for non-trivial contexts since they are often simple to compute.

As is typically the case, the anomaly measure is the most important aspect of any anomaly detection problem for sequences. Formally, any anomaly measure that takes an evaluation vector and a set of reference vectors as inputs and returns a real value is a candidate for $\mathcal{M}$. We now discuss a few such anomaly measures, following the classification in Section~\ref{sect:anomaly_measures}.

As previously mentioned, \emph{statistical measures} are attractive due to the theoretical justification they provide for anomaly detection. However, there are certain factors which render their use problematic for general applications. To begin with, it can generally not be assumed that the data belongs to any particular distribution, and parametric statistical measures are only appropriate in specific circumstances. Nevertheless, parametric statistical methods in sequences are an active area of research~\cite{TODO}.

Non-parametric methods are more widely applicable.

Since few non-parametric methods for anomaly detection in sequential data can take into account either collective anomalies or context, and since naive approaches are likely to suffer from convergence issues,~\footnote{For instance, the expectation maximization algorithm for Gaussian mixture models has convergence issues in high dimensions with low sample sizes~\cite{TODO}.} suggesting appropriate non-parametric methods for Task~\ref{task:main} is difficult. However, in the case of Task~\ref{task:frequency}, point anomalies (for which statistical methods have been extensively researched) are more interesting than collective anomalies, and statistical methods are likely to be applicable.

\emph{Information theoretical measures} are especially interesting for anomaly detection in categorical sequences. However, most such methods are essentially distance-based anomaly measures equipped with information theoretical distance measures. For this reason, we do not cover information theoretical anomaly measures separately from distance-based measures.

\emph{Classifier-based measures} have also shown promise, especially for the task of finding anomalous sequences in a set of sequences~\cite{chandola3}. Generally, any one-class classifier is potentially suitable for the task; see~\cite{classification} for an exhaustive discussion of this topic. While one-class classifiers produce binary output, appropriate anomaly vectors can still be produced through a suitable weighting scheme.

\emph{Predictive model-based measures} are also potentially interesting, since are naturally well suited for dealing with the novelty context. However, existing predictive model-based approaches seem to be lacking for the task at hand. In~\cite{chandola3}, a leading model-based novelty detection method~\cite{perkins2} which uses an autoregressive model was shown to perform relatively poorly.

\emph{Distance-based measures} are especially interesting, due to their flexibility and scalability. A few kNN-based anomaly measures were shown to perform very well for detecting anomalous sequences in sets of sequences in~\cite{chandola3}.

When dealing with distance-based problems, the choice of distance measure has a profound impact on which anomalies are detected. As with other aspects of anomaly measures, however, drawing conclusions about method efficacy through theory alone is difficult; implementing, evaluating, and comparing several measures is likely to be more useful.

Possible interesting measures include the \emph{Euclidean distance} or the more general \emph{Minkowski distance}; measures focused on time series, such as \emph{dynamic time warping}~\cite{dtw}, \emph{autocorrelation measures}~\cite{autocorrelation}, or the \emph{Linear Predictive Coding cepstrum}~\cite{cepstrum}; or measures developed for other types of series (accessible through transforms), such as the \emph{compression-based dissimilarity measure}~\cite{keogh2}.

Additionally, the choice of distance measure affects how well methods can be optimized. Naive approaches to distance-based problems typically scale prohibitively slowly, and are not suitable for large amounts of data. Optimizations typically involve exploiting properties of the distance measure in order to reduce the number of distance computations (for instance, the commonly used k-d tree nearest neighbor algorithm requires the distance to be a Minkowski metric).

---

As mentioned previously, the anomaly measure is likely the most important component. It is also the component with the largest number of interesting choices. Properly defining the required parameters for all of the anomaly measures discussed in Section~\ref{sect:prev_research} is not possible within the scope of this report, so we only discuss the most interesting anomaly measures here as indicated in~\cite{chandola3}.

Distance-based anomaly measures, and especially kNN-based anomaly measures are among these. Essentially, the kNN anomaly measure takes two parameters: a distance measure (defined on the specific type of sequences under consideration) $\delta$, and a k-value $k \in \mathbb{N}$, and given a sequence $\mathbf{x}$ and a set of context sequences $\{\mathbf{x}_1, \mathbf{x}_2, \dots, \mathbf{x}_n\}$, computes the anomaly score by taking the average of the $k$ smallest $\delta(\mathbf{x}, \mathbf{x}_i)$.

Thus, the kNN anomaly measure has the two parameters $k$ and $\delta$, where $\delta$ may (for instance) be any of the distance measures discussed in Section~\ref{sect:prev_research}. Note that the distance measure, in turn, may be parametrised (for instance, the Minkowski measure has an order parameter $p \in \mathbb{R}^+$).

TODO: discuss parameters for distance measures?

Classifier-based anomaly measures are also interesting. A support vector machine-based anomaly measure is shown to perform especially well in~\cite{chandola3}. Support vector machine-based anomaly measures take a few parameters: a kernel, eventual kernel parameters, and a soft margin parameter $C$. For a more thorough discussion of support vector machines and their parameters, see~\cite{TODO}.

Note that further anomaly measures can be constructed by chopping up the input sequences $\mathbf{x}$ and $\{\mathbf{x}_1, \mathbf{x}_2, \dots, \mathbf{x}_n\}$ into smaller sequences using filters before applying the distance measure, and then aggregating the result into an anomaly score using some aggregation function. This is done for the support vector machine anomaly measure in~\ref{chandola3}.

TODO: maybe mention a few other anomaly measures

\subsection{The aggregation function $\Sigma$}

Examples of anomaly detection problems which involve aggregation are hard to find in the literature. For this reason, suggesting appropriate choices of $\Sigma$ is difficult. A few choices which are likely to produce good results are $\Sigma$ on the form suggested in Section~\ref{sect:aggregation_function}, with $\sigma$ that produce either the \emph{maximum}, \emph{minimum}, \emph{median}, or \emph{mean} of its input values.

\section{Implementation}
\label{ch:implementation}

As stated in the abstract, the development of a software framework for the evaluation of anomaly detection methods, called \texttt{ad-eval} and available at \url{http://github.com/aeriksson/ad-eval}, was a significant part of the project. In this chapter, the design, development process, and features of \texttt{ad-eval} are discussed.

Essentially, \texttt{ad-eval} consists of three separate parts: a library implementing the component framework, a comprehensive set of utilities for evaluating the performance of methods and problem, and an executable leveraging the library. The entire project is written in Python.

In this chapter, \texttt{ad-eval} is described in detail. The three parts are described in Sections~\ref{sect:implemented_problems},~\ref{sect:evaluation_package}, and~\ref{sect:executable}, and some of the design choices made in the development of \texttt{ad-eval} are discussed in Section~\ref{sect:design}.

\subsection{Implemented components}
\label{sect:implemented_problems}

The anomaly detection part of \texttt{ad-eval} (given the Python package name \texttt{anomaly\_detection}) is a faithful implementation of the component framework, including the algorithm proposed in Section~\ref{sect:framework}. In order to preserve the modular nature of this framework, the individual components are implemented as autonomous modules, described below.

As there are no real alternatives found in the literature, the sliding window filter is the only implemented evaluation filter. Since the optimal window width $w$ and step length $s$ depend on the application, both of these parameters were left to be user-specified.

Since new new context functions can be implemented relatively easily, and since they have a relatively major impact on the analysis, all previously discussed contexts (specifically, the asymmetric and symmetric local contexts, the novelty context, the trivial context, and the semi-supervised context) were implemented.

As reference filters, a sliding window filter and the identity filter $F_E(X)=X$ were implemented, the latter because it is a better fit for dimension-independent distance measures.

Due to the limited scope of the project, the only implemented anomaly measures (henceforth referred to as \emph{evaluators}) were a variant of k-Nearest Neighbors (kNN), in which the distance to the $k$'th nearest element is considered, and a one-class support vector machine (SVM). For the kNN evaluator, the Euclidean distance as well as the compression-based dissimilarity measure~\cite{keogh2} and dynamic time warping~\cite{dtw} distances were made available. Furthermore, the symbolic aggregate approximation (SAX) and discrete Fourier transform (DFT) transformations were added as an optional pre-anomaly measure transformations. All parameters of the evaluators, distances, and transformations were left for the user to specify.

Finally, the mean, median, maximum, and minimum aggregators were implemented.

\subsection{Evaluation utilities}
\label{sect:evaluation_package}

The set of possible interesting tests that could be run on problems derived from Task~\ref{task:main} is considerable. A large number of component combinations can be used; most components have many possible parameter values, and it is important to assess how these affect the results; and a large set of methods with various optimizations and approximations can be proposed. For all of these choices, it is important that accuracy and performance are properly evaluated.

However, the performance of methods is highly dependent on the characteristics of the datasets to which they are applied. As mentioned in Section~\ref{sect:evaluation_data}, there is no hope to exhaustively cover the space of possible evaluation sets. Instead, sample data from the target application domain must be obtained before any tests are performed. Furthermore, obtaining adequate labeled test data is often difficult, and artificial anomaly generation must be considered as an option.

With this in mind, it was decided that an evaluation framework should be added to \texttt{ad-eval} to help facilitate the implementation, standardization, and duplication of accuracy and performance evaluations. Due to the variety of interesting tests highlighted above, an approach focused on the provision of tools that assist in scripting custom tests was deemed preferable to one focused on the construction of a single configurable testing program. To this end, utilities were developed for:

\begin{itemize}
    \item Saving and loading time series to/from file, with or without reference anomaly vectors.
    \item Pseudorandomly selecting and manipularing subsequences of series (e.g.\ for adding anomalies).
    \item Facilitating the testing of large numbers of parameter values.
    \item Generating various types of artificial anomalies and superpositioning them onto sequences.
    \item Calculating the anomaly vector distance measures $\epsilon_E$, $\epsilon_{ES}$ and $\epsilon_{FS}$ discussed in Section~\ref{sect:evaluation_measures}.
    \item Facilitating the automated comparison of several problems and methods on individual datasets.
    \item Automating the collection of performance metrics.
    \item Reporting results.
    \item Generating various types of custom plots from results.
\end{itemize}

With these tools in place, it is simple to write scripts that, for instance, generate large amounts of similar series containing random anomalies and evaluate the performance of several problems on this data in various ways. 

The tools were included in \texttt{ad-eval} as a separate Python module (called \texttt{eval\_utils} and located in the \texttt{evaluation} directory of the repository). This module was used to perform all tests in the evaluation phase of this project.

\subsection{Executable}
\label{sect:executable}

To enable the stand-alone use of the anomaly\_detection package, an executable was added to the \texttt{ad-eval} repository (called \texttt{anomaly\_detector} and located in the \texttt{bin} directory of the repository). This executable is used through a command-line interface and a \texttt{key:value} style configuration file, can perform supervised or semi-supervised anomaly detection on sequences from files or standard input, and can use any of the components implemented in \texttt{anomaly\_detection}.

To avoid having to modify this program every time a component in \texttt{anomaly\_detection} was changed, the executable was made unaware of all internal details of that package. Consequently, a command-line interface capable of configuring the components could not be implemented; instead, a configuration file parser is used to read and pass the component configuration to \texttt{anomaly\_detection}.

\subsection{Design}
\label{sect:design}

The development of \texttt{ad-eval} began at the start of the project. Initially, development efforts focused on the implementation of a few optimized methods found in the literature, to produce a Splunk app. However, as the project progressed and the issues discussed in Section~\ref{sect:adb} (that most methods were targeted at subtly different tasks, and that due to lacking evaluations, assessing which methods are really the `best' is not possible), this approach was recognized as fruitless, and abandoned.

Development then shifted towards an implementation of the component framework, with the goals of maximizing the ease of implementing and evaluating large amounts of components.

Consequently, modularity was a major focus throughout the development process, achieved through various means. As mentioned previously, the individual components were separated into independednt modules. Additionally, the evaluation utilities and the executable were decoupled from the component framework implementation, interfacing with it through only two method calls. Finally, the decision was made to distribute the configuration of the component framework implementation, letting each component handle its own configuration and making the rest of the package configuration-agnostic.

It was a natural choice to write the entire implementation in Python, for several reasons. First, Python is well suited for small, flexible projects such as \texttt{ad-eval}, thanks to its simplicity and flexibility. Furthermore, a number of great libraries for data mining and machine learning exist for Python, which were used to accelerate the development. Finally, if \texttt{ad-eval} becomes adopted for real-world use, Python's good C integration could be leveraged to write optimized code. 

Finally, the evaluation utilities were designed with ease of use and flexibility in mind. For instance, while classes facilitating test data generation, evaluation, and reporting are provided, their use is optional. As a result, evaluation scripts could be short and simple---the scripts used in the next chapter are all 30 to 70 lines long---without sacrificing flexibility.

\chapter{Results}
\label{ch:results}

In this chapter, a few basic results obtained by analysing time series using ADRT are presented. Since appropriate data could not be obtained for this project, and in order to limit the scope of this report, a comprehensive analysis could not be performed. Instead, a preliminary, qualitative evaluation was performed, with the twofold goal of gaining some insight into the how component choices affect accuracy and performance, as well as demonstrating how ADRT can be used to simplify and standardize the process of evaluating anomaly detection methods.

Specifically, the analysis presented in this chapter concerns the task of finding anomalous subsequences in long real-valued sequences. This task was chosen since it is a very interesting task that has not been extensively researched, and which is complex enough to require the full oracle as presented in Section~\ref{sect:oracle}.

All graphs and data in this chapter were obtained through scripts written using ADRT. Since one of the main goals of the evaluation was to demonstrate how ADRT can be used to perform standardized evaluations, all scripts used to obtain the figures and results in this chapter are available in the ADRT source code repository. Modifying these scripts to use other datasets or to evaluate other components (such as the SVM anomaly measure implemented in ADRT) is trivial.

\section{Method}
As the main goal of the analysis was to demonstrate the utility of the framework in reasoning about and evaluating anomaly detection approaches, the focus of this chapter is on using the framework and ADRT to assess how problem accuracy and performance varies over the problem set.

Since an analysis of the entire problem set would be prohibitively expensive, this is instead achieved by means of parametrising a constrained problem set, fixing a point (in other words, a problem) in this set, and measuring accuracy and performance in a neighbourhood (as defined by the parametrisation) of this point.

Since both the specific task studied and approach taken are novel, the error measures $\epsilon_E$, $\epsilon_F$, and $\epsilon_N$ implemented in ADRT must themselves be empirically evaluated to verify that they accurately capture intuitive error notions. Were this not the case, using them could invalidate the entire analysis. For this reason, an analysis of their performance is presented prior to the problem set analysis.

Due to difficulties in obtaining appropriate real-world test data, and due to the fact that evaluating anomaly detection methods on artificial data is dubious at best, the decision was made to limit the analysis to a qualitative one. Instead, necessary tools are provided (in the form of ADRT scripts) to make it trivial to perform a thorough analysis once the appropriate data is available.

\begin{figure}[h]
   %  \vspace{-10pt}
    \begin{center}
        \includegraphics[trim = 10mm 0mm 5mm 0mm, clip, width=\textwidth]{resources/reference_sequence}
    \end{center}
   %  \vspace{-20pt}
    \caption{\small{The sequence $s^*$ and corresponding reference anomaly vector $a^*$.}}
\label{fig:reference_sequence}
   %  \vspace{-10pt}
\end{figure}

To emphasise this qualitative nature, the analysis in this chapter has been performed on a single artificial sequence, using scripts which can be run on bigger datasets without modification. The chosen sequence, which will be referred to as $s^*$, is shown in Figure~\ref{fig:reference_sequence}, along with the reference anomaly vector used.

\section{Problem set}
Evaluating the set of all problems solvable by ADRT would make for a rather lengthy and dull analysis. To avoid this situation, a few constraints were placed on the problem set. Specifically, $M$ was restricted to kNN-based anomaly measures (as defined in Section~\ref{sect:anomeasure}, and $C$ was restricted to local contexts (as defined in Section~\ref{sect:context}) with $a = b = m$.

Next, a parametrisation of this problem set was defined by parametrising each of the allowed components separately. The filters $F_E$ and $F_R$, being sliding window filters, are naturally parametrised by the step length $s$ and window length $w$. The context function $C$, has only a single parameter $m$ given the above restrictions. The kNN anomaly measures can be considered to be parametrised by the value of $k$ and the distance measure $\delta$. Specific $\delta$ can in turn be parametrised individually. Since only four distinct values of $\Sigma$ are available, it does not need to be parametrised.

Finally, a single point in the restricted problem set was selected, and performance and accuracy characteristics were assessed in the neighbourhood of this point. The specific point, which will henceforth be referred to as $s^*$, is given by $s = 1$, $w = 10$, $m = 400$, $k = 1$, $\delta$ is the standard Euclidean distance, and $\Sigma = \Sigma_{mean}$. This point was chosen by using ADRT to optimise for $\epsilon_B$ over the given problem set.

To simplify the following discussion, we now introduce some notation. We will denote our fixed point by $P^*$, and let $P^*_{\alpha_1, \alpha_2, \dots, \alpha_n}$ be the set of problems where $\alpha_1, \alpha_2, \dots, \alpha_n$ are allowed to take on any value, and all other parameters are given by $P^*$ (e.g., $P^*_k$ corresponds to the set of problems where all parameters other than $k$ agree with the $P^*$). We will further denote the set of corresponding solutions for $s^*$ by $A_{\alpha_1, \alpha_2, \dots, \alpha_n}$.

\section{Error measures}
\label{sect:error_measure_eval}

\begin{figure}
    \centering
    \subbottom{\includegraphics[width=0.49\linewidth]{resources/normalized_euclidean_distance_heat_map}}
    \subbottom{\includegraphics[width=0.49\linewidth]{resources/equal_support_distance_heat_map}}
    \subbottom{\includegraphics[width=0.49\linewidth]{resources/full_support_distance_heat_map}}
    \subbottom{\includegraphics[width=0.49\linewidth]{resources/best_support_distance_heat_map}}
    \caption{Heat maps showing the normalized values of $\epsilon(A(P_{k, w}^*, s^*), a^*)$ for the three errors and $k, w = 1,2,\dots,50$. Blue and red correspond to low and high error values, respectively.}
\label{fig:error_heat_maps}
\end{figure}

In Section~\ref{sect:error_measures}, three error measures for anomaly vectors were proposed: $\epsilon_N$, $\epsilon_E$, and $\epsilon_F$. In this section, an investigation into how well these error measures capture intuitive notions of error is presented.

This investigation was performed by computing and graphing the errors \\ $\epsilon(O^*(P_{k, w}^*, s^*), a^*)$ for $\epsilon = \epsilon_N, \epsilon_E$ and $\epsilon_F$ and $k, w \in \{0,2,\dots,50\}$. Heat maps of the resulting values are shown in Figure~\ref{fig:error_heat_maps}. A few of the $A_{k, w}$ which were given the lowest values by each of the error measures are shown in Figure~\ref{fig:n_best_anomaly_vectors}.

As shown in the heat maps, the three error measures give similar results, attaining minima and maxima in the same regions. Since $\epsilon_E$ and $\epsilon_F$ operate on binary strings and thus have discrete domains, they often assign identical errors to nearby points. This is the cause of the relatively jagged appearance in the plots of these errors compared to the smoother appearance of the $\epsilon_N$ plot.

\begin{figure}[ht]
   %  \vspace{-5pt}
    \begin{center}
        \includegraphics[width=\textwidth]{resources/n_best_anomaly_vectors}
    \end{center}
   %  \vspace{-20pt}
    \caption{\small{The $n$th best $A_{k, w}$ according to the three error measures for $n = 1, 10, 50$ and $100$.}}
\label{fig:n_best_anomaly_vectors}
   %  \vspace{-10pt}
\end{figure}

Figure~\ref{fig:n_best_anomaly_vectors} shows the anomaly vectors with the $n$th lowest errors for the three distance measures. All three measures give similar anomaly vectors for $n = 1$ and $n = 10$, with $\epsilon_N$ and $\epsilon_F$ giving the same anomaly vector in both cases. For $n = 50$ and $n = 100$, however, $\epsilon_F$ seems to prioritize smooth anomaly vectors, while the other two anomaly measures prioritize anomaly vectors with few false positives.

One interesting aspect evidenced in the heat map plot is that while $\epsilon_E$ and $\epsilon_F$ are both very large for $A_{k,w}$ with small $w$, this tendency is not shared by $\epsilon_N$. As seen in Section~\ref{sect:w}, anomalies significantly larger than $w$ will not be detected by kNN methods, which means that assigning a large value to these anomaly vectors is reasonable. Since the $\epsilon_N$ (unlike the other two error measures) gives equal weight to every component, it will assign relatively low values to anomaly vectors that only partially capture anomalies as long as most of their elements are close to zero.  Indeed, this is the case for the $A_{k,w}$ with small $w$ since these anomaly vectors are close to constant everywhere except for a few spikes.

\begin{figure}[ht]
   %  \vspace{-5pt}
    \begin{center}
        \includegraphics[width=\textwidth]{resources/euclidean_problem}
    \end{center}
   %  \vspace{-20pt}
    \caption{\small{A reference anomaly vector for a long sequence and two corresponding candidate anomaly vectors. The first candidate vector, while noisy, correctly marks the anomaly. The second candidate does not mark the anomaly and marks two false anomalies. $\epsilon_N$, $\epsilon_E$ and $\epsilon_F$ for the two sequences are $8.3$ and $2.2$; $0.010$ and $0.99$; and $0.0050$ and $0.99$, respectively.}}
\label{fig:euclidean_problem}
   %  \vspace{-10pt}
\end{figure}

As an illustration of the potential problems this could cause, see Figure~\ref{fig:euclidean_problem}, which shows one reference anomaly vector and two candidate anomaly vectors for a long sequence. Here, the first anomaly vector, while noisy, accurately captures the anomaly while the second not only misses the anomaly, but also introduces two false positives. While $\epsilon_F$ and $\epsilon_E$ are significantly smaller for the first candidate than for the second, the reverse is true for $\epsilon_N$. This problem is amplified as the sequence length grows. These results indicate that $\epsilon_N$ should be used with caution, and that since other two error measures are preferable since they were defined specifically to avoid problems such as this.

% \clearpage

\section{Parameter values}
\label{sect:params}

The accuracy and performance of problems around $P^*$ is now studied for the sequence $s^*$, by means of varying each of $k, \delta, t, w, s, m,$ and $\Sigma$ individually and looking at the resulting anomaly vectors and oracle execution times.

As mentioned previously, since only a small subset of the problem space is studied, and only a single sequence is used, no conclusions about the global characteristics of the problem space, or about how well the results might extend to other sequences can be drawn. Instead, the analysis in this Section should be considered a first step towards establishing a broader understanding of how problem accuracy can vary over the problem sets for distance-based problems, and as an introduction to ADRT, including some useful ways to explore performance and accuracy characteristics.

\subsection{The k value}

\begin{figure}
    \centering
    \subbottom{\includegraphics[trim=5mm 5mm 0mm 0mm, width=0.95\linewidth]{resources/k_heat_map}}
    \subbottom{\includegraphics[trim=0mm 0mm 2mm 0mm, width=0.49\linewidth]{resources/k_error}}
    \subbottom{\includegraphics[trim=2mm 0mm 0mm 0mm, width=0.49\linewidth]{resources/k_time}}
        \caption{Results for $P^*_k$, where $k = 1,2,\dots,100$. The top panel shows the anomaly vectors $A_k$ (red and blue indicate high and low anomaly scores, respectively). The anomaly vectors have been individually normalised to lie in the unit interval. In the upper right panel, the errors $\epsilon_N$, $\epsilon_E$ and $\epsilon_F$ are plotted as a function of k. The errors curves have been individually normalised. Finally, oracle computation times for a single execution are plotted as a function fo $k$ in the lower right panel.}
\label{fig:k_plot}
\end{figure}

It is important to study how the anomaly vectors vary with $k$; first, because the choice of $k$ is likely to have a large impact on the appearance of the anomaly vector, regardless of the dataset; and second, because the kNN anomaly measure only operates on a single $k$ value at a time, it is in a sense the simplest distance-based anomaly measure, and thus an ideal tool for better understanding how the choice of $k$ impacts the analysis. This understanding is crucial in effectively designing other types of distance-based anomaly measures.

In order to understand how the $k$ value affects the resulting anomaly vectors, the problems $P^*_k$ for $k = 1,2,\dots,100$ were solved for $s^*$. The results are shown in Figure~\ref{fig:k_plot}.

The smoothness with which the $A_k$ vary with $k$ indicate that using several nearby $k$ in distance-based anomaly measures is not likely to significantly improve accuracy. Furthermore, at least in this case, $k=1$ minimizes all three error measures, and there is no indication that considering additional $k$ might help. While higher $k$ do lead to other regions being marked as anomalous, these regions do not correspond to relevant features. Were this to hold in general, considering $k$ higher than $1$, or using linear combinations of several $k$ is not likely to lead to any significant increase in accuracy. It is highly doubtful, however, that this holds fin general.

Since the implemented kNN method operates by brute force, the entire reference set must be evaluated regardless of $k$, so the constant evaluation time exhibited in the bottom right of the figure is expected. For any distance measure that is also a metric---such as the Euclidean distance---more efficient methods exist.

\subsection{The distance function}
% \FloatBarrier{}

\begin{figure}[ht]
   %  \vspace{-20pt}
    \begin{center}
        \includegraphics[width=\textwidth]{resources/delta_heat_map}
    \end{center}
   %  \vspace{-20pt}
    \caption{\small{Heat maps showing $A_{k, \delta}$ for the Euclidean and DTW distances.}}
   %  \vspace{-10pt}
\label{fig:delta_heat_map}
\end{figure}

For obvious reasons, the choice of the distance function $\delta$ can have a great impact on the anomaly vectors when using distance-based methods. The distance measures implemented in ADRT are the Euclidean distance, the dynamic time warp (DTW) distance, and the compression-based dissimilarity measure (CDM). To investigate the relative performance of these, the anomaly vectors $A_{k, \delta}$ were examined. Note that since $A_{\delta}$ consists of only one value per distance measure, calculating only $A_\delta$ would have yielded insufficient data.

% The ADRT implementation of the CDM performed poorly. To begin with, it ran significantly slower than the other methods, rendering any comprehensive analysis impossible. Furthermore, it produced poor anomaly vectors. There are a few possible explanations for this. First, the z-normalization step of the SAX transformation (in which each extracted subsequence is given zero empirical mean and unit variance) leads to poor results on random data regardless of the distance measure. Second, the window width of $10$ used in the standard configuration means that the extracted sequences are short and can not be efficiently compressed, leading to a roughly constant distance value. While the CDM will likely perform better and with other parameters, it was decided that the CDM would not be investigated further due to its slowness.

\begin{wrapfigure}{r}{0.5\textwidth}
\changecaptionwidth
\captionwidth{0.5\textwidth}
\includegraphics[width=0.5\textwidth]{resources/delta_errors}
\caption{\small{Errors for $A_{k, \delta}$.}}
\label{fig:delta_errors}
\end{wrapfigure}

Since the CDM distance measure performed seemed to perform poorly both in terms of accuracy and performance around $P^*$, it is not included in this analysis. Most likely, this was caused by the elements of reference set being relatively small; letting $F_E$ be the identity transformation should give better results.

Instead, the focus was placed on comparing the Euclidean and DTW distances. Heat maps of the resulting anomaly vectors are shown in Figure~\ref{fig:delta_heat_map} and a plot of the corresponding errors is shown in Figure~\ref{fig:delta_errors}. As is seen in the heat maps, there is generally little difference between the outcomes of the two distance measures; the DTW distance gives slightly `cleaner' (i.e.\@, with non-anomalous regions closer to $0$) anomaly vectors for very low values of $k$, while the Euclidean distance assigns a slightly lower score to the false anomalies encountered at high values of $k$. While there are some differences in the obtained errors---the DTW distance gives a better normalized Euclidean error, while the Euclidean distance generally gives better values of the other two errors---the evaluation is not sufficient to draw any conclusions about the relative merits of the two measures.

However, the fact that the DTW distance does not perform worse than the Euclidean distance in this evaluation is interesting. Since the DTW was designed to recognize long, shifted but relatively similar continuous sequences, it might be expected to perform poorly on other types of data, such as the short, noisy sequence used in this evaluation. The fact that this is not the case is a positive indication.

\clearpage

\subsection{Transformations}
% \FloatBarrier{}

\begin{figure}[h]
    \begin{center}
        \includegraphics[width=\textwidth]{resources/dft_heat_map}
    \end{center}
    \caption{\small{Anomaly vectors $A_{k, t} $ for $k = 1,2,\dots,100$ with and without the discrete Fourier transform.}}
\label{fig:dft_heat_map}
\end{figure}

\begin{wrapfigure}{r}{0.5\textwidth}
\changecaptionwidth
\captionwidth{0.5\textwidth}
\includegraphics[width=0.5\textwidth]{resources/dft_errors}
\caption{\small{Errors of the $A_{k, t}$.}}
\label{fig:dft_errors}
\end{wrapfigure}

As discussed in Chapter~\ref{ch:time_series}, applying transformations to extracted subsequences prior to evaluation, such as to perform dimensionality reduction, might assist in discovering certain types of anomalies. While a large number of compressions and other transformations deserving investigation have been proposed, due to time constraints, only the discrete Fourier transform (DFT) was properly implemented in ADRT.

The performance of the DFT was investigated by evaluating $s^*$ for $k = 1,2,\dots, n$ with and without the DFT\@. A heat map of the results is shown in Figure~\ref{fig:dft_heat_map}, and a plot of the corresponding errors is shown in Figure~\ref{fig:dft_errors}.

While the DFT gave fairly accurate anomaly vectors for low values of $k$, it performed poorly overall, returning less accurate anomaly vectors and higher error values over all $k$. This is reasonable: the DFT is not expected to perform well on random data. A proper evaluation of the performance characteristics of kNN methods using the DFT would require a more diverse dataset.

\subsection{The sliding window width}
\label{sect:w}

\begin{figure}
    \centering
    \subbottom{\includegraphics[trim=5mm 5mm 0mm 0mm, width=0.96\linewidth]{resources/w_heat_map}}
    \subbottom{\includegraphics[trim=0mm 0mm 2mm 0mm, width=0.49\linewidth]{resources/w_error}}
    \subbottom{\includegraphics[trim=2mm 0mm 0mm 0mm, width=0.49\linewidth]{resources/w_time}}
    \caption{Results for $P^*_w$, where $w = 1, 2, \dots, 50$.}
\label{fig:w_plot}
\end{figure}

Since the sliding window width $w$ determines the size of the elements used by the anomaly measure, it can be expected to have a significant impact on the size of detected features. To determine if this was the case, the anomaly vectors $A_w$ for $w = 1$, $2$, \dots, $50$ were computed and examined. The results are shown in Figure~\ref{fig:w_plot}.

As seen in the figures, very low values of this parameter are associated with a very high error. This is expected, since as $w$ tends to $1$, the target anomaly type is reduced to point anomalies. Furthermore, all errors increase sharply as $w$ nears $20$, indicating that large values of $w$ can lead to inaccurate results.

Interestingly, as seen in the figure, beyond $w \approx 3$, increasing $w$ essentially amounts to smoothing the resulting anomaly vectors. Since the anomaly in $s^*$ has a relatively small width of $40$, and since its surroundings have low anomaly values for low values of $w$, this could help explain why the anomaly is not detected after $w \approx 40$.

It is further interesting to note that while the errors are at a minimum when $w \approx 5$, the anomaly vectors in this area contain three separate spikes in the vicinity of the anomaly, rather than a single smooth bump. Arguably, the anomaly vectors at $w \approx 10$ are preferable, since they more clearly mark the anomaly. This suggests that the error measures may need refinement.

Finally, while the evaluation time ought to be roughly independent of $w$ (or proportional to the evaluation time of the distance metric with vectors of length $w$), there is a clear decrease in the computation time as $w$ grows. This is unsurprising, considering that the relatively small width of the evaluation sequence means fewer elements are evaluated as $w$ grows.

\subsection{The sliding window step}
\label{sect:s}

\begin{figure}
    \centering
    \subbottom{\includegraphics[trim=5mm 7mm 0mm 0mm, width=0.96\linewidth]{resources/s_heat_map}}
    \subbottom{\includegraphics[trim=0mm 0mm 2mm 0mm, width=0.49\linewidth]{resources/s_error}}
    \subbottom{\includegraphics[trim=2mm 0mm 0mm 0mm, width=0.49\linewidth]{resources/s_time}}
    \caption{Results for $P^*_s$, where $s = 1,2,\dots,10$. As expected, the oracle execution times scale as $O(1/s^2)$}
\label{fig:s_plot}
\end{figure}

The sliding window step length, $s$, is interesting mainly for the large effect it has on the execution time. For a brute-force kNN anomaly measure with the trivial context and sliding window filters, the number of comparisons performed on a sequence of length $L$ is $O({(L/s)}^2)$. It is therefore desirable to choose a value of $s$ that is as large as possible. However, it is likely that all three errors increase with $s$ for all sequences, and large $s$ values might lead to poor results.

To gain some insight into how kNN methods perform with higher $s$, the anomaly vectors $A_s$ were computed for $s = 1$ to $10$ (the value of $w$ is $10$ in the default configuration). The results are shown in Figure~\ref{fig:s_plot}.

As seen, the anomaly vectors are fairly accurate for all $s$. No major false anomalies are exhibited for $s < 8$, and the actual anomaly is still clearly detected over all $s$. This is reflected in the error plot: all errors are low until $s \geq 8$. Additionally, the evaluation time plot follows the expected $O(1/s^2)$ trend.

In light of these results, perhaps a multi-resolution scheme should be considered, in which a preliminary, `coarse' evaluation (corresponding to high $s$), and a `fine' evaluation (corresponding to low $s$) is performed only on those subsequences which are given the highest anomaly scores in the coarse evaluation. Depending on how the subsequences for the fine evaluation are selected, and on the context type, such an algorithm could achieve either lower computational complexity or an evaluation time reduction by a constant factor. If, as indicated in this evaluation, false positives but no false negatives are introduced as $s$ increases, fine evaluation would only rule out false anomalies, and there would be no loss of analytical power.

\subsection{The context width}
\label{sect:m}

\begin{figure}
    \centering
    \subbottom{\includegraphics[trim=7mm 7mm 3mm 0mm, width=0.96\linewidth]{resources/m_heat_map}}
    \subbottom{\includegraphics[trim=0mm 0mm 2mm 0mm, width=0.49\linewidth]{resources/m_error}}
    \subbottom{\includegraphics[trim=2mm 0mm 0mm 0mm, width=0.49\linewidth]{resources/m_time}}
    \caption{Results for $P^*_m$, where $m = 20,21,\dots,400$. Note the false anomaly present at the left end of the anomaly vectors until $m \approx 330$.}
\label{fig:m_plot}
\end{figure}

Which values of the context width $m$ are appropriate depends heavily on the application domain and on the types of anomalies present in the data. Ideally, the importance of the context width should be evaluated by considering several sequences with a natural context concept, such as the bottom series in Figure~\ref{fig:anomaly_types}. Constructing representative artificial datasets of such sequences is likely to be difficult, so real-world series should be used for such an evaluation.

While such datasets are not available, a simple evaluation on the available data can still prove illuminating. The sequence $s^*$ is highly homogeneous and admits no non-standard contexts. Thus, all errors should be expected to decrease monotonically with increasing $m$. To confirm this, the anomaly vectors $A_m$ were computed for $m = 20$ to $400$. The results of this evaluation are shown in Figure~\ref{fig:m_plot}.

As these figures demonstrate, the anomaly vectors identify a false anomaly at the left end until $m \approx 330$, at which point the false anomaly disappears and the errors decline sharply. That this false anomaly appears for small context widths is understandable since, as seen in Figure~\ref{fig:reference_sequence}, the sequence includes values at its left end that are not seen again until the right end. As expected, the error is minimized when the trivial context (corresponding to $m > 390$) is incorporated.

Finally, it should be noted that while the size of the reference set, and consequently the evaluation time, grows linearly with the size of the context, the average context size only grows linearly with $m$ when $m$ is much smaller than the sequence length. When $m$ is close to the sequence length, the context size for a large portion of the subsequences extracted by the evaluation filter will be limited by the sequence edges. This leads to the curve in the bottom right of the figure.

\subsection{The aggregation function}
\label{sect:A}

To get an idea of how the choice of aggregation function $\Sigma$ affects the analysis, the anomaly vectors $A_{k, \Sigma}$ were computed and analyzed for the minimum, maximum, median and mean aggregation functions, with $k = 1,2,\dots,100$. Heat map plots of the results are shown in Figure~\ref{fig:aggregator_heat_map}, and plots of the corresponding error measures are shown in Figure~\ref{fig:aggregator_error}. Single anomaly vectors for $k=1$ are shown in Figure~\ref{fig:aggregator_vectors}.

As seen in Figures~\ref{fig:aggregator_error} and~\ref{fig:aggregator_vectors}, $\Sigma_{min}$ and $\Sigma_{max}$ produce blocky, piecewise constant anomaly vectors, while $\Sigma_{mean}$ (and, to a lesser extent, $\Sigma_{median}$) produces smooth anomaly vectors.

% \begin{figure}
%     \centering
%     \subbottom[Heat map of the $A_m$. Note the false anomaly present at the left end of the anomaly vectors until $m \approx 330$.]{%
%         \includegraphics[width=\linewidth]{resources/aggregator_heat_map}}
%     \subbottom[Errors.]{%
%         \includegraphics[width=0.45\linewidth]{resources/aggregator_error}}
%     \subbottom[Evaluation times.]{%
%         \includegraphics[width=0.45\linewidth]{resources/aggregator_vectors}}
%     \caption{Results for $P^*_m$, where $m = 20,21,\dots,400$}
% \label{fig:m_plot}
% \end{figure}

\begin{figure}[!ht]
    % \vspace{-20pt}
    \begin{center}
        \includegraphics[trim = 5mm 2mm 10mm 2mm, clip, width=\textwidth]{resources/aggregator_heat_map}
    \end{center}
    % \vspace{-10pt}
    \caption{\small{Heat maps showing $A_{k, \Sigma}$ for the four aggregation functions.}}
\label{fig:aggregator_heat_map}
    % \vspace{-15pt}
\end{figure}

As is seen in the plots, $\Sigma_{mean}$ and $\Sigma_{median}$ performed almost identically for all error measures.

As could be expected, $\Sigma_{min}$ consistently leads to the highest values of $\epsilon_F$. It is likely to give a low score to a point if a single element containing that point has a low anomaly score, which effectively means that parts of anomalies will tend to be undervalued---something $\epsilon_F$ is sensitive to. In contrast, $\Sigma_{max}$ consistently led to the lowest values for $\epsilon_F$. This is also as expected, since $\Sigma_{max}$ will assign high values to any point contained in an anomalous subsequence.

Similar, but less clear, results were obtained for $\epsilon_E$. $\Sigma_{min}$ consistently performed the worst with low $k$, while $\Sigma_{max}$ performed the best, on average, with $k$ up to $40$.

Finally, $\epsilon_N$ was consistently lower for $\Sigma_{min}$ than for $\Sigma_{max}$. This is likely a consequence of the fact that $\Sigma_{min}$ tends to assign scores close to zero to all elements except for a few, while $\Sigma_{max}$ tends to assign scores close to zero to only a few elements. As discussed in Section~\ref{sect:error_measure_eval}, $\epsilon_N$ has a bias in favor of anomaly vectors where most elements are close to zero.

\begin{figure}[!ht]
    % \vspace{-5pt}
    \begin{center}
        \includegraphics[width=\textwidth]{resources/aggregator_error_2}
    \end{center}
    % \vspace{-18pt}
    \caption{\small{Errors of the anomaly vectors $A_{k, \Sigma}$.}}
\label{fig:aggregator_error}
    % \vspace{-10pt}
\end{figure}

\begin{wrapfigure}{r}{0.49\textwidth}
\changecaptionwidth
\captionwidth{0.5\textwidth}
\includegraphics[width=0.5\textwidth]{resources/aggregator_vectors}
\caption{\small{Plot of the $A_{k, \Sigma}$ for $k=1$.}}
\label{fig:aggregator_vectors}
\end{wrapfigure}

In conclusion, all evaluated choices of $\Sigma$ performed roughly equally well on $s^*$ (arguably, the minimum $\Sigma$ performed slightly worse than the others). If this holds in general, then it appears that the choice of $\Sigma$ is mainly one of aesthetics.

\include{discussion}
\chapter{Conclusions}
\label{ch:conclusions}

The report is concluded with a short summary and a discussion of a few possible directions for future work.

\section{Summary}

Overall, the project was successful. The new theory introduced in the form of the optimisation problem formulation and the framework represent a step towards powerful, user-friendly tools for automated anomaly detection research. Additionally, \texttt{ad-eval} has shown that the framework can be implemented and used to effectively find anomaly detection methods for real applications. Furthermore, the evaluation utilities and evaluation scripts in \texttt{ad-eval} have shown that producing open, reproducible anomaly detection research need not be difficult. Finally, as summarized in the next section, the project illuminated several new frontiers for future work.

However, there were some shortcomings. Initially, the project focus was on the implementation and evaluation of a few specific methods for specific datasets. This focus gradually shifted towards a more theoretical one throughout the course of the project, which meant that a large body of work was produced that was ultimately discarded.

Moreover, a proper evaluation of \texttt{ad-eval} could not be performed due to lack of data and time. As a result, exactly how practical and useful the framework really is for actual research is still an open question. However, this has been partially mitigated by the provision of reusable evaluation scripts and utilities, which can be used to properly answer this question when the right data becomes available.

\section{Future work}
Several interesting directions for future work have been opened up through this project, a few of which are now discussed.

\subsection{Optimisation}
Developing efficient optimisation methods is key to the successful implementation of the optimisation problem and framework. Currently, only naive optimisation methods have been implemented in \texttt{ad-eval}; faster and better methods must be implemented for it to be practical for real-world applications.

There are several interesting paths to take towards this goal. As indicated in the previous chapter, at least for distance-based anomaly measures, errors seem to vary rather smoothly over problem sets, with few non-global error minima or other irregularities. Thus, it can be expected that improved optimisation heuristics, such as simulated annealing, could have a large impact.

Another possibility would be perform the optimisation over a larger set of problems, for instance by allowing combinations of elements of the problem set. This would be a fairly trivial extension, and would allow for the optimisation to produce ensemble-style combinations of methods.

Finally, it seems likely that the optimisation itself could be learned to some degree. For instance, if correlations between the accuracy of problems and the underlying characteristics of the data being analyzed could be found, then these could likely be exploited to provide very efficient optimisation methods. Another, related possibility would be to allow for user-assisted optimisation.

The optimisation process could also be made more efficient through performance improvements. Performance has been deliberately deemphasized in favor of simplicity in the development of \texttt{ad-eval}, and it can be expected that dramatic performance improvements could be reaped fairly easily by, for instance, enabling parallelism or rewriting \texttt{ad-eval} in a programming language better suited for numeric computations. A performance optimisation that can be expected to have drastic results is the multi-resolution heuristic suggested in Section~\ref{sect:s}.

\subsection{New applications}
There is much work to be done still on applying the framework to sequences. To begin with, task involving finding anomalous sequences, as well as tasks involving discrete and categorical data have yet to be studied and implemented in \texttt{ad-eval}. There are also several interesting transformations and anomaly measures which are encountered in the literature, but which have yet to be implemented in \texttt{ad-eval}. Finally, how \texttt{ad-eval} performs on sets of more diverse real-valued sequences needs to be researched.

Of course, applying the framework and \texttt{ad-eval} to entirely different application domains, such as fraud detection or bioinformatics would also be interesting. Essentially, this should be as simple as defining the data and problem sets, as well as (optionally) implementing a few more component choices.

\subsection{Deployment}
A major goal in the design of \texttt{ad-eval} was for it to be as modular and flexible as possible, so that creating a user interface for it or hooking it up for use with data analysis software such as Splunk would be easy. This was achieved in two ways: first, by designing it as a collection of command line tools, and second, by making it scriptable. Extending \texttt{ad-eval} to work with existing data analysis software or a custom user interface could help further elucidate just how sound and practical the framework is for everyday analysis use, as well as how user-friendly it can be made.

One of the main benefits of the optimisation problem formulation and framework is that they provide a venue for automating the work-intensive aspects anomaly detection research, and replacing them with an automated optimisation process. As was mentioned in Chapter~\ref{ch:background}, anomaly detection research typically necessitates the involvement of both a domain expert and an export on anomaly detection, and is a laborious process involving plenty trial and error. In contrast, \texttt{ad-eval} lets users focus almost entirely on more significant aspects of the process: finding methods suitable for a given domain becomes a matter of writing a script which defines the data, places a few restrictions on the problem set, and runs an optimisation. The declarative nature of this work could be taken one step further by defining a domain-specific language for describing the data and the problem set. Reimplementing \texttt{ad-eval} in a language with an expressive type system, such as Haskell, would be a good first step in this direction.


%-----------------------------------------------------------
\addcontentsline{toc}{chapter}{\numberline{}Bibliography}
\include{bibliography}
\bibliography{mybib1}
\bibliographystyle{acm}

%-----------------------------------------------------------
%\appendix
%-----------------------------------------------------------
\end{document}
