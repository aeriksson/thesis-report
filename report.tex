\documentclass[a4paper,twoside]{report}
\usepackage[utf8]{inputenc}
\usepackage[T1]{fontenc}
\usepackage[english]{babel}
\usepackage{amsmath}
\usepackage{amssymb}
\usepackage{wrapfig}
\usepackage{algpseudocode}
\usepackage[a4paper,inner=2.5cm,outer=2.5cm,top=2.5cm,bottom=2.5cm,pdftex]{geometry}
\usepackage[pdftex]{graphicx}
\usepackage{epsfig}
\usepackage{epstopdf}
\usepackage{psfrag}
\usepackage{fancyhdr}
\usepackage{natbib}
\usepackage{babelbib}
\usepackage[hidelinks]{hyperref}
\usepackage{caption}
\usepackage{subcaption}
\usepackage{placeins}
\usepackage{float}
\usepackage{graphicx}
\usepackage{color,soul}

\date{}

%Section number in equation numbering
\numberwithin{equation}{section}


% Line spacing
\makeatletter
\let\std@footnotetext\@footnotetext
\usepackage{setspace}
\setstretch{1.4}
\let\@footnotetext\std@footnotetext
\makeatother


%Section and subsection styling
\makeatletter
\renewcommand{\chapter}{%
\@startsection
        {chapter}{1}{-20pt}{-4ex \@plus -1ex \@minus -.2ex}%
        {3.4ex \@plus.2ex}{\centering\Huge\scshape}%
}

\renewcommand{\section}{%
\@startsection
	{section}{1}{-20pt}{-4ex \@plus -1ex \@minus -.2ex}%
	{2.3ex \@plus.2ex}{\centering\Large\scshape}%
}
\renewcommand{\subsection}{%
\@startsection
	{subsection}{1}{-20pt}{-2.5ex \@plus -1ex \@minus -.2ex}%
	{1.8ex \@plus.2ex}{\centering\large\scshape}%
}
\makeatother

%No section numbering. References become empty.
%\renewcommand{\thesection}{}

%Margins
%\setlength{\topmargin}{0.4in}
%\setlength{\textheight}{9in}
%\setlength{\oddsidemargin}{.125in}
%\setlength{\textwidth}{6.25in}
\setlength{\headheight}{15pt}

%Settings for references
\bibpunct{[}{]}{;}{n}{,}{,}

%Theorems, Lemmas, Definitionos and Formulations
\newtheorem{definition}{Definition}[chapter]
\newtheorem{task}{Task}[chapter]
\newtheorem{problem}{Problem}[chapter]

\newcommand{\vect}[1]{\boldsymbol{#1}}
\newcommand{\argmin}{\operatornamewithlimits{argmin}}
\newcommand{\argmax}{\operatornamewithlimits{argmax}}

\newcommand{\HRule}{\rule{\linewidth}{0.5mm}}

\setcounter{tocdepth}{1}

%Begin Document
\begin{document}

\title{A Framework for Anomaly Detection with Applications to Sequences}
\author{André Eriksson}
\date{May 2014}
\blurb{Master's Thesis at NADA\\Supervisor: Hedvig Kjellström\\Examiner: TODO}
\trita{TRITA xxx yyyy-nn}


%Headers and footers
\pagestyle{fancy}
\fancyhead{}
\fancyhead[RE,LO]{\leftmark}
%\newcommand{\hdrsection}{\markright}
\renewcommand{\headrulewidth}{0pt}
%\fancyhead[L]{}
%\fancyhead[R]{\thesection \hdrsection}

\setcounter{page}{0}
%\thispagestyle{empty}

\begin{abstract}

Anomaly detection is an important issue in data mining and analysis, with applications in almost every area in science, technology and business that involves data collection. The development of general anomaly detection techniques can therefore have a large impact on data analysis across many domains. In spite of this, little work has been done to consolidate the different approaches to the subject.

In this report, this deficiency is addressed in the target domain of temporal machine-generated data. To this end, new theory for comparing and reasoning about anomaly detection tasks and methods is introduced, which facilitates a problem-oriented rather than a method-oriented approach to the subject. Using this theory as a basis, the possible approaches to anomaly detection in the target domain are discussed, and a set of interesting anomaly detection tasks is highlighted.

One of these tasks is selected for further study: \emph{the detection of subsequences that are anomalous with regards to their context within long univariate real-valued sequences}. A framework for relating methods derived from this task is developed, and is used to derive new methods and an algorithm for solving a large class of derived problems. Finally, a software implementation of this framework along with a set of evaluation utilities is discussed and demonstrated.
 
\end{abstract}

\tableofcontents
\clearpage

\setlength{\parindent}{0pt}
\setlength{\parskip}{2ex plus 0.5ex minus 0.2ex}

\chapter*{Foreword}

This report is the result of a master's thesis project at the KTH Royal Institute of Technology performed in cooperation with Splunk Inc., based in San Francisco, California, USA. The goal of the project was to find and implement efficient and general methods of anomaly detection suitable for the types of data typically found in Splunk.

Splunk is essentially a database and tool for indexing and analyzing very large sets of sequential machine-generated data (i.e. machine-generated data consisting of discrete events with associated time stamps). The goal of this project was chosen to be the identification of more powerful anomaly detection tools and their incorporation into Splunk, since it was estimated that a large share of Splunk users could benefit from such tools.

Due to the wide scope of this task, the project was split into three tasks, as follows:
\begin{enumerate}
    \item Research into which anomaly detection tasks are the most relevant to the target domain.
    \item Development of a software framework for evaluation of anomaly detection methods focused on the target domain.
    \item Evaluation of the performance of several different approaches on data from the given domain.
\end{enumerate}

Due to a lack of structured evaluations of and comparisons between methods in the anomaly detection literature, the first part proved particularly challenging. To address this task sufficiently, new theory was required. The majority of this report deals with the introduction of this theory and the discussion of anomaly detection tasks relevant to the target domain.

Based on this discussion and theory, a software implementation of a broad class of anomaly detection methods was developed. Called \texttt{ad-eval}, the implementation was designed as a minimalistic and flexible Python framework and released under an open source license at \url{http://github.com/aeriksson/ad-eval}.

Finally, an evaluation of implemented methods was performed using \texttt{ad-eval}. Unfortunately, a proper evaluation could not be performed due to time constraints and a lack of appropriate data. Instead, a simple, qualitative evaluation was performed, in order to give a general impression of how different components and parameter values affect the analysis, as well as demonstrate the capabilities and usage of \texttt{ad-eval}. The source code used in this evaluation was designed to be highly modular and was made available as part of the \texttt{ad-eval} source code repository, to guarantee the reproducibility of the results and to facilitate more thorough evaluations once adequate data becomes available.

\chapter*{Introduction}

\begin{figure}[htb]
    \vspace{-20pt}
    \begin{center}
        \includegraphics[trim= 30mm 10mm 30mm 10mm, clip, width=\textwidth]{resources/citations}
    \end{center}
    \vspace{-20pt}
    \caption{\small Approximate number of papers (by year) published between $1980$ and $2011$ containing the terms "anomaly detection", "outlier detection" and "novelty detection". All three terms exhibit strong upward trends in recent years. Source: Google Scholar.}
    \vspace{-0pt}
    \label{fig:citations}
\end{figure}

In recent years, anomaly detection has become increasingly important in a variety of domains in business, science and technology. Roughly defined as the automated detection within data sets of elements that are somehow abnormal, anomaly detection encompasses a broad set of techniques and problems. In part due to the emergence of new application domains, and in part due to the evolving nature of many traditional domains, new applications of and approaches to anomaly detection and related subjects are being developed at an increasing rate, as indicated in figure \ref{fig:citations}.

In spite of this increase in research activity, relatively few papers have been published that compare different approaches to anomaly detection tasks in relation to specific domains. As mentioned in the foreword, the main objective of this thesis project was to mitigate this deficiency in the context of large sets of temporal machine-generated data, with the goal of identifying and implementing the methods most appropriate for this domain.

The term \emph{machine-generated data} (or \emph{machine data} for short) refers to any data consisting of discrete events that have been created automatically from a computer process, application, or other machine without the intervention of a human. Common types of machine-generated data include computer, network, or other equipment logs; environmental or other types of sensor readings; or other miscellaneous data, such as location information \cite{machine_data}. Splunk is designed for this type of data, especially data sets where each event has an associated time stamp; such data is referred to as \emph{temporal machine data} throughout this report.

This project resulted in several contributions to the subject of anomaly detection. New theory was introduced in the form of the \emph{tasks and problems framework} and the \emph{component framework}. The former can be used to consolidate the different approaches to anomaly detection and to simplify reasoning about and comparing anomaly detection methods and problems systematically. It is used in this report to present the subject of anomaly detection in detail and to determine the anomaly detection tasks most pertinent to the target domain. Based on this discussion, a specific, previously unstudied task is chosen as a focus for the remainder of the report.

The component framework is used to simplify reasoning about and comparing problem formulations derived from this target task. It is used to derive a single, simple algorithm for solving a large class of problem formulations derived from the target task, as well as to derive a set of interesting problem formulations belonging to this class.

Finally, a software implementation of the component framework along with an extensive set of evaluation utilities, called \texttt{ad-eval}, is introduced. This software, which was released under an open-source license at \url{http://github.com/aeriksson/ad-eval} was developed as part of this project in order facilitate the unbiased and reproducible evaluation of anomaly detection methods on the target task, as well as to mitigate some of the deficiencies in evaluations found in the literature. While an appropriate evaluation could unfortunately not be performed within the scope of this project due to a lack of data, a qualitative evaluation of the methods implemented in \texttt{ad-eval} was performed and included in this report. All scripts used in this evaluation were included in the \texttt{ad-eval} source code repository, so that they can be reused for a proper evaluation once sufficient data is available.

In chapter \ref{ch:background}, various background information pertinent to the rest of the report is presented. Specifically, the subject of anomaly detection is presented in more depth, along with some of the issues with current approaches to the subject; the framework of tasks and problems is introduced; and the application of anomaly detection to Splunk is discussed. 

Chapter \ref{ch:tasks} contains a thorough, structured review of anomaly detection within the framework of tasks and problems. Each of the factors required to specify an anomaly detection problem is investigated, emphasizing the enumeration of choices of these factors. Where pertinent, the implications of and relations between these choices are discussed.

Since the types of anomalies that can be captured by a problem formulation depend heavily on how the data sets are presented, the selection of appropriate data representations constitutes an important aspect of the design of anomaly detection methods. The lack of structure often exhibited by machine-generated data sets makes this task especially challenging. Chapter \ref{ch:transformations} is dedicated to the discussion of various aspects of data representations and how they relate to sets in the target domain. 

In chapter \ref{ch:problems}, the preceding two chapters are used as a basis to derive the most important anomaly detection tasks when analyzing the types of machine-generated data encountered in Splunk and similar applications. The specific task of detection of contextual collective anomalies in long univariate time series is selected for implementation and analysis.

In chapter \ref{ch:methods}, the chosen task is treated in detail. The component framework is presented and used to identify potentially interesting problem formulations. Ways in which the framework may be modified to address other tasks is presented.

In chapter \ref{ch:evaluation}, some of the difficulties in evaluating the accuracy of anomaly detection methods for the target task are considered. A few error measures for evaluations on the target task are also proposed.

The design and structure of \texttt{ad-eval}, which consists of an implementation of the component framework and a comprehensive set of evaluation utilities, is discussed in chapter \ref{ch:implementation}. 

The results of the evaluation are presented in chapter \ref{ch:results}. Specifically, this chapter contains an explanation of the evaluation methodology used, a brief exposition relating the concept of parameter spaces to the framework developed in chapter \ref{ch:methods}, and a thorough qualitative evaluation of how parameter choices affect the analysis for kNN-based methods. 

Finally the report is concluded in chapter \ref{ch:discussion} with a summary of the project and a few possible directions for future study.

\chapter{Background}
\label{ch:background}

This chapter gives a brief introduction to the subject of anomaly detection. The framework of tasks and problems used throughout the paper is presented and justified.

\section{Anomaly detection}
\label{sect:adb}

In essence, anomaly detection is the task of automatically detecting items (\emph{anomalies}) in data sets that in some sense do not fit in with the rest of those data sets (i.e.\ are \emph{anomalous} with regard to the rest of the data). The nature of both the data sets and anomalies are dependent on the specific application in which anomaly detection is applied, and vary drastically between application domains. As an illustration of this, consider the two data sets shown in Figures~\ref{fig:example1} and~\ref{fig:example1}. While these are similar in the sense that they both involve sequences, they differ in the type of data points (real-valued vs.\ symbolic), the structure of the data set (one long sequence vs.\ several sequences), as well as the nature of the anomalies (a subseqeuence vs.\ one sequence out of many). Several surveys~\cite{hodge}~\cite{bakar}~\cite{chandola}~\cite{agyemang} and books~\cite{barnett}~\cite{hawkins}~\cite{leroy} have been published which treat various anomaly detection applications in greater depth.

\begin{figure}[htb]
    \centering
    \includegraphics[width=\textwidth]{resources/anomaly_example}
    \caption{\small Real-valued sequence with an anomaly at the center.}
    \vspace{-0pt}
\label{fig:example2}
\end{figure}

Like many other concepts in machine learning and data science, the term `anomaly detection' does not refer to any single well-defined problem. Rather, it is an umbrella term encompassing a collection of loosely related techniques and problems. Anomaly detection problems are encountered in nearly every domain in business and science in which data is collected for analysis. Naturally, this leads to a great diversity in the applications and implications of anomaly detection techniques. Due to this wide scope, anomaly detection is continuously being applied to new domains despite having been researched for decades.

\begin{figure}[htb]
    \centering
    \begin{tabular}{| l | l l l l l l l l |}
        \hline
        $\mathbf{S_1}$ & login & passwd & mail & ssh & \dots & mail & web & logout \\ \hline
        $\mathbf{S_2}$ & login & passwd & mail & web & \dots & web & web & logout \\ \hline
        $\mathbf{S_3}$ & login & passwd & mail & ssh & \dots & web & web & logout \\ \hline
        $\mathbf{S_4}$ & login & passwd & web & mail & \dots & web & mail & logout \\ \hline
        $\mathbf{S_5}$ & login & passwd & login & passwd & login & passwd & \dots & logout \\\hline
    \end{tabular}
    \caption{Several sequences of user commands. The bottom sequence is anomalous compared to the others.}
\label{fig:example1}
\end{figure}

In other words, anomaly detection as a subject encompasses a diverse set of problems, methods, and applications. Different anomaly detection problems and methods often have few similarities, and no unifying theory exists. Indeed, the eventual discovery of such a theory seems highly unlikely, considering the subjectivity inherent to most anomaly detection problems. Even the term `anomaly detection' itself has evaded any widely accepted definition~\cite{hodge} in spite of multiple attempts.

Despite this diversity, anomaly detection problems from different domains often share some structure, and studying anomaly detection as a subject can be useful as a means of understanding and exploiting such common structure. Anomaly detection methods are vital analysis tools in a wide variety of domains, and the set of scientific and commercial domains which could benefit from improved anomaly detection methods is huge. Indeed, due to increasing data volumes, exhaustive manual analysis is (or will soon be) prohibitively expensive in many domains, rendering effective automated anomaly detection critical to future development.

\section{On Anomaly Detection Research}

Most anomaly detection research work consists of either taking existing methods and applying them to new applications (i.e.\ on new types of data), or investigating new methods for previously studied applications. In order to handle the increasing need for effective anomaly detection in many areas of business and science it is vital that these activities can be performed in a highly automated and straight-forward manner. However, there are a few issues with the current state of the subject, which make anomaly detection research needlessly complicated.

Firstly, comparing different anomaly detection methods found in the literature is difficult, since even though it might not appear so at first glance, papers on anomaly detection often target subtly different problems. For instance, TODO. This renders direct comparisons problematic and makes it hard to assess which methods are appropriate to use in new applications. A systematic way of comparing anomaly detection methods would be helpful in mitigating this problem.

The second problem is that there is often a lack of reproducibility of produced results. Due in part to the subjective nature of the subject, and in part to a historical lack of freely available datasets, new methods are often not adequately compared to previous methods. Furthermore, the performance of many anomaly detection methods is often highly dependent on parameter choices, and only the results for the best parameter values (which might be difficult to find) are often presented~\cite{keogh5}. Finally, source code is often unavailable, which makes verification a tedious process. These issues, when taken together, make it hard to reproduce results, which in turn makes anomaly detection research needlessly difficult.

This work attempts to simplify anomaly detection research by addressing the above issues. First, a general framework for systematically comparing anomaly detection problem formulations is presented, the purpose of which is to help highlight similarities and differences between problems, and thereby simplify the application of existing methods to new domains by mitigating the first problem above.

This framework is then used to formalize the subject of anomaly detection in the domain of real-valued time series, and to reformulate the activity of finding appropriate methods for specific data sets in this domain as an optimization problem over the set of possible algorithms. It is then shown that solutions to this optimization problem can be algorithmically approximated for a large class of algorithms (including most previously published methods).

Finally, a software implementation of this optimization problem is presented, along with some preliminary performance results. This mitigates the second problem above for anomaly detection in real-valued time series, by providing an environment in which previous methods can be easily replicated and compared on arbitrary datasets.

\chapter{A Framework for Anomaly Detection}
\label{ch:framework}

In this chapter, a framework for reasoning about anomaly detection problems is presented. In this report, the framework is mainly utilised as a means of limiting the scope of the optimisation problem outlined in the previous chapter, but it is also useful as a means of relating different anomaly detection methods and finding new methods for specific applications.

A core idea of the framework is to classify and interrelate anomaly detection methods based on a few key factors. In Section~\ref{sect:the_framework}, the basic principles and motivations behind the framework are presented. In sections~\ref{sect:data_format} through~\ref{sect:output_format}, the individual factors, as well as common choices for them, are presented.

Finally, in Section~\ref{sect:related_tasks}, a few problems that are typically considered related to anomaly detection are presented in relation to the framework.

\section{Description}
\label{sect:the_framework}

As previously mentioned, the optimisation problem presented in section~\ref{sect:problem_formulation} is intractable since to the set of possible problem formulations is so large. In order to successfully apply the optimisation problem to some specific application domain, the set of problems under consideration must be restricted. However, appropriate restricted set of problems can be problematic. Ideally, such a problem set should contain all problems previously used for the specific domain, while remaining small enough to be tractable. Without some method of systematically relating problems and approaches, however, finding such a set can be complicated.

We now present a framework, using which finding an appropriate problem set can be done systematically. The key idea behind this framework is that if anomaly detection problems can be described using a few choices of standardised factors, then comparing problems, as well as restricting the problem set and finding gaps in the methods researched for applications becomes much simpler.

We here propose using the following factors to classify anomaly detection problems:
\begin{description}
  \item[Dataset format] How the datasets (method inputs) are structured.
  \item[Solution format] How the solutions (method outputs) should be structured.
  \item[Training data] What type of training data is available to methods.
  \item[Anomaly type] Which structural properties of the data should be considered.
  \item[Anomaly measure] The heuristic used to assess how anomalous items are.
\end{description}
As mentioned previously, one major advantage of an approach based on factors is that it facilitates comparisons of methods. Studying common factor choices between problems can be useful in illuminating differences and similarities. Furthermore, as we will see, specific factor choices often generalise other factor choices. Recognising this is crucial to finding general methods. Thus, the framework can be utilized to recognize when problems are similar or (partially) generalise other problems.

The remainder of this chapter deals with each of the factors presented above in turn, listing choices treated by existing methods, and introducing a few new concepts along the way.

\section{Dataset format}
\label{sect:data_format}

Obviously, the available choices of dataset format are typically heavily restricted by the application under consideration. However, transforming source datasets to some more appropriate format is a common and important task. Since the format of the data also limits the set of suitable problem definitions, selecting an appropriate format is important.

In the literature, various classifications are used to distinguish different dataset formats. In this section, a few of these are presented.

To begin with, assuming that input is given as a data set $D = (d_1, d_2, \dots, d_n)$, where the $d_i$ belong to some set $X$, the structure of $X$ is of interest. A distinction is typically made between categorical, discrete, and real-valued datasets based on the properties of $X$. A dataset is said to be \emph{categorical} (or \emph{symbolic}) if $X$ is finite, \emph{discrete} if $X$ is countable and \emph{real-valued} if $X \subseteq \mathbb{R}^n$ for some $n$. It is also frequently the case that $X$ consists of some combination of categorical, discrete and real-valued data. In this case, $D$ is referred to as \emph{mixed}.

Categorical (or \emph{symbolic}) data arises in many contexts and is comparatively easy to handle. In many cases, methods for handling categorical data are relatively mature. For instance, categorical sequences are exhaustively treated in~\cite{TODO}. Often, there is no ordering or other relation between the elements in the underlying set, which makes the analysis simpler.

Many techniques used for categorical data (such as information-theoretic measures and certain probabilistic models) are not applicable to discrete or real-valued data. In such cases, a suitable discretisation of the underlying set might be useful. TODO: move discussion about discretisation here.

Real-valued data sets encountered in applications are often samples of processes that are assumed to be continuous. When the ordering of these samples is reflected in the data set, the dataset is sometimes referred to as \emph{continuous}.

Anomaly detection in mixed datasets is relatively poorly understood; typically, such data is split into several separate datasets that are either categorical, discrete or real-valued before analysis.

It is fairly common to use \emph{numerosity reduction} techniques to convert discrete data to categorical data, or to compress categorical data. TODO: put numerosity reduction discussion here

Another important classification considers the dimensionality of the data points. A distinction is typically made between univariate and multivariate data. A dataset $D = (d_1, d_2, \dots, d_n)$ is said to be \emph{univariate} if the $d_i$ are scalars, and \emph{multivariate} if the $d_i$ are vectors of length greater than one.

While the distinction between uni- and multivariate data might seem superfluous, it proves important in applications. Most machine learning methods take significantly longer to learn (both in terms of time and convergence) as the dimensions of the data increase. Furthermore, many methods are not applicable to multivariate data at all.

To mitigate the difficulties of analysing high-dimensional data, \emph{dimensionality reduction} is often performed as a pre-analysis step on multidimensional datasets. TODO: put dimensionality reduction discussion here.

In applications there is usually additional structure present, such as orderings or other relations between elements within datasets, that can be utilised to improve the analysis. Such additional structure is discussed in Sections~\ref{sect:anomaly_types} and~\ref{sect:anomaly_measures}.

\section{Training data}
\label{sect:training_data}

\begin{wrapfigure}{r}{0.5\textwidth}
    \vspace{-25pt}
    \begin{center}
        \leavevmode
        \includegraphics[width=0.5\textwidth]{resources/supervision}
    \end{center}
    \caption{{\small Euler diagram of the available training data for the four types of supervision.}}
\label{fig:supervision}
    \vspace{-40pt}
\end{wrapfigure}

As is customary in most areas of machine learning, anomaly detection problems are classified as either \emph{supervised}, \emph{semi-supervised} or \emph{unsupervised} based on the availability of \emph{training} (or \emph{training}) data. In contrast to the \emph{evaluation data}, which is the data in which anomalies are to be found, the training data acts as a baseline, defining what constitutes normal and/or anomalous. The three classes of training data are now discussed and presented as tasks, starting with supervised anomaly detection:

\begin{task}[Supervised anomaly detection]
\label{task:supervised}\ \\
    Given training sets $N$ (containing normal training data) and $A$ (containing anomalous training data), find anomalies (with regard to $N$ and $A$) in a data set $D$.
\end{task}
In essence, supervised anomaly detection constitutes a traditional supervised classification problem. As such, it can be handled by any two-class classifier, such as regular support vector machines. However, characteristics shared by most anomaly detection applications make supervised approaches unsuitable.

First, anomalous training data is almost always relatively scarce, potentially leading to skewed classes (described in~\cite{phua} and~\cite{joshi}). Furthermore, supervised anomaly detection methods are by definition unable to detect types on anomalies that are not represented in $A$, and so can not be used to find \emph{novel} anomalies. This is problematic as it is often not feasible to construct an $A$ containing all possible anomalies.

Semi-supervised anomaly detection, on the other hand, assumes the availability of either only $A$ or only $N$. Divided according to which of the two training classes is available, the following two tasks can be specified:
\begin{task}[Semi-supervised anomaly detection with normal training data]
  Given a training set $N$ containing normal training data, find anomalies (with regard to $N$) in a data set $D$.
\end{task}
\begin{task}[Semi-supervised anomaly detection with anomalous training data]
\label{task:semisupervised_anomaly_detection}
  Given a training set $A$ containing anomalous training data, find anomalies (with regard to $A$) in a data set $D$.
\end{task}

While the latter task has been discussed (for instance in~\cite{dasgupta}), the vast majority of methods focus on the former. Considering the difficulties involved in obtaining anomalous training data, as mentioned above, this should not be surprising.

Semi-supervised methods are often used more frequently than supervised methods due to the relative ease with which they can be configured.

Finally, unsupervised anomaly detection requires no training data set, and can be defined by the following task:
\begin{task}[Unsupervised anomaly detection]
\label{task:unsupervised_anomaly_detection}
  Find subsets of some data set $D$ that are anomalous with regard to the rest of $D$.
\end{task}

For unsupervised anomaly detection, $D$ itself is referred to as the training data.

Since training data is not always available, unsupervised methods are typically considered to be of wider applicability than both supervised and semi-supervised methods~\cite{chandola}. However, unsupervised methods are unsuitable for certain tasks. Since training data can not be manually specified, it is more difficult to sift out uncommon but uninteresting items in unsupervised anomaly detection than in semi-supervised anomaly detection. Furthermore, unsupervised methods will not detect anomalies that are common but unexpected (although such items are arguably not anomalies by definition).

It is useful to note that unsupervised anomaly detection tasks can often be reduced to semi-supervised anomaly detection tasks by setting $A = D$ and modifying the underlying anomaly measure such that all elements $a_i \in A$ are judged as dissimilar to themselves.

Of course, it is sometimes not feasible or desirable to compare items with the entirety of the training data set. This is mainly the case when the data set supports additional structure, such an ordering or metric, which gives rise to a natural concept of locality within the data set. As a concrete example of such an application, consider unsupervised anomaly detection in a long sequence: often how an item compares to those items `closest' to it (in the ordering) is much more relevant to whether or not that item should be considered an anomaly than how it compares to the rest of the sequence.

In such cases, it is reasonable to associate with each individual data item a subset of the training data, and let this subset constitute the training data for that item. Throughout this report, such subsets will be referred to as the \emph{contexts} of the individual data items. Formally, this can be represented through a \emph{context function} that maps each subset $D'$ of the data set (or at least those subsets of interest to the analysis) to some set $C(D')$ such that $D' \cap C(D') = \emptyset$. The requirement that $D' \cap C(D') = \emptyset$ is necessary to keep elements from lowering their own anomaly scores when performing unsupervised anomaly detection. For the purposes of this report, contexts are mainly interesting for unsupervised anomaly detection, and can henceforth be assumed to refer to subsets of the evaluation data unless explicitly stated.

\begin{figure}[thb]
    \vspace{-4pt}
    \begin{center}
        \leavevmode
        \includegraphics[width=1\textwidth]{resources/contextz}
    \end{center}
    \vspace{-15pt}
    \caption{{\small Schematic view of a data set illustrating a few contexts. In each panel, the black dots represent selected items, the dark grey dots represent items in the context of the selected items, and the light grey dots indicate items not in the context of the selected items. The left panel shows the trivial context---all items are part of the context. The middle panel shows a local context of a single item. The right panel shows a local context of a subset of the data set.}}
\label{fig:contexts}
    \vspace{-5pt}
\end{figure}

Consider again the example of a long sequence. Writing this sequence as $S = (s_1, s_2, \dots, s_n)$, a reasonable context function defined for individual points could be the following:
\[
    C(s_i) = \{s_{i-w}, s_{i - w + 1}, \dots, s_{i - 1}, s_{i + 1}, s_{i + 2}, \dots, s_{i + w}\}.
\]
When using this context, which is referred to as the \emph{symmetric local context}, the local characteristics of the sequence around $s_i$ are taken into account, while the rest of the sequence is ignored.

Context functions $C(d)$ defined on individual elements $d \in D$ (such as the one above) can be naturally extended to subsets $D' \subseteq D$ of the data by defining
\[
    C(D') = \bigcup_{d \in D'} C(d) \setminus D'.
\]
The context functions encountered in this report are all on this form. For this reason, there is little need to make a distinction between contexts defined for single elements and contexts defined for subsets.

Note that contexts can be seen as a generalization of the concept of training data. For instance, the \emph{trivial context}, given by, for $d \in D$; $C(d) = D \setminus d$, corresponds to unsupervised anomaly detection. It is obtained when the scope of any local context grows large enough.

Figure~\ref{fig:contexts} shows a schematic view of a data set, along with three contexts. The leftmost panel illustrates the trivial context of a single element, the middle panel illustrates a local context of a single element, and the rightmost panel illustrates the natural extension of this local context to subsets.

\section{Anomaly types}
\label{sect:anomaly_types}

\begin{figure}[htb]
    \begin{center}
        \includegraphics[width=1\textwidth]{resources/filters}
    \end{center}
    \caption{{\small Schematic illustration of filters and contexts acting on an evaluation sequence $S = (s_1, s_2, \dots, s_{40})$. The top panel shows the evaluation set $E = \mathcal{F}(S) = \{e_1, e_2, \dots, e_{19}\}$ extracted by a sliding window filter with width $4$ and step $2$. The bottom panel shows the local symmetric context of $e_{10}$ with width $w = 12$: $C(e_{10}) = \{c_1, c_2, \dots, c_{24}\}$, as well as the training data set $R_{e_{10}} = \mathcal{F}_R(e_{10}) = \{r_1, r_2, \dots, r_{10}\}$ extracted by an analogous sliding window training filter.}}
\label{fig:filters}
\end{figure}

An important aspect of any problem is which subsets of the data set $D$ to consider when looking for anomalies; i.e.\ which subsets of $D$ should constitute the \emph{evaluation set} $E$. If all subsets are considered, the size $|E|$ of $E$ is $2^{|D|}$. This number is obviously too large to handle effectively, and the evaluation set must somehow be limited.

Fortunately, only a small fraction of all possible subsets is typically of interest in any given application. Precisely which subsets are interesting depends on the structure of $D = \{d_1, d_2, \dots, d_n\}$. If $D$ lacks additional structure (such as an ordering or metric) inducing a concept of locality, then it is reasonable to consider only the singleton sets, i.e. $E = \{\{d_i\} | d_i \in D\}$. When such additional structure exists (and is pertinent to the analysis), it is reasonable to let $E$ consist of subsets in which all elements are `close' (with regards to this additional structure).

As an example, consider a sequence $S = (s_1, s_2, \dots, s_n)$. As mentioned in the previous section, a locality concept is naturally induced by the sequence ordering, and it reasonable to let $E$ consist of contiguous subsequences of $S$:
\[
    E = \{(s_{a_1}, s_{a_1 + 1}, \dots , s_{b_1}) , (s_{a_2}, s_{a_2 +1}, \dots, s_{b_2}), \dots, (s_{a_k}, s_{a_k+1}, \dots, s_{b_k})\}.
\]
For such $E$, it is the case that $|E| \in O(|D|^2)$. Furthermore, it is often the case that not all contiguous subsequences must be evaluated---for instance it may suffice to treat only subsequences of some specific length, leading to $|E| \in O(|D|)$. Finally, if the ordering is not relevant to the analysis, then $E$ should be the singleton sets of $S$, and $|E| = |D|$. Thus, placing reasonable restrictions (based on the structure of the data set) on $E$ can render the analysis much more manageable.

To facilitate the construction of $E$, the concept of $\emph{filters}$ can be used. An \emph{evaluation filter} is a function $\mathcal{F}_E(D): D \mapsto E \subset 2^D$ that constructs the evaluation set. One evaluation filter for sequences used extensively in this report is the \emph{sliding window filter}:
\[
    \mathcal{F}_E(S) = \{(s_1, s_2, \dots, s_w), (s_{s+1}, s_{s+2}, \dots, s_{s+w}), \dots, (s_{n-w}, s_{n-w+1}, \dots, c_{n})\}\footnote{It is here assumed that $ (s+w) | n$. If this is not the case, the last element extracted might be a bit different.}
\]
with width $w$ and step $s$. This filter is the most reasonable choice for sequences when all items in $E$ must be of the same length (as is typically the case).

Of course, it is often reasonable to, in addition to the evaluation set $E$, also construct a training set, with regards to which to compare the elements in $E$. Naturally, with each element $e_i \in E$ should be associated one such set $R_{e_i}$, consisting of subsets of the context $C(e_i)$. Analogously with evaluation filters, \emph{training filters} can be introduced, which simplify the construction of such $R_{e_i}$. Since the context is a set of sets, these should have the form $\mathcal{F}(e_i): C(e_i) \mapsto R_{e_i} \subset 2^D$. As an example of a training filter, consider the sliding window training filter for sequences with length $w$ and step $s$:
\[
    \mathcal{F}_R(e_i) = \bigcup_{(c_1, c_2, \dots, c_n) \in C(e_i)}\{(c_1, c_2, \dots, c_w), (c_{s+1}, c_{s+2}, \dots, c_{s+w}), \dots, (c_{n-w}, c_{n-w+1}, \dots, c_{n})\}.
\]

A schematic illustration of the operation on a sequence of sliding window filters and a local context is shown in Figure~\ref{fig:filters}. Here, an evaluation set consisting of $19$ subsequences of a sequence of length $40$ is constructed. With each element $e_i \in E$ is associated a training set $R_{e_i}$ (as is seen in the figure, $R_{e_{10}} = 10$). An anomaly detection algorithm could compare each of the $e_i$ to the corresponding $R_{e_i}$ in turn in order to detect contextual collective anomalies (defined below).

\begin{figure}[htb]
    \begin{center}
        \includegraphics[width=1\textwidth]{resources/types_of_anomalies}
    \end{center}
    \caption{{\small Different types of anomalies in a real-valued continuous sequence. In the middle of each series is an aberration---shaded black---corresponding to a specific type of anomaly. Appropriate contexts for these anomalies are shaded dark grey, while items not part of the contexts are shaded light grey. The top panel contains a point anomaly---a point anomalous with regard to all other points in the series. The second panel contains a contextual anomaly---a point anomalous with regard to its context (in this case, the few points preceding and succeeding it), but not necessarily to the entire series. The third panel contains a collective anomaly---a subsequence anomalous with regard to the rest of the time series. The fourth contains a contextual collective anomaly---a subsequence anomalous with regard to its context.}}
\label{fig:anomaly_types}
\end{figure}

To simplify the discussion of contexts and evaluation/training sets, it is helpful to introduce a few different \emph{anomaly types}, based on which appropriate choices of evaluation and training filters can be inferred. For the purposes of the discussion in this report, an introduction of four anomaly types will suffice. In order of increasing generality, these are \emph{point anomalies}, \emph{contextual point anomalies}, \emph{collective anomalies}, and \emph{contextual collective anomalies}. An illustration of these anomaly types in the context of real-valued sequences is shown in Figure~\ref{fig:anomaly_types}.

\emph{Point anomalies} are arguably the simplest out of these anomaly types. They correspond to single points in the data set (i.e.\ $E$ consists of the singleton sets of $D$) that are considered anomalous with regard to the entire training set (i.e.\ a trivial context is appropriate). Finding them can be formulated as the following task:

\begin{task}[Point anomaly detection]
\label{task:point}
  Given a data set $D$, detect individual $d_i \in D$ that are anomalous with regard to the entire training set.
\end{task}

Point anomalies are often referred to as \emph{outliers} and arise in many domains~\cite{eskin}. Their detection is often relatively straightforward. Statistical methods have been shown to be well suited for handling point anomalies, and are usually sufficient.

Of course, point anomalies are not often not sufficient to describe all anomalies when $D$ admits a concept of locality. In this case, \emph{contextual point anomalies} can capture a more general class of anomalies. Contextual anomalies are individual items that are anomalous with regards to their context (as given by some non-trivial context function); i.e.\ while they might seem normal when compared with all elements in the training data, they are anomalous when compared to the other items in their context. Detecting contextual point anomalies can be formulated as the following task:

\begin{task}[Contextual anomaly detection]
\label{task:contextual}
  Given a set $D$ and a non-trivial context function $C(d)$, detect elements $d \in D$ anomalous with regards to $C(d)$.
\end{task}

Furthermore, detecting individual anomalous points $d \in D$ might not always suffice (specifically, when the anomalies are continuous), and \emph{collective anomalies} might be required to capture relevant phenomena. These correspond to contiguous sets of non-anomalous points that, when taken as a whole, are anomalous with regards to the entire training set (i.e.\ a trivial context is appropriate). The task of detecting such anomalies can be formulated using filters:

\begin{task}[Collective anomaly detection]
\label{task:collective}
  Given a set $D$ and a (non-singleton) filter $\mathcal{F}$, detect point anomalies in the evaluation set $\mathcal{F}(D)$.
\end{task}

Finally, \emph{contextual collective anomalies} are the most general class of anomalies, and correspond to contiguous sets of non-anomalous points that are anomalous with regard to a specific context but not to the entire training set.

\begin{task}[Contextual collective anomaly detection]
\label{task:contextual_power}
  Given a data set $D$, a (non-singular) filter $\mathcal{F}$ and a non-trivial context function $C$, detect elements $e_i$ in the evaluation set $\mathcal{F}(D)$ that are anomalous with regards to the context $C(e_i)$.
\end{task}

An illustration of the above concepts in real-valued sequences is shown in Figure~\ref{fig:anomaly_types}. Assuming that unsupervised anomaly detection is used, Task~\ref{task:point} amounts to disregarding the information provided by the ordering and detecting only `rare' items. While the task can capture the aberration in the first sequence in Figure~\ref{fig:anomaly_types}, none of the aberrations in the other sequences would be considered point anomalies.

While the value at the aberrant point at the center of the second sequence occurs elsewhere in that sequence, it is anomalous with regards to its local context, and as such, should be considered a contextual point anomaly and can be captured by Task~\ref{task:contextual}.

Since the third time series is continuous, the aberration present at its center can not be captured by Tasks~\ref{task:point} Or~\ref{task:contextual}. It is, however, a collective anomaly, and can be accurately captured by Task~\ref{task:collective}.

Finally, neither of the Tasks~\ref{task:point},~\ref{task:contextual} or~\ref{task:collective} captures the aberration in the fourth sequence, as it is both continuous and occurs elsewhere in the sequence. However, with an appropriate choice of (local) context, it can be deemed a contextual collective anomaly, and can be captured by Task~\ref{task:contextual_power}.

It should be noted that while all of the above tasks are special cases of contextual collective anomaly detection, it is often possible to reduce each of the tasks to detection of point anomalies, as well. Task~\ref{task:collective} reduces to detection of point anomalies by its definition (i.e.\ it corresponds to detecting point anomalies in a higher-dimensional space), while data normalization can sometimes be utilized to perform this reduction for contextual anomaly detection (see~\cite{meckesheimer} for instance).

\section{Anomaly measures}
\label{sect:anomaly_measures}

Arguably the most significant aspect of an anomaly detection method is what heuristic is used to decide if items are anomalous or not. This choice defines (often in unpredictable ways) what types of features will be considered anomalous, so it is vital to make an appropriate choice.

Many heuristics have been proposed in the literature, with varying degrees of justification and success. No exhaustive presentation of these is given here; instead a selection of some of the more common approaches is presented. We begin by describing statistical and information theoretic approaches---two areas which provide theoretic justifications for what should be considered anomalous. We then present a few other approaches taken from traditional machine learning.

Statistical measures usually address the following task:
\begin{task}[Statistical anomaly detection]
\label{task:statistical}
  Given a data set $D$, find items $d \in D$ that are unlikely to have been generated by the same distribution as the rest of $D$ (or some training set $R$).
\end{task}
In general, statistical measures work by estimating a statistical distribution underlying the data, and then labeling data points based on how likely they are to have been generated by this distribution. They have been applied to a wide range of domains, often with good results. Several books and surveys have been published on statistical anomaly detection, including~\cite{barnett},~\cite{bakar},~\cite{leroy} and~\cite{hawkins}.

Statistical measures are usually classified based on whether the distribution is known in advance (with unknown parameters) or not. We formulate these two cases as the following two tasks:
\begin{task}[Parametric statistical anomaly detection]
  Given a data set $D$, generated from the distribution $\mathcal{D}(\theta)$ with the unknown parameter $\theta$, estimate $\theta$ based on $D$ (or some training data set $R$), and find items $d \in D$ that are unlikely to have been generated by $\mathcal{D}(\theta)$.
\end{task}
\begin{task}[Non-parametric statistical anomaly detection]
  Given a data set $D$, estimate the distribution $\mathcal{D}$ of $D$ (or some training data set) from a set of basis functions and find items $d \in D$ that are unlikely to have been generated by $\mathcal{D}$.
\end{task}
While non-parametric approaches are more widely applicable (the distribution of data is usually not known), the extra information provided to parametric methods mean that they converge faster and are more accurate. Both these statistical tasks suffer from scalability issues. Furthermore, it can be difficult to modify statistical methods to take context into account.

A relatively novel and interesting approach to anomaly measures is \emph{information theoretic measures}. Mainly used for symbolic data sets, these measures judge similarity by estimating how much information is shared between items or subsets of items (i.e.\ by computing measures of shared information between elements). Like statistics, information theory can be used to give a theoretical justification for what is considered anomalous.

Several different measures of shared information have been implemented, such as the compressive-based dissimilarity measure (CDM)~\cite{keogh2} and (relative) conditional entropy~\cite{xiang}. While information theoretic approaches are mainly useful for discrete data, they have shown promise for describing anomalies in continuous data when combined with a discretization transformation~\cite{keogh2}. Information theoretical anomaly detection can be summarized as the following task:

\begin{task}[Information theoretic anomaly detection]
  Given a data set $D$, find items in $D$ that share little information with the rest of $D$ (or some training data set).
\end{task}

Anomaly measures inspired by traditional machine learning are also common and have been extensively researched in various contexts. We here present two tasks, using classifier-based and point-based measures, which together cover a majority of the methods proposed in the literature. We begin by introducing classifier-based measures:

\begin{task}[Classifier-based anomaly detection]
  Given a data set $D$, train a two-class classifier and use this classifier to determine whether or not the elements in $D$ are anomalous.
\end{task}

While classifiers are commonly used with Task~\ref{task:anomalous_set}, weighing schemes can be utilized to produce other forms of output. Depending on the type of training data, different types of classifiers must be used. Semi-supervised problems require one-class classifiers, while unsupervised methods require unsupervised classifiers. Furthermore, classifiers that do not support iterative addition and removal of elements can not be used for unsupervised anomaly detection.

Another common approach is distance-based heuristics:
\begin{task}[Distance-based anomaly detection]
  Given a data set $D \subset X$ for some set $X$ and a distance measure $\delta(d_i, d_j): X \times X \rightarrow \mathbb{R}$, determine if the $d_i \in D$ are anomalous or not based on $\delta(d_i, d_j)$ for $d_j \in D, i \neq j$.
\end{task}
Obviously, such measures are not always applicable, since they require $X$ to allow some form of useful distance function. Derived problem formulations often base their anomaly measures on the local point density of or the distances to the few nearest points in the training set around evaluated points. Such methods include \emph{k-nearest neighbors} (kNN) and, more generally, \emph{variable kernel density estimation}.

This task is very flexible, since any distance measure be used, and the distance values so obtained can be utilized in many ways. It is also well suited for both semi-supervised and unsupervised anomaly detection.

In some situations, it is useful to build a predictive model:
\begin{task}[Detecting anomalies based on a predictive model]
\label{task:predictive_model}
  Given a sequence $S$, build a model that predicts the next state based on the previous states. Determine if each $s_i \in S$ is an anomaly or not based on how much it diverges from the value predicted by the model.
\end{task}
Models that have been studied include Markov chains, hidden Markov models, autoregressive models, as well as several others.

While the above tasks cover a majority of methods in the literature, several heuristics not covered by them have been proposed. Furthermore, there is often overlap between these tasks. For instance, most predictive models used in Task~\ref{task:predictive_model} are based on statistical measures, and problems using such models could just as well be said to target Task~\ref{task:statistical}

\section{Solution format}
\label{sect:output_format}

As can be expected, the expected format of solutions to an anomaly detection problem affects both the difficulty of the analysis and the usefulness of its results. As such, the choice of this factor must depend on the target presentation format and performance requirements. In this section, the most common solution formats are presented. To this end, we will consider a set of possible anomalies $A$. Depending on the filter (see Section~\ref{sect:anomaly_types}) is used, each element in $A$ can either be a subset or an element of the original (input) dataset.

In many applications, relative anomaly rankings are not essential, and a list of anomalous elements might suffice. In such cases, specifying a \emph{set of anomalous elements} as the output format might be appropriate. Essentially, this output format turns the problem of anomaly detection into a two-class classification problem, so techniques traditionally used for classification (such as support vector machines) are naturally well suited to it.

Producing a set of anomalous elements is typically appropriate when the goal of the analysis is to highlight potential anomalies to an analyst. It can also be problematic, however, since it requires some sort of threshold to be used to determine whether items are marked as anomalous or not.

A solution format that has received a lot of attention in recent years (\cite{keogh1},~\cite{bu},~\cite{yankov},~\cite{fu},~\cite{lin}) is \emph{discords}, or the set of the $k$ most anomalous elements in $A$. Since discords are less computationally intensive to produce than other solution formats, they are appropriate for analysis of very large data sets. However, discords might not be appropriate for anomaly detection in large amounts of small datasets, or for monitoring applications.

The most general solution format is to associate \emph{anomaly scores} with the elements of $A$. Typically, anomaly scores are positive real-valued numbers, and higher anomaly scores signify more anomalous elements. Of course, the fact that anomaly scores outputs have size $|A|$ means that they are comparatively computationally intensive to produce, especially if the size of $A$ is larger than the size of the original dataset. If this is the case, anomaly scores can be produced for the elements of the input dataset by weighing the anomaly scores for $A$.

While the anomaly scores of non-anomalous elements might be of little interest, producing them can still be useful for visualisation purposes. Furthermore, both of the previously mentioned solution formats can be constructed from anomaly scores, which makes anomaly scores especially useful when the output requirements are not clear.

\section{Related high-level tasks}
\label{sect:related_tasks}

Anomaly detection is closely related to several other problems in data mining, machine learning and signal processing. We here briefly present a few high-level tasks, analogous to Task~\ref{task:anomaly_detection}, for a few such problems.

There are multiple related problems in signal processing that involve the detection and removal of anomalies in data. Typically referred to as \emph{noise removal}, \emph{data cleaning}~\cite{meckesheimer}, or \emph{signal recovery}, these problems are summarized in the following task:
\begin{task}[Signal detection]
  Given a data set $D$ consisting of a relevant signal with added noise, remove all anomalies caused by the noise.
\end{task}

\emph{Novelty detection}~\cite{chandola} refers to the detection of novel, or previously unseen, items or subsequences in a sequence~\footnote{It should be noted that the term `novelty detection' is occasionally used in the literature to refer to semi-supervised anomaly detection.}, as formalized by the following task:
\begin{task}[Novelty detection]
\label{task:novelty_detection}
  Given a sequence $S$, detect subsequences $(s_i, s_{i+1}, \dots, s_j) \subset S$ that are anomalous with regards to earlier subsequences (i.e. $(s_i, s_{i+1}, \dots, s_j) \subset S$ for $k < i$ and $l < j$).
\end{task}
Most of the high-level tasks presented in this chapter can be combined with novelty detection, which is reasonable considering that it can be reduced to Task~\ref{task:contextual_power} through the choice of an appropriate context function. It is especially interesting when used in monitoring applications and in the context of long sequences.

\emph{Change detection} (or \emph{event detection}) is the problem of detecting points at which sequences change in some way.
\begin{task}[Change detection]
      Given a sequence $S$, detect points at which the underlying distribution or model generating the $s_i$ changes.
\end{task}
Change detection is almost exclusively associated with time series, and has been used mainly in the study of climate data~\cite{gopala} and image analysis~\cite{radke}.

Other tasks that are often interesting in the same contexts as anomaly detection include \emph{clustering}~\cite{clustering} and \emph{motif discovery}~\cite{motif}~\cite{motif2}.


\chapter{Presentation}
\label{ch:transformations}

The \emph{presentation} of data is an important aspect of any anomaly detection problem. When talking about presentation, we refer to the structure of the data set presented to the anomaly detection algorithm. Real-world data sets can typically be presented in a myriad of different ways, and the choice of presentation has a great impact both on the difficulty of the analysis and on the set of applicable tasks.

For large data sets, only selected subsets are typically used for analysis. Furthermore, data sets are often aggregated, compressed or otherwise subjected to \emph{transformations} to convert them into a more manageable form before analysis. Transformations are typically applied to simplify or speed up the analysis, but they also have an impact on what types of anomalies can be detected.

In this chapter, a thorough overview is given of various aspects of the presentation of temporal machine data and how they affect the analysis thereof.

As previously, we adhere to the framework of tasks and problems and present tasks where appropriate. However, we do not try to be as exhaustive as in the previous chapter; instead we briefly introduce various presentations and transformations, and avoid presenting tasks for presentations and transformations not used in the remainder of the report.

We begin in section \ref{sect:splunk_techniques} by discussing the typical structure of data in the target domain (large databases of temporal machine data). A focus is placed on sequences, and two high-level tasks that cover anomaly detection in the target domain are presented.

In section \ref{sect:terminology}, some terminology regarding time series and sequences is presented.

In sections \ref{sect:unimultivariate}, \ref{sect:aggregation} and \ref{sect:compression}, we discuss various transformations and aspects of the presentation and how they impact the analysis.

Finally, in section \ref{sect:series_mining} we discuss the automated extraction of interesting univariate sequences in data sets consisting of long sequences. 

\section{Presentation of temporal machine data}
\label{sect:splunk_techniques}

Unlike those found in most application domains, the data sets encountered in temporal machine data applications are typically very large, unstructured, and contain large amounts of non-pertinent information. For this reason, culling and aggregating the data before analysis is essential, and selecting a presentation is an important part of constructing problem formulations.

Machine data is almost exclusively recorded and stored as strings (corresponding to events) of plain-text data (especially in Splunk). Each of these strings (or events), in turn, encodes a data point of one- or multidimensional, usually mixed, data. Obviously, letting anomaly detection methods handle large amounts of such strings directly is not optimal, and the data must given an alternative presentation before it can be analyzed.

Fortunately, Splunk and other databases targeted at machine data can parse such events automatically and extract the interesting \emph{fields} (i.e. individual data dimensions). In a sense, a data set of temporal machine data that has gone through such a field-extraction process can be seen as a long sequence $S$ of multi-dimensional data with associated time stamps
\[
  S = (s_1, s_2, \dots, s_n), \quad \text{ where } \quad \forall i: s_i = (s_i, t_i) \in X \times T \quad \text{ and } \quad \forall i,j: i > j \Rightarrow t_i > t_j
\]
where $X$ is the Cartesian product of all fields present in any data item and $T$ is a set of associated time stamps.

Of course, analysing such databases in their entirety is often costly and unnecessary: a majority of events are likely to be unrelated or non-pertinent. Instead, databases are typically split into multiple \emph{indexes} containing related events. While limiting the analysis to an individual index will typically simplify the analysis, we can not assume that the items in an index are sufficiently homogeneous to warrant analysing the index as a whole. Instead, we need to complement the analysis with methods for extracting interesting series, something we discuss in section \ref{sect:series_mining}. Nevertheless, we can discern the following major anomaly detection tasks for temporal data:

\begin{task}[Detecting anomalous subsequences]
  \label{task:ad_splunk}
  Given a sequence $S = (s_1, s_2, \dots)$ where $\forall s \in S: s \in X \times T$ (for some mixed-data set $X$ and the set of time stamps $T \subset \mathbb{R}$) and $\forall (s_i, t_i), (s_j, t_j) \in S: i > j \Rightarrow t_i \geq t_j$, detect subsequences $S' \subset S$ that are anomalous. 
\end{task}
\begin{task}[Detecting anomalous sequences in a set of sequences]
  \label{task:ad_splunk2}
  Given a set of sequences\\ $\{S_1, S_2, \dots, S_n\}$ where $S_i = (s_{i,1}, s_{i,2}, \dots, s_{i, n_i})$ and $\forall s_j \in S_i: s_j \in X \times T$ (for some mixed-data set $X$ and the set of time stamps $T \subset \mathbb{R}$) and $\forall (s_i, t_i), (s_j, t_j) \in S: i > j \Rightarrow t_i \geq t_j$, detect sequences $S_i$ that are anomalous.
\end{task}

These are the most general tasks we can specify, in the sense that any task specifying additional factors must exclude at least some problems that could be relevant. Thus, before more interesting tasks and problems can be formulated, the relative values of the specific principal factor choices must be estimated. This is done in chapter \ref{ch:problems}.

Note that task \ref{task:ad_splunk2} is mainly interesting when the data set consists of large amounts of similar sequences, for instance when analyzing user command records (as in figure \ref{fig:calls}). While this task is relevant in many applications, it is not as generally applicable as \ref{task:ad_splunk} to large databases of temporal machine data. While such databases will always contain plenty of long sequences that are interesting to study individually, there is no guarantee that they contain enough similar short sequences to warrant the application of \ref{task:ad_splunk2}. We here simply note that many papers have been published on task \ref{task:ad_splunk2} (\cite{blender}, \cite{chan}, \cite{ye}, \cite{forrest}, \cite{sekar1}, \cite{sekar2}), and refer to \cite{chandola2} and \cite{chandola3} for more thorough reviews.

Finally, it should be noted that the tasks \ref{task:ad_splunk} and \ref{task:ad_splunk2} are related, in the sense that a problem of finding anomalous subsequences of some long sequence $S$ can be reduced to the problem of finding anomalous sequences in a set of sequences by extracting and comparing all subsequences (or all subsequences of some specific length) of $S$.

\section{Terminology}
\label{sect:terminology}

Since we have assumed that the data sets we are dealing with consist of discrete events with associated time stamps, we now turn our focus to sequences and time series, and their relation to anomaly detection. Since various incompatible definitions of sequences and time series are used in the literature, we begin by defining these concepts.

From here on, we take a \emph{sequence} to mean a progression (finite, or in the case of monitoring applications, infinite) $S = (s_1, s_2, \dots)$, where $\forall i: s_i \in X$ for some set $X$. Furthermore, we will take a \emph{time series} to be any sequence $T = ((s_1, t_1), (s_2, t_2), \dots)$, where $\forall i: (s_i, t_i) \in X \times \mathbb{R}^+$ for some set $X$ and $\forall i, j: i > j \rightarrow t_i \geq t_j$. In other words, a time series is any sequence in which each item is associated with a point in time. We refer to sequences and time series as symbolic/categorical, discrete, real-valued, vector-valued et cetera based on the characteristics of $X$.

When the time series is sampled at regular intervals in time (i.e. $\forall i, j: t_{i+1} - t_i = t_{j+1} - t_j$), we say that the time series is \emph{regular}. We will treat regular time series as sequences, suppressing the $t_i$ and writing $T = (s_1, s_2, \dots)$. Henceforth we will take all time series to be equidistant unless explicitly stated\footnote{As mentioned above, there is some confusion surrounding these terms in the literature. Specifically, what we would refer to as `symbolic sequences' and `regular time series' are often simply referred to as `sequences' or `time series'. In such contexts, other types of sequences (and time series) are usually ignored}.

\section{Aggregation}
\label{sect:aggregation}

\begin{figure}[htb]
    \begin{center}
        %\leavevmode
        \begin{tabular}{| l | l l l l l l l l |}
            \hline
            $\mathbf{S_1}$ & login & passwd & mail & ssh & \dots & mail & web & logout \\ \hline
            $\mathbf{S_2}$ & login & passwd & mail & web & \dots & web & web & logout \\ \hline
            $\mathbf{S_3}$ & login & passwd & mail & ssh & \dots & web & web & logout \\ \hline
            $\mathbf{S_4}$ & login & passwd & web & mail & \dots & web & mail & logout \\ \hline
            $\mathbf{S_5}$ & login & passwd & login & passwd & login & passwd & \dots & logout \\\hline
        \end{tabular}
    \end{center}
    \caption{{\small Multiple symbolic sequences consisting of user commands. In this context, anomaly detection tasks involving finding individual anomalous sequences are interesting. Arguably, problems based on such tasks should capture $\mathbf{S_5}$ as an anomaly due to its divergence from the other sequences. Based on a figure in \cite{chandola2}.}}
    \label{fig:calls}
\end{figure}

In general, sequences are much easier to deal with algorithmically than irregular time series. For this reason, irregular time series are commonly aggregated to form regular time series, which can be treated as sequences. Different aggregations are used depending on the nature of the underlying data. We now present a few aggregations useful in the context of Splunk.

As indicated in task \ref{task:ad_splunk}, we assume that the data we are interested consists of mixed-data time series. Depending on the context in which the data is generated, three main potentially interesting tasks can be discern, involving different transformations of the series. Firstly, we might be interested only in the relative ordering of elements. Secondly, we might be interested in the frequency of elements. Finally, we might want to approximate some (continuous) function we assume the series samples. We begin by formalizing the first and simplest case:

\begin{task}[Sequential anomaly detection]
  \label{task:sequential_anomaly_detection}
  Given a sequence or collection of sequences $S$, detect elements or collections of elements in $S$ that are anomalous by considering their order.
\end{task}

This task is well suited for symbolic data, since such data can generally not be aggregated (due to its lack of structure). This task is relatively well understood - an in-depth review is given in \cite{chandola2}. An example of the task is shown in figure \ref{fig:calls}. It is not suited for the second case (since it ignores frequency information) or the third case (since it will give a bad approximation if the series is irregular) above. For the second case, the following task is more appropriate:

\begin{task}[Frequential anomaly detection]
  \label{task:frequential_anomaly_detection}
  Given a sequence or collection of sequences $S$, detect elements or collections of elements in $S$ that are anomalous by considering how often they occur.
\end{task}

As an example of this task, consider a series corresponding to the times at which a certain error or warning message appears on some system. In such cases, how the relative frequency of the message evolves over time is the main point of interest---not the structure of each message (which might be always identical).

Several methods can be used to generate sequences appropriate for this task from time series, such as histograms, sliding averages, etc. We can generalize these as the following transformation:

Given a time series $T = ((s_1, t_1), (s_2, t_2), \dots)$ where $\forall i: s_i \in X$, with associated weights $w_i$ and some envelope function $e(s, t): X \times \mathbb{R} \rightarrow X$ as well as a spacing and offset $\Delta, t_0 \in \mathbb{R}^+$, a sequence $S' = ((s_{1}^{'}, \tau_1), (s_{2}^{'}, \tau_2), \dots)$ is constructed, where $\tau_i = t_0 + \Delta \cdot i$ and $s_{i}^{'} = \sum_{(s_j, t_j) \in S} s_i w_i e(t_j - \tau_i)$.

The $\tau_i$ can then be discarded and the regular time series treated as a sequence. Histograms are recovered if $e(s, t) = 1$ when $|t| < \Delta/2$ and $e(x, t) = 0$ otherwise. Note that this method requires multiplication and addition to be defined for $X$, and is thus not applicable to most symbolic/categorical data. Also note that $S'$ is really just a sequence of samples of the convolution $f_S \ast e$ where $f_S = \sum_i \delta(t_i) s_i w_i$.

How this aggregation is performed has a large and often poorly understood impact on the resulting sequence. As an example, when constructing histograms, the bin width and offset have implications for the speed and accuracy of the analysis. A small bin width leads to both small features and noise being more pronounced, while a large bin width might obscure smaller features. Similarly, the offset can greatly affect the appearance of the histograms, especially if the bin width is large. There is no `optimal' way to select these parameters, and various rules of thumb are typically used \cite{density_estimation}.

Finally, the third case outlined at the start of this section can be summarized by the following task: 

\begin{task}[Continuous anomaly detection]
  \label{task:continuous_anomaly_detection}
  Given a sequence or collection of sequences $S$, detect elements or collections of elements in $S$ that indicate anomalies in the process the elements in $S$ are sampled from.
\end{task}

This task can be treated in essentially the same way as the previous task. The convolution operation must be combined with a normalization step to account for the point density. It can also be beneficial to replace $f_S$ with an interpolation of the points in $S$. Moving average techniques are a common special case of this method.

Note that all of the above transformations involve a loss of data. They are used because they simplify the analysis without (hopefully) losing information pertinent to the anomaly detector.

\section{Uni- and multivariate data}
\label{sect:unimultivariate}

\begin{wrapfigure}{r}{0.5\textwidth}
    \vspace{-25pt}
    \begin{center}
        \leavevmode
        \includegraphics[trim = 20mm 3mm 20mm 5mm, clip, width=0.5\textwidth]{resources/multi_vs_univariate}
    \end{center}
    \vspace{-15pt}
    \caption{{\small Two sine curves regarded as two separate univariate time series (dotted lines) and as one multivariate time series (solid lines).}}
    \label{fig:multi_vs_univariate}
    \vspace{-25pt}
\end{wrapfigure}

The dimensionality of the individual elements in a sequence has a profound impact on the difficulty and efficiency of the analysis. Representing a sequence by $S = (s_1, s_2, \dots)$, where $s_i \in X$ for some set $X$, we say that $S$ is \emph{multivariate} if $X$ is the Cartesian product of some collection of non-empty sets $X_i$, and that $S$ is \emph{univariate} otherwise. The difference between uni- and multivariate time series is illustrated in figure \ref{fig:multi_vs_univariate}.

In a sense, univariate and multivariate sequences are two sides of the same coin. Obviously, any multivariate sequence $S = (s_1, s_2, \dots)$, where $s_i = (s_{i,1}, s_{i,2}, \dots, s_{i,n})$ may be split into multiple univariate time sequences $S_1, S_2, \dots, S_n$, where $S_i = (s_{1,i}, s_{2, i}, \dots ).$ Conversely, any $S_1, S_2, \dots, S_n$ can be aggregated into a multivariate time series (although this is meaningless unless the $S_i$ were all sampled at the same times).

Multivariate time series are arguably more powerful, in the sense that a data point $(\vect{s_i}, t_i)$ where $\vect{s_i} = (s_{i,1}, s_{i,2}, \dots, s_{i,n})$ may be anomalous with regard to the rest of the data, while its individual components $s_{i,j}$ are not, as illustrated in figure \ref{fig:dimensionality_reduction}.

On the other hand, analyzing higher-dimensional data is often difficult (in terms of both analysis time and accuracy), making multivariate analysis both computationally and conceptually more difficult. For this reason, it might be beneficial to (when safe) split multivariate sequences or otherwise reduce their dimensionality before performing the analysis. 

A majority of the literature on anomaly detection in sequences, and especially in real-valued time series, focuses on univariate data, and while some algorithms can be extended to handle multivariate time series, this is not always possible. Keeping this in mind, it makes more sense to focus on univariate sequences, at least initially.

\section{Compression}
\label{sect:compression}

\begin{figure}[htb]
\begin{center}
\leavevmode
\includegraphics[width=\textwidth]{resources/multivariate}
\end{center}
\caption{\small{An example of dimensionality reduction in a point anomaly detection problem in $R^2$. The left figure shows a set of $500$ data points $(x_i, y_i)$ containing one anomaly. The top right figure shows a histogram of the $x_i$, while the bottom right figure shows a histogram of the distance from the center point. In each figure, the location of the anomalous point is marked by an arrow. While the anomaly is easy to detect in the left and bottom right figures, it can not be seen in the top right figure. This is due to the linear inseparability of the data, and illustrates how dimensionality reduction can lead to information losses if not performed properly.}}
\label{fig:dimensionality_reduction}
\end{figure}

Analyzing high-dimensional data can be difficult, both due to increased computational time and due to topological properties of high-dimensional spaces (a phenomenon commonly referred to as the \emph{curse of dimensionality}). In many analysis tasks, it is therefore desirable to (if possible) reduce the dimensionality of data in such a way that only the aspects of the data most pertinent to the analysis remain.

Many techniques have been designed with this goal in mind. A distinction is typically made between \emph{feature selection} and \emph{feature extraction} approaches. Feature selection approaches try to select a subset of the dimensions present in the original data. Feature extraction approaches, in contrast, transform the data into some new space, in which the relevant features are hopefully more apparent. Feature extraction methods are commonly employed as a pre-processing step in anomaly detection algorithms, while feature selection methods are not commonly used. We will here focus on feature extraction methods.

Common feature extraction methods for data in $\mathbb{R}^n$ include \emph{principle component analysis} (PCA), \emph{semidefinite embedding}, \emph{partial least-squares regression}, and \emph{independent component analysis}. Which methods are appropriate to use depends heavily on the problem domain. In order to illustrate the concept of feature extraction, we will now briefly present a typical example of $\mathbb{R}^n$ feature extraction, before moving on to methods designed specifically for sequences (and time series in particular).

%Consider the two-dimensional data set in the left panel of figure \ref{fig:multi_vs_univariate}. Here, all data points are generated from the same distribution except for a single anomalous point at the center of the panel. While the anomaly is readily identified in the left panel, due the radial symmetry of the distribution, it can not be seen when projected linearly onto any direction, as illustrated in the upper right panel. The task in dimensionality reduction is to present a projection onto a lower dimensional space (in this case $\mathbb{R}$) in which the classes (anomalous/non-anomalous) are separated, something that is surprisingly difficult even for this relatively simple data set. Indeed, most dimensionality reduction methods, including all feature selection methods and some feature extraction methods (such as linear PCA) can not accomplish this on the given data - non-linear feature extraction methods are required. This illustrates the difficulty inherent to dimensionality reduction.

\begin{figure}[htb]
  \begin{center}
    \leavevmode
    \includegraphics[width=\textwidth]{resources/types_of_data}
  \end{center}
  \caption{\small{Illustration of numerosity and dimensionality reduction in a conversion of a real-valued sequence to a symbolic sequence. The top frame shows a real-valued time series sampled from a random walk. The second frame shows the resulting series after a (piecewise constant) dimensionality reduction has been performed. In the third frame, the series from the second frame has been numerosity-reduced through rounding. The bottom frame shows how a conversion to a symbolic sequence might work; the elements from the third series is mapped to the set $(a,b,c,d,e,f)$.}}
  \label{fig:types_of_data}
\end{figure}

When it comes to sequences, each individual component of a sequence is treated as an independent dimension, complicating the problem of dimensionality reduction significantly and rendering most ordinary methods intractable. Luckily, the temporal ordering of time series can often be used to efficiently perform feature extraction despite the high-dimensional data of the series. The literature on dimensionality reduction in sequences has mainly focused on real-valued sequences. In this context, the task of feature extraction can be rephrased as follows: \emph{Given a sequence $S = s_1, s_2, \dots, s_n$ where $s_i \in \mathbb{R}$, find a set of basis functions $\phi_1, \phi_2, \dots, \phi_n$ where $m < n$ that $T$ can be projected onto, such that the analysis is simplified and $T$ can be recovered with little error.} Many different methods for obtaining such bases have been proposed, including the discrete Fourier transform \cite{faloutsos1}, discrete wavelet transforms \cite{pong}, various piecewise linear and piecewise constant functions \cite{keogh3} \cite{geurts}, and singular value decomposition \cite{keogh3}. An overview of different representations is provided in \cite{fabian}.

Arguably the simplest time series representation involves using a basis of piecewise constant functions $(\phi_1, \phi_2, \dots, \phi_n)$:
\[
  \phi_i(t) = \left\{
    \begin{array}{l l}
      1 & \quad \text{ if } \tau_i < t < \tau_{i+1} \\
      0 & \quad \text{ otherwise.} \\
    \end{array} \right.
\]
where $(\tau_1, \tau_2, \dots \tau_n)$ is a partition of $[t_1, t_n]$. The $s_i$ are then aggregated over the $\phi_i$ as in section \ref{sect:aggregation}. 

Different piecewise constant representations have been proposed, corresponding to different partitions. The simplest of these, corresponding to a partition with constant $\tau_{i+1} - \tau_i$ is proposed in \cite{keogh4} and \cite{faloutsos2} and is usually referred to as \emph{piecewise aggregate approximation (PAA)}. As shown in \cite{keogh5}, \cite{keogh3} and \cite{faloutsos2}, PAA rivals the more sophisticated representations listed above. 

Additionally, reducing the cardinality of the underlying set can be useful. This is often referred to as \emph{numerosity reduction} and corresponds to discretizing real-valued data or reducing the resolution of discrete data. This is primarily done to give real-valued time series symbolic representations, which allows for symbolic sequence analysis methods to be applied to them. The most widely used symbolic representation is \emph{symbolic aggregate approximation} (SAX) \cite{sax}.

\section{Sequence extraction}
\label{sect:series_mining}

In light of the discussion of uni- and multivariate data above, it seems reasonable to emphasize extracting univariate series from the long multivariate sequences encountered in temporal machine data sets (i.e. task \ref{task:ad_splunk}). Indeed, the provision of tools for manually or automatically extracting interesting time series is likely necessary for the uptake of large-scale anomaly detection (and indeed several other related tasks) in databases such as Splunk. Ideally, methods should be available that automatically examine all potentially pertinent time series and report on those in which anomalies are found. However, the total number of such time series is likely to grow quickly with the number of fields and distinct values of categorical variables in the data data sets, possibly rendering such methods intractable.

In order to estimate the feasibility of these methods, it can be helpful to consider the size of the search space---i.e., the number of possible interesting time series that can be constructed from a typical Splunk index. For example, consider an index consisting of web logs for a single web server. Assume that a log message is generated for every new HTTP request, containing the following fields: time stamp; host name and ident of the user; HTTP request type, URL, status and request/response size/time/latency; and the user agent and referrer. All of these factors are categorical, apart from the HTTP request/response size/time/latency, which ought to be considered real-valued. Let us denote the number of categories encountered in each categorical factor as $n_1, n_2, \dots, n_k$. For most websites, each of the $n_i$ ought to lie between a few hundred and a few million per day.

We first consider the size of the set of potentially interesting univariate sequences. Interesting univariate sequences could be constructed from at least the frequencies of every value of every categorical factor; the continuous factors over time; or any factor conditioned by any single value of any categorical factor (such as the sequence of URLs or the latency over time for any single user). These combinations alone lead to a number of $N = \sum_i n_i (1 + k + \sum_{j \neq i} n_j)$ potentially interesting time series. If the $n_i$ are sufficiently small, the quadratic nature of this expression is workable. However, if the $n_i$ are large (as is the case in our example, and is likely to be the case in most Splunk applications), a very large number of time series would have to be extracted and analyzed. Of course, detecting anomalies in series conditioned by two or more variables could also be interesting - causing the addition of higher-degree terms to the expression of $N$. While linearity could be achieved by disregarding conditioned series, this would lead to a significant reduction in analysis power.

Considering this, and the fact that anomaly detection is typically a computationally expensive task (especially when performed on the large data sets), completely automated series extraction is not likely to be feasible except on small indexes. However, if combined with manual pruning of the search space, automated extraction might be feasible. For instance, if in our example we were only interested in detecting anomalous user patterns, the analysis could be limited to a number of $n_u (k + \sum_{i \neq u} n_i)$ sequences (where $n_u$ is the number of distinct users), or perhaps as few as $n_u \cdot k$.

Developing the tools to perform this type of extraction and analysis in Splunk should prove relatively straightforward, and this would be the logical next step once appropriate anomaly detection methods have been implemented.

\chapter{Important tasks}
\label{ch:problems}

Having discussed principal factors and transformations in depth, we are finally in a position to reason about which tasks might be the most relevant to anomaly detection in the target domain.

This chapter begins in section \ref{sect:goals} with a presentation and discussion of objectives in designing anomaly detection methods for machine data.

In section \ref{sect:relevant_tasks}, the principal factors and high level tasks presented in chapter \ref{ch:tasks} are considered, from the perspective of their applicability to temporal machine data and how well they match the chosen objectives.

Finally, the chapter is concluded in section \ref{sect:suggested_tasks} with a summary of the anomaly detection tasks most relevant to the target domain.

\section{Objectives}
\label{sect:goals}

A set of objectives for anomaly detection methods dealing with temporal machine data are now presented, which will be used to reason about which tasks and methods are appropriate for such data.

The most fundamental objective is \emph{accuracy}. Methods must detect anomalies as accurately as possible. For obvious reasons, a high rate of false negatives will render the analysis useless, while a high false positive rate will require a high degree of manual analysis of the results---something that is is problematic since such an analysis might be hard to perform and can be sensitive to bias on part of the analyst. This objective can be stated as follows:

  \emph{Methods should be as accurate as possible. They should have low misclassification rates.}

Another important property is \emph{generality}. Many methods are accurate only on few sets of data. For instance, parametric statistical methods typically rely on the assumption that the data is sampled from a specific distribution. When this assumption fails, the results are usually poor. As the target domain is arbitrary sets of machine data, methods should require as few assumptions as possible. This objective is summarized as follows:

  \emph{Methods should perform well on as many data sets as possible. They should make as few assumptions as possible about the data.}

A related objective is that methods should have as \emph{few parameters} as possible. The goal is to provide tools that facilitate the monitoring and analysis of very large sets of data, and methods that require a lot of parameter tweaking in order to work properly will likely hamper this effort, since appropriately configuring such methods requires knowledge of the intricacies of both methods and data. Furthermore, methods with many parameters can easily lead to misinterpreted results \cite{keogh2}. Finally, the amount of `peripheral' configuration (selection of training data, output format, etc.) should be kept to a minimum. This leads to the following objective:

  \emph{Methods should require as few parameters as possible. Configuration should be kept to a minimum.}

Finally, it is vital that methods be \emph{scalable}. Since the hope is to provide methods that can be useful in monitoring and diagnosis scenarios, methods that can handle large sets of data are required. Specifically, this means that methods should have low complexity in terms of both time and memory, and be able to give accurate results even for large data sets. Furthermore, it is preferable for methods can be distributable and work in a streaming fashion, e.g. not require random disk accesses or multiple passes over the data. This objective can be stated as follows:

  \emph{Methods should be scalable: they should be able to handle very large amounts of data without long computation times or disk issues.}

\section{High-level tasks}
\label{sect:relevant_tasks}

Which high level tasks are relevant for temporal machine data is now discussed. The principal factors are treated sequentially, according to the order they were encountered in chapter \ref{ch:tasks}.

This section is also based in large part on the discussions in the previous section and those of chapter \ref{ch:transformations}. 

\subsection{Data format}
\label{sect:relevant_data_format}

As per the discussion in sections \ref{sect:splunk_techniques} and \ref{sect:series_mining}, it is appropriate to focus on extracting sequences from indexes for use with tasks \ref{task:ad_splunk} or \ref{task:ad_splunk2}.

What remains is to settle on a specific presentation of these sequences. While all three aggregation tasks presented in section \ref{sect:aggregation} are potentially useful for temporal machine data, their relative utility to the stated objectives can still be judged. We will now discuss each of the three tasks in turn, and estimate their utility.

First, consider task \ref{task:sequential_anomaly_detection} (sequential anomaly detection). While interesting in certain contexts (such as system call or web session traces), it is not likely to be able to capture a majority of interesting sequences in temporal machine data sets, since it is not appropriate for non-categorical sequences. While there are certainly interesting categorical sequences that can be analyzed in many temporal machine data sets, it can be expected that these constitute a low share of the total number of interesting sequences in Splunk applications. For this reason, it ought to be argued that this task, while interesting in some cases, should have lower priority than the other two aggregation tasks.

Task \ref{task:frequential_anomaly_detection} (frequential anomaly detection) is more interesting. Since it enables anomaly detection to be applied to the frequency of any recurring event or value, it should be applicable to a wider range of data sets. However, it is limited in the sense that it can not accurately capture continuous quantities or other types of quantities that vary over time. This limits its applicability to temporal machine data sets, since a large share of the interesting quantities in such data (at least in Splunk applications) vary over time.

In such cases, task \ref{task:continuous_anomaly_detection} is more appropriate, as it allows for correct estimates of evolving quantities. As pointed out in section \ref{sect:aggregation}, the aggregation transformations involved in the tasks \ref{task:frequential_anomaly_detection} and \ref{task:continuous_anomaly_detection} are very similar, and implementing them both should not be difficult. However, they are likely to require different anomaly measures; to limit the scope of the project, it was decided that the focus would be placed on the latter.

The next issue to settle is whether uni- or multivariate sequences should be the focus. As discussed in section \ref{sect:unimultivariate} multivariate sequences are more difficult to analyze than univariate sequences, and can often be broken down into univariate sequences without significant losses in analysis power. Therefore, it is reasonable to (at least initially) focus solely on univariate sequences.

Finally, it must be decided which of the tasks \ref{task:ad_splunk} (detecting anomalous subsequences) and \ref{task:ad_splunk} (detecting anomalous sequences in a set of sequences) to focus on. While the latter task is interesting and occurs naturally in many contexts, e.g. system call traces (figure \ref{fig:calls}), it is more reasonable to focus on the former since temporal machine data sets typically feature long and largely dissimilar sequences.

Based on the above, the following two tasks ought to be promoted as important, with the latter being preferential as a first step:
\begin{task}
  \label{task:frequency}
  Detect anomalous subsequences in real-valued univariate sequences containing the frequency of some event over time.
\end{task}
\begin{task}
  \label{task:continuous}
  Detect anomalous subsequences in real-valued univariate sequences sampled from a continuous process.
\end{task}

\subsection{Reference data}

One of the four objectives stated in section \ref{sect:goals} is that methods should be as generally applicable and require as little configuration as possible. Obviously, the reference data tasks presented in section \ref{sect:reference_data} meet these demands to varying degrees. Since unsupervised anomaly detection (task \ref{task:unsupervised_anomaly_detection}) does not require the user to provide reference data, it is superior to supervised and semi-supervised anomaly detection in this regard.

However, it is not clear whether unsupervised methods can scale adequately; there seems to be a trade-off between the amount of required supervision and the scalability of algorithms. Most naive methods scale as $O(f(R \cdot N))$, where $f(n) \in \Omega(n)$, $R$ is the number of reference items, and $N$ is the number of items to be evaluated. When performing semi-supervised anomaly detection, it is usually the case that $R < N$. In unsupervised anomaly detection, however, $N = R$, and so methods scale as $O(f(N^2))$. This is problematic, and some form of pruning is required to allow large-scale analysis. One form of pruning that should perform relatively well is the use of some form of constant-size context (e.g., the local context). Nevertheless, it was decided that unsupervised methods should be the focus of this project.

\subsection{Anomaly types}

The choice of both anomaly types and anomaly measure together define which anomalies will be captured by problem formulations.

Reviewing section \ref{sect:anomaly_types}, and especially figure \ref{fig:anomaly_types}, can help when reasoning about anomaly types. \emph{Point anomalies} are obviously not appropriate, as they can not capture most anomalies in continuous sequences (to see this, consider that none of the aberrations in the second, third, or fourth panels of figure \ref{fig:anomaly_types} are point anomalies).

\emph{Contextual point anomalies} are more relevant, especially in the context of task \ref{task:frequency}. Here, each individual data point in the resulting data series encodes the frequency of some event at some specific time. Assuming that these frequencies among points sharing context are independent, contextual anomaly methods should be sufficient to find temporary anomalous frequencies.

However, contextual point anomalies do not occur in continuous sequences, such as those encountered in task \ref{task:continuous} or in the third and fourth panels of figure \ref{fig:anomaly_types}.

\emph{Collective anomalies}, on the other hand, are very interesting in continuous time series, since they can cover all anomalies that arise in such series, as long as the anomalies are global (i.e., they do not constitute normal behaviour elsewhere in the series).

However, non-global anomalies do occur, especially in long time series such as those encountered in monitoring applications. With this in mind, it is reasonable to conclude that being able to take \emph{contextual collective anomalies} into account is highly desirable. Further, based on the fact that all the previous anomaly types are special cases of contextual collective anomalies, it can be concluded that finding this type of anomaly should be sufficient\footnote{Note that while this is certainly true, it does not mean that methods optimized for finding contextual anomalies (for instance) would not be better at that specific task than methods for finding contextual collective anomalies. A thorough evaluation of all methods---for both tasks---would have to be performed to answer this question}.

As with any task involving contextual anomalies, a \emph{context function} must be specified before this method can be useful. Natural choices of sequence context functions include the \emph{symmetric local context}, which considers $k$ items before and after the subsequence in question, and the \emph{asymmetric local context}, which considers $k$ items before and $j$ items after the subsequence.

Since which context functions are appropriate depends heavily on the data set, context functions should be evaluated empirically.

\subsection{Anomaly measures}

As stressed in section \ref{sect:anomaly_measures}, the anomaly measure is probably the most important part of any anomaly detection problem. However, due to the unpredictability of how they affect results, there is not much that can said about which anomaly measures are the most appropriate for the target domain. For this reason, we will leave this factor unspecified, and instead focus on evaluating as many different choices of it as possible.

\subsection{Output format}

In one sense, the choice of output format is cosmetic, as it depends entirely on the context in which the results should be presented. 

However, as noted in section \ref{sect:output_format}, most other output format tasks can be reduced to task \ref{task:anomaly_scores} (anomaly scores). In order to reduce the risk of focusing on an inappropriate output format, this task was chosen as a focus.

\subsection{High-level tasks related to anomaly detection}

Most of these tasks are not immediately useful for anomaly detection in Splunk. For instance, signal recovery is a complex task that is unlikely to be required for most machine data and for which using specialized software is more appropriate.

The task of \emph{novelty detection} (task \ref{task:novelty_detection}), however, has potential in both monitoring and diagnosis applications. In the former case, anomalies that occur periodically will be diagnosed every time they occur, possibly leading to an unnecessarily large number of discovered anomalies. In the latter application, anomalies that occur periodically will be completely ignored (i.e. treated as normal if they occur many times), which can be problematic. Novelty detection methods avoid these problems.

Note that (as stated in section \ref{sect:related_tasks}) novelty detection is really just contextual (or contextual collective) anomaly detection with a specific context function. For this reason, we do not treat novelty detection on sequences as a task separate from anomaly detection, but as a specific choice of context, which we refer to as the \emph{novelty context}. Finally, the problems in monitoring and diagnosis applications described above can also be mitigated by other context functions. 

\section{Suggested tasks}
\label{sect:suggested_tasks}

The tasks selected in the previous section into complete tasks in which all principal factors are fixed.

Based on the discussion in section \ref{sect:relevant_data_format}, the following task ought to be given the highest priority:
\begin{task}
  \label{task:main}
  Detect contextual collective anomalies in, and assign anomaly scores to, the elements of a single univariate real-valued sequence sampled from a continuous process.
\end{task}

The remainder of this report will focus on this task. In the next chapter, the component framework, which simplifies the specification of remaining factors for this task, is proposed. Both ad-eval and the evaluation performed as part of this project focus on this task as well.

Of course, there are other, similar tasks that could also be very important to machine data anomaly detection. A set of tasks which naturally complement task \ref{task:main} in monitoring and diagnosing of sets of machine data are now presented.:

To begin with, as discussed in section \ref{sect:relevant_data_format}, frequential anomaly detection is very similar to continuous anomaly detection and is likely to be relevant to Splunk applications. We present this as the following task:

\begin{task}
  Detect contextual collective anomalies in, and assign anomaly scores to the elements of, a single univariate real-valued sequence containing the frequency of some event over time.
\end{task}

Furthermore, symbolic time series are potentially interesting in several applications, suggesting the following task (which is purposely vague, since it requires a different choice of principal factors): 

\begin{task}
  Detect anomalous subsequences in long symbolic sequences.
\end{task}

Of course, applying task \ref{task:ad_splunk2} (detecting anomalous sequences in a set of sequences) to the above tasks can also be interesting. The resulting tasks can be briefly stated as follows:

\begin{task}
  Detect anomalous sequences in a set of univariate real-valued time series sampled from continuous processes.
\end{task}

\begin{task}
  Detect anomalies sequences in a set of univariate real-valued time series containing the frequency of some event over time.
\end{task}

\begin{task}
  Detect anomalous sequences in a set of similar symbolic sequences.
\end{task}

Since all of the tasks presented in this section are complementary, and since they share many similarities, they should ideally be implemented and evaluated in tandem. However, due to time and space constraints, it was not possible to study most of them in sufficient depth this project (although the component framework and \texttt{ad-eval} can both support these tasks).

\chapter{Component framework}
\label{ch:methods}

Before Task~\ref{task:main} can be evaluated, an understanding of the set of problems that can be derived from it must be acquired. In this chapter, the \emph{component framework} is proposed, which facilitates a structured approach to this problem.

First, Section~\ref{sect:framework} introduces the component framework. In Section~\ref{sect:related_tasks_framework}, modifying the component framework to handle various related tasks is discussed. Finally, the component framework is used in Section~\ref{sect:deriving_problems} to discuss and derive interesting problems from Task~\ref{task:main}.

\section{The framework}
\label{sect:framework}

While Task~\ref{task:main} has not been previously studied, several papers exist covering special cases of it (typically corresponding to certain combinations of context functions, filters and anomaly measures). A key realization in the development of the component framework was that almost all these methods could be specified using a set of independent components, used by a common algorithm to find anomalies. If the components could be formalized and abstracted to pertain to Task~\ref{task:main}, then an understanding of the set of problems derived from that task could be obtained through the study of the individual components.

The first step towards this goal was to consider which factors are needed to specify a problem from Task~\ref{task:main}. We note that, at minimum, the following factors must be specified:

\begin{description}
  \item[The context function.] Since we are dealing with contextual collective anomalies, the context must be defined.
  \item[The filters.] Filters for extracting both \emph{evaluation subsequences} (from the input sequence) and \emph{reference subsequences} (from the context) must be specified.
  \item[The anomaly measure.] Arguably the most significant factor still unspecified.
  \item[Transformations.] Which transformations, if any, are to be applied to the data before analysis.
  \item[The aggregation method.] A method for aggregating individual anomaly scores into an anomaly vector must be provided.
\end{description}

Next, these five \emph{components} were defined mathematically, in order to facilitate their standardization and use in algorithms. Specifically, the tuple of components $(\mathcal{F}_E, \mathcal{C}, \mathcal{F}_R, \mathcal{M},\mathcal{A})$ was defined, where (letting $T \in X^n$ be the sequence under consideration and $X$ be some arbitrary set, $S_T$ be the set of subsequences of $T$, and $2^{\mathbb{Z}_n}$ be the set of corresponding indices in $T$):
\begin{description}
   \item[The evaluation filter $\mathcal{F}_E$] selects which subsequences of $T$ are to be evaluated by the anomaly measure. We let $\mathcal{F}_E(T) = \{(T_1, I_1), (T_2, I_2), \dots, (T_k, I_k)\} \subseteq \mathcal{S}_T \times 2^{\mathbb{Z}_n}$ be a set of tuples of subsequences of $T$ and their corresponding indices.
   \item[The context function $\mathcal{C}$] is responsible for selecting the appropriate context for subsequences. Formally, for $(T_i, I_i) \in \mathcal{F}_e(T)$, $\mathcal{C}(I_i) \subset \mathcal{S}_T$ gives one or more subsequences constituting the context.
   \item[The reference filter $\mathcal{F}_E$] extracts subsequences from the context to produce a set of reference items.
   \item[The anomaly measure $\mathcal{M}$] assigns an anomaly score to each of the selected items with regard to their reference set: $\mathcal{M}(T_i, \mathcal{F}_r{\mathcal{C}(I_i)}) \in \mathbb{R}$.
   \item[The aggregation function $\mathcal{A}$] aggregates the anomaly scores to produce an anomaly vector; i.e.\  if $S = {\{(a_i, I_i)\}}_{i \in \mathbb{Z}_k}$ is a vector of anomaly scores and indices for individual elements, then $\mathcal{A}(S) \in \mathbb{R}^{n}$.
\end{description}

Finally, an algorithm could be formulated that solves an arbitrary problem specified using this component framework, as follows:
\begin{algorithmic}
    \Require{A sequence $T$ and a tuple $(\mathcal{F}_E, \mathcal{C}, \mathcal{F}_R, \mathcal{M}, \mathcal{A})$.}
    \State{$S \gets \emptyset$ \Comment{initialize anomaly score container}}
    \For{$(T_i, I_i) \in \mathcal{F}_E(T)$} \Comment{iterate over subsequences (and indexes) selected by filter}
        \State{$C \gets \mathcal{C}(I_i)$ \Comment{acquire context}}
        \State{$R \gets \mathcal{F}_R(C)$ \Comment{extract reference set from context}}
        \State{$S \gets S \cup (\mathcal{M}(T_i, R), I_i)$ \Comment{save anomaly score with indexes of subsequence}}
    \EndFor{}
    \State{\Return{$\mathcal{A}(S)$ \Comment{aggregate scores to form anomaly vector}}}
\end{algorithmic}

Restating existing problems within this framework provides several advantages. Firstly, it facilitates comparison of methods, making it easier to see their similarities and differences. Secondly, it makes it easy to overview the entire set of problems and propose novel problems based on combinations of existing components. Finally, the components can be easily interchanged, simplifying method evaluations.

The fact that a single algorithm can be used for all problems suggests the possibility of automated, accurate handling of diverse datasets. Typically, the extent to which problems can capture anomalies varies drastically between series from different sources. If this ability could be related to underlying characteristics of the series, a preliminary analysis could guide the choice of components. Since the proposed framework simplifies automated analysis and allows for problems to be applied interchangeably on the same data, it could help both in finding these underlying characteristics and in applying them to anomaly detection.

As an example of such a characteristic, in~\cite{chandola3}, various anomaly measures are compared on a task derived From~\ref{task:ad_splunk2} (finding anomalous real-valued sequences in a set of similar sequences) and a correlation is noted for some problems between the share of anomalies captured and the periodicity of the sequences. Based on this, investigating links between problem accuracy and characteristics of series in the frequency domain may prove fruitful.

\section{Related tasks}
\label{sect:related_tasks_framework}

It is important to note that the component framework is not limited to Task~\ref{task:main}, but can be modified to handle several related tasks as well. For instance:
\begin{description}
    \item[Semi-supervised] tasks are retrieved by specifying a reference sequence $R$ in addition to the evaluation sequence $T$, and letting $\mathcal{C}(T_i) \equiv R$.
    \item[Collective anomaly] detection tasks are retrieved by specifying $\mathcal{C}(T_i) = T \setminus T_i$.
    \item[Point anomaly and contextual anomaly] detection tasks can be retrieved by specifying $\mathcal{F}_{E,R}(T) = {\{(t_i)\}}_{t_i \in T}$ and selecting an appropriate context function.
    \item[Sequential and frequential] tasks need no modifications; they merely require different types of pre-processing.
    \item[Alternative anomaly scores] can be retrieved through trivial modifications of the aggregation method $\mathcal{A}$.
    \item[Finding anomalous sequences among a set of sequences] mainly requires simplification: $\mathcal{F}_E$, $\mathcal{C}$, $\mathcal{F}_R$ and $\mathcal{A}$ can all be ignored for such tasks. The only relevant part is the anomaly measure.
\end{description}

Based on this, the component framework could help elucidate how different sequential anomaly detection tasks are related. Furthermore, since all derived problems can be solved using a single algorithm when formulated using the component framework, the implementation and evaluation of a large class of anomaly detection tasks could be performed relatively easily. Indeed, it would be simple to modify \texttt{ad-eval} (which implements the component framework) to handle any of the above tasks.

\section{Components}
\label{sect:deriving_problems}

We now discuss possible choices of the components $\mathcal{F}_E$, $\mathcal{C}$, $\mathcal{F}_R$, $\mathcal{M}$ and $\mathcal{A}$ and the implications of these choices.

While Task~\ref{task:main} has not been previously studied, components based on existing methods for other tasks can be used with it. In the following discussion, several such component choices, as well as a few new components, are mentioned.

\subsection{The evaluation filter}

The evaluation filter takes a sequence and extracts subsequences from it to be used in the analysis. In the literature, \emph{sliding window} approaches are almost always taken. In these methods, evenly spaced subsequences of equal length are extracted from the series:
\[
  \mathcal{F}_E((t_1, t_2, \dots, t_n)) = \{(t_1, t_2, \dots, t_w), (t_{1 + s}, t_{2+s}, \dots, t_{w+s}), \dots, (t_{\alpha-s}, t_{\alpha-s+1}, \dots, t_{\alpha})\},
\]
where $w$ is the window length, $s$ is the displacement, and $\alpha = \lfloor \frac{n}{s} \rfloor \times w$. A constant value of $w$ is almost always used, since most anomaly measures can not deal with items of varying dimensionality. The value of $s$ is typically taken to be $1 \leq s \leq w$. The total number of elements is $\lfloor \frac{n-w}{s} \rfloor + 1$, and thus decreases with larger $s$.

Other approaches might be appropriate for anomaly measures which support variable-size elements. In~\cite{keogh2}, a binary search is performed on $T$ to find discords. Note that such an approach requires a slight modification of the algorithm given above.

\subsection{The context function}
As mentioned before, several different context functions are plausible for use in sequential anomaly detection. We have previously mentioned the context (assuming that $T = (t_1, t_2, \dots, t_n)$ is the series under consideration):
\[
    \mathcal{C}((t_i, t_{i+1}, \dots, t_j)) = \{ (t_{i-m}, t_{i-m+1}, \dots t_{i-1}), (t_{j+1}, t_{j+2}, \dots, t_{j+n}) \},
\]
which we refer to as the \emph{symmetric local context} if $m = n$ and the \emph{asymmetric local context} otherwise; the \emph{novelty context}:
\[
    \mathcal{C}((t_i, t_{i+1}, \dots, t_j)) = \{ (t_1, t_2, \dots, t_{i-1}) \};
\]
and the \emph{trivial context} (for detecting collective or point anomalies):
\[
    \mathcal{C}((t_i, t_{i+1}, \dots, t_j)) = \{ (t_1, t_2, \dots, t_{i-1}), (t_{j+1}, t_{j+2}, \dots, t_n) \}.
\]
We also introduce the \emph{semi-supervised context}, in which $\mathcal{C}(T_i) \equiv X$ for some reference sequence $X$.

Note that the number of elements in the context will typically have a significant impact on analysis time. While the sizes of the local and semi-supervised contexts are both constant as functions of the sequence length, the novelty and trivial contexts grow linearly, which is likely to cause problems in large datasets.

\subsection{The reference filter}

The reference filter works like the evaluation filter, but extracts subsequences from the context instead of from the evaluation series. As with the evaluation filter, it is usually best to use a sliding window as the reference filter. If the anomaly measure does not require items of equal length, the reference filter can be ignored, i.e.\  $\mathcal{F}_R(T) = T$.

\subsection{The anomaly measure}

As is typically the case, the anomaly measure is the central part of any problem derived from Task~\ref{task:main}. Formally, any anomaly measure that takes an evaluation vector and a set of reference vectors as inputs and returns a real value is a candidate for $\mathcal{M}$. We now discuss a few such anomaly measures, following the classification in Section~\ref{sect:anomaly_measures}.

As previously mentioned, \emph{statistical measures} are attractive due to the theoretical justification they provide for anomaly detection. However, there are certain factors which render their use problematic for the task at hand. Firstly, as discussed in Section~\ref{sect:goals} we can not assume that the data belongs to any particular distribution, so parametric problem formulations are not appropriate.

Since few non-parametric methods for anomaly detection in sequential data take into account either collective anomalies or context, and since naive approaches are likely to suffer from convergence issues,~\footnote{For instance, the expectation maximization algorithm for Gaussian mixture models has convergence issues in high dimensions with low sample sizes.} suggesting appropriate non-parametric methods for Task~\ref{task:main} is difficult. However, in the case of Task~\ref{task:frequency}, point anomalies (for which statistical methods have been extensively researched) are more interesting than collective anomalies, and statistical methods are likely to be applicable.

\emph{Information theoretical measures} are potentially interesting for the chosen task. However, most such methods are essentially distance-based anomaly measures equipped with information theoretical distance measures. For this reason, we do not cover information theoretical anomaly measures separately from distance-based measures.

Instead, we emphasize \emph{distance-based measures}. Since they encompass a wide variety of methods, which have been shown to be successful for related problems~\cite{chandola3}, we dedicate Section~\ref{sect:distance_mezurez} to discussing them.

\emph{Classifier-based measures} have also shown promise for related problems~\cite{chandola3}. Generally, any \emph{one}-class classifier is potentially suitable for the task; see~\cite{classification} for an exhaustive discussion on this topic. While one-class classifiers produce binary output, appropriate anomaly vectors can still be produced through a suitable choice of $\mathcal{A}$.

\emph{Predictive model-based measures} are also potentially interesting. They are naturally well suited for novelty detection, since they necessarily implement a novelty context. However, existing predictive model-based approaches seem to be lacking for the task at hand. In~\cite{chandola3}, a leading model-based novelty detection method~\cite{perkins2} which uses an autoregressive model was shown to perform poorly on a related task. While models are important as a means of reducing the computational complexity of methods, they appear appropriate to consider only as approximations to other anomaly measures.

\subsubsection{Distance measures}
\label{sect:distance_mezurez}

When dealing with distance-based problems, the choice of distance measure has a profound impact on which anomalies are detected. As with other aspects of anomaly measures, however, drawing conclusions about method efficacy through theory alone is difficult; implementing, evaluating, and comparing several measures will likely prove more instructive.

Possible interesting measures include the \emph{Euclidean distance} or the more general \emph{Minkowski distance}; measures focused on time series, such as \emph{dynamic time warping}~\cite{dtw}, \emph{autocorrelation measures}~\cite{autocorrelation}, or the \emph{Linear Predictive Coding cepstrum}~\cite{cepstrum}; or measures developed for other types of series (accessible through transforms), such as the \emph{compression-based dissimilarity measure}~\cite{keogh2}.

Additionally, the choice of distance measure affects how well methods can be optimized. Naive approaches to distance-based problems typically scale prohibitively slowly, and are not suitable for large amounts of data. Optimizations typically involve exploiting properties of the distance measure in order to reduce the number of distance computations (for instance, the commonly used k-d trees algorithm requires the distance to be a Minkowski metric).

\subsubsection{Transformations}
Anomaly measures can be combined with one or more transformations of the data extracted by the filters to speed up the analysis or to facilitate the detection of certain types of anomalies.

One commonly used data transformation is the Z-normalization transform, which modifies a sequence to exhibit zero empirical mean and unit variance. It has been argued that comparing time series is meaningless unless the Z-normalization transform is used~\cite{keogh5}. However, this is doubtful, as the transform masks anomalies consisting of subsequences similar to their surrounding data but with different variance or mean.

Transformations that transform the data into some alternative domain can also be useful. For example, transformations based on the \emph{discrete Fourier transform} (DFT) and \emph{discrete wavelet transform} (DWT)~\cite{fu} have shown promise for sequential anomaly detection.

Furthermore, transformations for real-valued data which produce symbolic sequences are very important, since they enable the application of symbolic approaches (studied in bioinfomatics, for instance) to real-valued sequences. While several such transformations exist, \emph{symbolic aggregate approximation} (SAX)~\cite{sax} is by far the most commonly used.

\subsection{The aggregation method}
\label{sect:aggregation_method}

Once the extracted subsequences have been assigned anomaly scores, they must be aggregated before an anomaly vector can be formed. While little has been written on this subject, it is not difficult to describe reasonable aggregation methods. It seems clear that all aggregation functions, or \emph{aggregators}, ought to have the form
\[
    \mathcal{A}(\{(a_1, I_1), (a_2, I_2), \dots, (a_k, I_k) \}) = (f(\{a_i: 1 \in I_i\}), f(\{a_i: 2 \in I_i\}), \dots, f(\{a_i: n \in I_i\})),
\]
where $I_i$ are intervals, $a_i$ are assigned anomaly scores, and $f$ is some aggregator-specific function acting on a vector $\mathbb{R}^n$ and producing an aggregate score in $\mathbb{R}$. The \emph{maximum}, \emph{minimum}, \emph{median} and \emph{mean} of the values in $S$ all constitute reasonable choices of $f(S)$. We will refer to aggregators using any of these four functions as maximum, minimum, median, and mean aggregators, respectively.

\chapter{Evaluation}
\label{ch:evaluation}

As mentioned throughout this report, thorough empirical evaluations on real-world data sets are pivotal to elucidating the performance characteristics of problems, methods and tasks. In this chapter, two factors important when performing evaluations are discussed: evaluation data and error measures.

\section{Evaluation data}
\label{sect:evaluation_data}

For obvious reasons, the choice of evaluation data largely determines the outcome of the evaluation. Since the spaces of possible evaluation items for most anomaly detection methods are enormous (often $\mathbb{R}^n$ for very large $n$), there is little hope of performing exhaustive analyses. Furthermore, while only relatively small subsets of these spaces are typically interesting for any given application, discovering and delimiting these subsets is usually impossible, due to a limited understanding of the processes underlying the data.

For this reason, it is critical that a set of evaluation items is used, containing as representative a selection of relevant phenomena exhibited by data items in the application domain as possible. Since such sets can not be obtained through purely synthetic means (arguably, the existence of a model that could accurately create such sets would presuppose a level of understanding of the application domain that would render anomaly detection unnecessary), real-world data must be used.

However, simply obtaining real-world data is \emph{not} sufficient; if the anomalies in the data are not labeled, no baseline can be established, and the evaluation is useless. In the context of task \ref{task:main}, this means that, along with each sequence $S_i = (s_1, s_2, \dots, s_n)$ in the evaluation set, a reference anomaly vector $A_i = (a_1, a_2, \dots, a_n)$ must be provided, where each $a_i$ indicates how anomalous $s_i$ is.

When appropriate data is unavailable, there are two possible options. The first is to generate both series and anomalies, in order to create completely artificial sequences. The second is to take unlabeled real-world data sets and superposition artificially generated anomalies onto them.

Due to how greatly the data set affects the outcome of anomaly detection analyses, the first approach is not recommended, other than as a means of gaining a qualitative understanding of tasks, problems or methods. Even if the artificial data appears highly similar to data from the target domain, it is very unlikely that the "best" methods for such data and real-world data will be the same. The second approach, while also problematic, can at least suggest how well methods are able to recognize normal data from the target domain from anomalous data.

In the initial stages of this project, it was hoped that sufficient data would be obtained by searching the internet and existing papers on the subject, or by examining sequences of machine data with a domain expert. However, enough suitable series could not be obtained through these means, and it became clear that completely artificial data would have to be used in the evaluation. Because such data can not adequately capture the intricacies of the target domain, a qualitative evaluation of implemented methods was performed instead. Still, since a major goal of the project was to facilitate evaluations using real-world data, utilities for performing appropriate evaluations were added to \texttt{ad-eval}.
 
\section{Error measures}
\label{sect:evaluation_measures}

For any evaluation, some way of assessing the accuracy of the produced anomaly vectors is required. Formally, given a sequence $S = (s_1, s_2, \dots, s_n)$, a reference anomaly vector $R = (r_1, r_2, \dots, r_n) \in \mathbb{R}^{n}$ and a method that takes a sequence as input and produces an anomaly vector $A = (a_1, a_2, \dots, a_n) \in \mathbb{R}^{n}$, a function $\epsilon(A, R): \mathbb{R}^n \times \mathbb{R}^n \rightarrow \mathbb{R}^+$ is required that assigns a similarity score, or \emph{error measure} to $A$ based on how well it approximates $R$.

Considering the difficulties involved in obtaining any kind of labeled sequences, and the fact that ranking elements based on relative degrees of anomalousness is even more difficult, it is reasonable to assume that all reference anomaly vectors are binary: $R = (r_1, r_2, \dots, r_n) \in \{0,1\}^n$. Since any method derived from task \ref{task:main} produces real-valued output, this means that error measures should work by comparing one binary and one real-valued vector; i.e. $\epsilon(A, R): \mathbb{R}^{n} \times \{0,1\}^n \rightarrow \mathbb{R}^+$.

Since constant factors do not affect how accurate an anomaly vector appears, any $\epsilon(A, A_R)$ should be invariant under uniform translations and scalings of $A$. This means that regular real-valued distance measures (such as the Euclidean distance) are not applicable. This can be avoided by normalizing $A$---by scaling and translating it such that all its elements lie in $[0,1]$---before computing the distance. For instance, the \emph{normalized Euclidean error} $\epsilon_E$, given by
\[
  \epsilon_E = \sqrt{\sum_{i=1}^n (\frac{a_i-a_{min}}{a_{max}-a_{min}} - r_i)^2}
\]
ought to constitute a reasonable choice of error measure. However, since $\epsilon_E$ is equally sensitive to the behaviour of $A$ on normal elements as on anomalous elements, how $A$ treats normal elements (which should not matter unless they have high anomaly scores) can significantly harm the accuracy of this measure, as we will see in \ref{sect:error_measure_eval}. 

This pitfall can be avoided by converting $A$ to a binary string $A_B$ as well, and using a binary distance between $A_B$ and $R$ as the error measure. Since this is equivalent to selecting a set $S_A$ of indexes of elements from $A$ and comparing it to the set $S_R$ of indexes of nonzero elements of $R$, such error measures can be seen as evaluating the performance of $A_B$ on task \ref{task:anomalous_set}, which is reasonable considering the binary nature of $R$.

What remains is to decide how $A_B$ should be constructed from $A$ and what binary distance measure to use. Since the individual elements of $S_R$ have equal weight, and since the locations of discrepancies between $A_B$ and $R$ are not important, the \emph{Hamming distance} $\delta_H$ constitutes a suitable choice of distance measure. Furthermore, $A_B$ should be constructed such that $\forall i, j \leq n: a_i < a_j \wedge i \in S_A \Rightarrow j \in S_A$, which is equivalent to setting a threshold $a_{min} \leq \tau \leq a_{max}$ and letting $A_B$ be given by 
\[
    A_B = (a_{b,1}, a_{b,2}, \dots, a_{b,n}), \quad \text{where } a_{b,i} = \left\{ 
    \begin{array}{l l}
        0 & \quad \text{if } a_i < \tau \\
        1 & \quad \text{if } a_i \geq \tau
    \end{array} \right..
\]

This gives rise to a number of possible $A_B$ equal to the number of distinct values of $A$. We here propose two error measures corresponding to different choices of $\tau$, both based on $\delta_H$. The \emph{equal support error} $\epsilon_{ES}$ corresponds to setting $\tau$ such that $|S_A| = |S_R|$, while the \emph{full support error} $\epsilon_{FS}$ corresponds to choosing the largest $\tau$ such that $\forall i: i \in S_R \Rightarrow i \in S_A$. While other choices (such as choosing the $\tau$ that minimizes $\delta_H$) might also be appropriate, only these two were selected in order to limit the scope of the discussion.

Since the three error measures proposed above are expected to produce different results, and since they have not been (to the author's knowledge) used in the context of series anomaly detection, they should be evaluated empirically before being used. Such an evaluation is performed in section \ref{sect:error_measure_eval}.

\chapter{ad-eval}
\label{ch:implementation}

As stated in the abstract, the development of a software framework for the evaluation of anomaly detection methods, called \texttt{ad-eval} and available at \url{http://github.com/aeriksson/ad-eval}, was a significant part of the project. In this chapter, the design, development process, and features of \texttt{ad-eval} are discussed.

Essentially, \texttt{ad-eval} consists of three separate parts: a library implementing the component framework, a comprehensive set of utilities for evaluating the performance of methods and problem, and an executable leveraging the library. The entire project is written in Python.

In this chapter, \texttt{ad-eval} is described in detail. The three parts are described in sections \ref{sect:implemented_problems}, \ref{sect:evaluation_package}, and \ref{sect:executable}, and some of the design choices made in the development of \texttt{ad-eval} are discussed in section \ref{sect:design}.

\section{Implemented components}
\label{sect:implemented_problems}

The anomaly detection part of \texttt{ad-eval} (given the Python package name \texttt{anomaly\_detection}) is a faithful implementation of the component framework, including the algorithm proposed in section \ref{sect:framework}. In order to preserve the modular nature of this framework, the individual components are implemented as autonomous modules, described below.

As there are no real alternatives found in the literature, the sliding window filter is the only implemented evaluation filter. Since the optimal window width $w$ and step length $s$ depend on the application, both of these parameters were left to be user-specified.

Since new new context functions can be implemented relatively easily, and since they have a relatively major impact on the analysis, all previously discussed contexts (specifically, the asymmetric and symmetric local contexts, the novelty context, the trivial context, and the semi-supervised context) were implemented.

As reference filters, a sliding window filter and the identity filter $\mathcal{F}_E(X)=X$ were implemented, the latter because it is a better fit for dimension-independent distance measures.

Due to the limited scope of the project, the only implemented anomaly measures (henceforth referred to as \emph{evaluators}) were a variant of k-Nearest Neighbors (kNN), in which the distance to the $k$'th nearest element is considered, and a one-class support vector machine (SVM). For the kNN evaluator, the Euclidean distance as well as the compression-based dissimilarity measure \cite{keogh2} and dynamic time warping \cite{dtw} distances were made available. Furthermore, the symbolic aggregate approximation (SAX) and discrete Fourier transform (DFT) transformations were added as an optional pre-anomaly measure transformations. All parameters of the evaluators, distances, and transformations were left for the user to specify.

Finally, the mean, median, maximum, and minimum aggregators were implemented.

\section{Evaluation utilities}
\label{sect:evaluation_package}

The set of possible interesting tests that could be run on problems derived from task \ref{task:main} is considerable. A large number of component combinations can be used; most components have many possible parameter values, and it is important to assess how these affect the results; and a large set of methods with various optimizations and approximations can be proposed. For all of these choices, it is important that accuracy and performance are properly evaluated.

However, the performance of methods is highly dependent on the characteristics of the data sets to which they are applied. As mentioned in section \ref{sect:evaluation_data}, there is no hope to exhaustively cover the space of possible evaluation sets. Instead, sample data from the target application domain must be obtained before any tests are performed. Furthermore, obtaining adequate labeled test data is often difficult, and artificial anomaly generation must be considered as an option.

With this in mind, it was decided that an evaluation framework should be added to \texttt{ad-eval} to help facilitate the implementation, standardization, and duplication of accuracy and performance evaluations. Due to the variety of interesting tests highlighted above, an approach focused on the provision of tools that assist in scripting custom tests was deemed preferable to one focused on the construction of a single configurable testing program. To this end, utilities were developed for:

\begin{itemize}
    \item Saving and loading time series to/from file, with or without reference anomaly vectors.
    \item Pseudorandomly selecting and manipularing subsequences of series (e.g. for adding anomalies).
    \item Facilitating the testing of large numbers of parameter values.
    \item Generating various types of artificial anomalies and superpositioning them onto sequences.
    \item Calculating the anomaly vector distance measures $\epsilon_E$, $\epsilon_{ES}$ and $\epsilon_{FS}$ discussed in section \ref{sect:evaluation_measures}.
    \item Facilitating the automated comparison of several problems and methods on individual data sets.
    \item Automating the collection of performance metrics.
    \item Reporting results.
    \item Generating various types of custom plots from results.
\end{itemize}

With these tools in place, it is simple to write scripts that, for instance, generate large amounts of similar series containing random anomalies and evaluate the performance of several problems on this data in various ways. 

The tools were included in \texttt{ad-eval} as a separate Python module (called \texttt{eval\_utils} and located in the \texttt{evaluation} directory of the repository). This module was used to perform all tests in the evaluation phase of this project.

\section{Executable}
\label{sect:executable}

To enable the stand-alone use of the anomaly\_detection package, an executable was added to the \texttt{ad-eval} repository (called \texttt{anomaly\_detector} and located in the \texttt{bin} directory of the repository). This executable is used through a command-line interface and a \texttt{key:value} style configuration file, can perform supervised or semi-supervised anomaly detection on sequences from files or standard input, and can use any of the components implemented in \texttt{anomaly\_detection}.

To avoid having to modify this program every time a component in \texttt{anomaly\_detection} was changed, the executable was made unaware of all internal details of that package. Consequently, a command-line interface capable of configuring the components could not be implemented; instead, a configuration file parser is used to read and pass the component configuration to \texttt{anomaly\_detection}.

\section{Design}
\label{sect:design}

The development of \texttt{ad-eval} began at the start of the project. Initially, development efforts focused on the implementation of a few optimized methods found in the literature, to produce a Splunk app. However, as the project progressed and the issues discussed in section \ref{sect:adb} (that most methods were targeted at subtly different tasks, and that due to lacking evaluations, assessing which methods are really the `best' is not possible), this approach was recognized as fruitless, and abandoned.

Development then shifted towards an implementation of the component framework, with the goals of maximizing the ease of implementing and evaluating large amounts of components.

Consequently, modularity was a major focus throughout the development process, achieved through various means. As mentioned previously, the individual components were separated into independednt modules. Additionally, the evaluation utilities and the executable were decoupled from the component framework implementation, interfacing with it through only two method calls. Finally, the decision was made to distribute the configuration of the component framework implementation, letting each component handle its own configuration and making the rest of the package configuration-agnostic.

It was a natural choice to write the entire implementation in Python, for several reasons. First, Python is well suited for small, flexible projects such as \texttt{ad-eval}, thanks to its simplicity and flexibility. Furthermore, a number of great libraries for data mining and machine learning exist for Python, which were used to accelerate the development. Finally, if \texttt{ad-eval} becomes adopted for real-world use, Python's good C integration could be leveraged to write optimized code. 

Finally, the evaluation utilities were designed with ease of use and flexibility in mind. For instance, while classes facilitating test data generation, evaluation, and reporting are provided, their use is optional. As a result, evaluation scripts could be short and simple---the scripts used in the next chapter are all 30 to 70 lines long---without sacrificing flexibility.

\chapter{Results}
\label{ch:results}

Due to the lack of appropriate evaluation data mentioned in chapter \ref{ch:evaluation}, and in order to limit the scope of this report, a comprehensive evaluation of the implemented problems could not be performed. Instead, a preliminary, qualitative evaluation was performed, with the goal of gaining some insight into the relative performance of problem formulations derived from task \ref{task:main}, and demonstrating how \texttt{ad-eval} can be used to simplify and standardize the process of evaluating anomaly detection problems and methods. 

In this chapter, the results of this evaluation are presented. Since distance-based evaluators can be combined with a wider set of other components than classifier-based or other types of evaluators, and to keep this report from becoming overly long, the focus was placed entirely on the kNN evaluator implemented in \texttt{ad-eval}. The performance of this evaluator was investigated through the identification of all remaining unspecified parameters of the components $(\mathcal{F}_E, \mathcal{C}, \mathcal{F}_R, \mathcal{M},\mathcal{A})$, and individual investigations of how these parameters affect the analysis.

Of course, all graphs and data in this chapter were obtained using \texttt{ad-eval}. Since one of the main goals of the evaluation was to demonstrate how \texttt{ad-eval} can be used to perform standardized evaluations, all scripts used to obtain the figures and results in this chapter are available in the \texttt{ad-eval} source code repository. Modifying these scripts to use other data sets or to evaluate other components (such as the SVM evaluator implemented in \texttt{ad-eval}) is trivial.

\section{Evaluation approach}

Since the number of possible combinations of components and parameter values is enormous (even with the limited number of components implemented in \texttt{ad-eval}), it was not feasible to include a comprehensive evaluation of all these combinations in this report. Instead, the components were studied individually, using a preset configuration and varying individual parameter values. Note that this approach only allows for the assessment of local characteristics of the parameter space around the specific configuration.

As a second limitation, it was decided that the analysis would be focused on a single artificial sequence, owing to the lack of proper data sets to evaluate.
While a large artificial data set could have been generated for the evaluation, this would have been counterproductive for several reasons.
Specifically, as discussed in chapter \ref{ch:evaluation}, generating data set sufficiently diverse to accurately reflect the characteristics of real-world data sets in target domain is practically impossible, and creating even a rough artificial approximation would require substantial effort.
Any such data set that could fit within the scope of this project would thus be severely limited, and would not be appropriate for assessing real-world performance.
To highlight these deficiencies, and to simplify the exposition, it was decided that the focus would be placed on examining the performance characteristics of problems on a single artificial sequence.
Rather than performing evaluations with different types of artificial sequences, all scripts used in the evaluation were added to the \texttt{ad-eval} source repository, and were constructed such that performing similar evaluations on new sequences would be trivial.

\begin{figure}[h]
    \vspace{-10pt}
    \begin{center}
        \includegraphics[trim = 10mm 0mm 5mm 0mm, clip, width=\textwidth]{resources/reference_sequence}
    \end{center}
    \vspace{-20pt}
    \caption{\small{The standard sequence with corresponding reference anomaly vector.}}
    \label{fig:reference_sequence}
    \vspace{-10pt}
\end{figure}

Before choosing a sequence, there were a few things to consider. Since evaluating a large number of problem should be possible on modest hardware in short time spans, and since anomaly detection methods are typically rather slow, the sequence should be short. However, the sequence should still contain both normal data, homogeneous enough to establish a baseline of `normal' behaviour, as well as an anomaly that deviates appropriately from this baseline. "Appropriate", in this case, means that it should be relatively easy to detect with most reasonable parameter choices, but difficult enough to be detectable regardless of the parameter choices. A sequence of length $400$ was settled on, generated by a random walk, with added noise in the range $180$ to $220$. This sequence, referred to in the remainder of this chapter as the \emph{standard sequence} or $s^*$, is shown in figure \ref{fig:reference_sequence} along with the corresponding reference anomaly vector $a^*$.

\section{Parameter space}

When considering the set of problems derived from some task, it can be helpful to regard the set of variables necessary to fully specify a problem from it as the \emph{parameter space} of that task. Individual variables correspond to dimensions in the space, while problems correspond to points. General tasks are associated with large, high-dimensional parameter spaces while more specific tasks have smaller, more manageable spaces.  Searching for an optimal problem derived from some task for some data set, then, means searching for a point in the parameter space of the task at which the corresponding problem can most efficiently find the anomalies in the data set. Equivalently, this can be thought of as an attempt to minimize some error function over the parameter space. 

In this case, the parameter space corresponds to the free parameters of the components $C = (\mathcal{F}_E, \mathcal{C}, \mathcal{F}_R, \mathcal{M},\mathcal{A})$. We will denote these by $\Theta$. Assuming that some function $A(\Theta, s)$ is provided that solves the problem corresponding to $\Theta$ for a sequence $s$ (or equivalently uses the anomaly detector corresponding to $\Theta$ to evaluate $s$) and outputs an anomaly vector, the task of finding an optimal problem formulation for some data set $S$ can be seen as the task of finding $\argmin_\Theta E(\Theta, S)$ for some error function $E$ that evaluates the error of $A(\Theta, s)$ for each $s \in S$. Assuming that this error function is a linear combination of the errors according to some error measure $\delta$ of the elements in $S$, this can be written as 
\[
    E_{S, \delta}(\Theta) = \sum_{(s_i, a_i) \in (S, A)} \delta(A(\Theta, s_i), a_i).
\]
Given a data set and a set of possible components, this task is relatively straight-forward. While filters, contexts and aggregators with infinite parameter spaces are possible, the components discussed in this report have relatively small, finite parameter spaces. This means that an exhaustive search would be possible in theory.

However, most choices of $C$ will typically have a large number of free parameters, resulting in a high-dimensional parameter space. This, in combination with the fact that $E_{S, \delta}(\Theta)$ typically takes a long time to evaluate, even for small $S$, renders such exhaustive searches prohibitively computationally expensive in practice.

\section{Standard configuration}
\label{sect:standard_config}

Due to the limited computational resources available when performing the evaluation, and in order to simplify the presentation, the parameters had to be studied in isolation. This amounts to studying the behaviour of $E_{S, \delta}$ near a single point in the parameter space by considering its behaviour along orthogonal lines meeting at this point. This is not sufficient to draw any strong conclusions about global minima of the error over the parameter space, unless the parameters influence $E$ independently, which is certainly not the case for task \ref{task:main}. However, this is not problematic; the aim of the evaluation was to gain insight into how the individual parameters affect the outcome, not to find global minima. 

The fixed point in the parameter space will be referred to as the \emph{standard configuration}, and denoted by $\Theta^* = (\theta_1^*, \theta_2^*, \dots, \theta_n^*)$. The choice of $\Theta^*$ is now described.

As mentioned previously, $\mathcal{E}$ was fixed as a kNN evaluator. This evaluator has the free parameters $k$ (which of the nearest neighbors to use when calculating the anomaly score---set to 1 by default), $\delta$ (the distance function---the standard Euclidean distance $\delta_E$, by default) and an optional transformation $t: \mathbb{R}^n \rightarrow \mathbb{R}^m$ to apply to the items in the reference set before the evaluation. By default, no transformation is used.

Because the distance measures implemented in \texttt{ad-eval} (except the compression-based dissimilarity measure) require extracted elements to be of the same length, sliding window filters were the only reasonable choice. It was thus decided that sliding window filters would be used for both the evaluation and reference filters, with the free parameters $w$ (window width---10 by default) and $s$ (step length---1 by default).

Since the evaluation was performed on artificial data, there was no reason to use either the novelty or asymmetric local context. Furthermore, since the trivial context is just a special case of the local symmetric context (with the context width $m$ set to a maximum), the trivial context was selected to be the default, with the single free parameter $m$, set to $400$.

Finally, since the four aggregators implemented in \texttt{ad-eval} are all parameter-free, it was decided that the aggregator itself would be treated as a parameter. Since it was assumed that the mean aggregator would give the most accurate results, it was selected as the default.

In summary, then, the parameter space for this evaluation is parametrized by $(k, \delta, t, w, s, m, \mathcal{A})$. To simplify the following discussion, we will take $\Theta^*$ to be the point where all parameters take on the values specified above, and let $\Theta^*_{\alpha_1, \alpha_2, \dots, \alpha_n}$ be the set of points where all parameters except $\alpha_1, \alpha_2, \dots, \alpha_n$ take on these values (e.g., $\Theta^*_k$ corresponds to the set of points where all parameters other than $k$ agree with the standard configuration). We will further denote the set of corresponding anomaly vectors on $s^*$ as $A_{\alpha_1, \alpha_2, \dots, \alpha_n}$. The $A_\alpha$ for $\alpha = k, \delta, t, w, s, m$, and $\mathcal{A}$ are examined in section \ref{sect:params}.

\section{Error measures}
\label{sect:error_measure_eval}

\begin{figure}[H]
    \centering
    \begin{subfigure}[b]{0.32\textwidth}
        \centering
        \includegraphics[trim = 30mm 5mm 35mm 15mm, clip, width=\textwidth]{resources/normalized_euclidean_error_heat_map}
        \caption{Normalized Euclidean error}
    \end{subfigure}
    \begin{subfigure}[b]{0.32\textwidth}
        \centering
        \includegraphics[trim = 30mm 5mm 35mm 15mm, clip, width=\textwidth]{resources/equal_support_error_heat_map}
        \caption{Equal support error}
    \end{subfigure}
    \begin{subfigure}[b]{0.32\textwidth}
        \centering
        \includegraphics[trim = 30mm 5mm 35mm 15mm, clip, width=\textwidth]{resources/full_support_error_heat_map}
        \caption{Full support error}
    \end{subfigure}
    \caption{Heat maps showing the normalized values of $\epsilon(A(\Theta_{k, w}^*, s^*), a^*)$ for the three errors and $k, w = 1,2,\dots,50$. Blue and red correspond to low and high error values, respectively.}
    \label{fig:error_heat_maps}
    \vspace{-10pt}
\end{figure}

In section \ref{sect:evaluation_measures}, three error measures for anomaly vectors were introduced: the normalized Euclidean error $\epsilon_E$, the equal support error $\epsilon_{ES}$, and the full support error $\epsilon_{FS}$. Since these methods have not been previously studied with regards to sequential anomaly detection, how well they capture the intuitive notions of accuracy must be assessed before they can be used to evaluate problem formulations.

Such an assessment was performed by computing and graphing the errors $\epsilon(A(\Theta_{k, w}^*, s^*), a^*)$ for $\epsilon = \epsilon_E, \epsilon_{ES}$ and $\epsilon_{FS}$ and $(k, w) \in \{1,2,\dots,50\}^2$. Heat maps of these values are shown in figure \ref{fig:error_heat_maps}. A few of the $A_{k, w}$ given the lowest values by each of the error measures were also graphed (figure \ref{fig:n_best_anomaly_vectors}.

As shown in the heat maps, the three error measures give similar results, attaining minima and maxima in the same regions. Since $\epsilon_{ES}$ and $\epsilon_{FS}$ operate on binary strings and thus have discrete domains, they often assign identical errors to nearby points. This is the cause of the relatively jagged appearance in the plots of these errors compared to the smoother appearance of the $\epsilon_{E}$ plot.

\begin{figure}[ht]
    \vspace{-5pt}
    \begin{center}
        \includegraphics[width=\textwidth]{resources/n_best_anomaly_vectors}
    \end{center}
    \vspace{-20pt}
    \caption{\small{The $n$th best $A_{k, w}$ according to the three error measures for $n = 1, 10, 50$ and $100$.}}
    \label{fig:n_best_anomaly_vectors}
    \vspace{-10pt}
\end{figure}

Figure \ref{fig:n_best_anomaly_vectors} shows the anomaly vectors with the $n$th lowest errors for the three distance measures. All three measures give similar anomaly vectors for $n = 1$ and $n = 10$, with the normalized Euclidean and full support errors giving the same anomaly vector in both cases. For $n = 50$ and $n = 100$, however, the full support error seems to prioritize smooth anomaly vectors, while the other two anomaly measures prioritize anomaly vectors with few false positives.

One interesting aspect evidenced in the heat map plot is that while $\epsilon_{ES}$ and $\epsilon_{FS}$ are both very large for $A_{k,w}$ with small $w$, this tendency is not shared by the Euclidean error. As seen in section \ref{sect:w}, anomalies significantly larger than $w$ will not be detected by kNN methods, which means that assigning a large value to these anomaly vectors is reasonable. Since the normalized Euclidean error (unlike the other two distances) gives equal weight to every component, it will assign relatively low values to anomaly vectors that only partially capture anomalies as long as most of their elements are close to zero.  Indeed, this is the case for the $A_{k,w}$ with small $w$ since these anomaly vectors are close to constant everywhere except for a few spikes.

\begin{figure}[ht]
    \vspace{-5pt}
    \begin{center}
        \includegraphics[width=\textwidth]{resources/euclidean_problem}
    \end{center}
    \vspace{-20pt}
    \caption{\small{A reference anomaly vector for a long sequence and two corresponding candidate anomaly vectors. The first candidate vector, while noisy, correctly marks the anomaly. The second candidate does not mark the anomaly and marks two false anomalies. $\epsilon_{E}$, $\epsilon_{ES}$ and $\epsilon_{EF}$ for the two sequences are $8.3$ and $2.2$; $0.010$ and $0.99$; and $0.0050$ and $0.99$, respectively.}}
    \label{fig:euclidean_problem}
    \vspace{-10pt}
\end{figure}

As an illustration of the potential problems this could cause, see figure \ref{fig:euclidean_problem}, which shows one reference anomaly vector and two candidate anomaly vectors for a long sequence. Here, the first anomaly vector, while noisy, accurately captures the anomaly while the second not only misses the anomaly, but also introduces two false positives. While $\epsilon_{FS}$ and $\epsilon_{ES}$ are significantly smaller for the first candidate than for the second, the reverse is true for $\epsilon_E$. This problem is amplified as the sequence length grows. These results indicate that $\epsilon_E$ should be used with caution, and that since other two error measures are preferable since they were defined specifically to avoid problems such as this.

\clearpage

\section{Parameter values}
\label{sect:params}

Each of the free parameters $(k, \delta, t, w, s, m, \mathcal{A})$ described in section \ref{sect:standard_config} are now covered in detail, by means of studying their anomaly vectors on the standard sequence $A_\alpha$ as well as the corresponding errors and evaluation times as $\alpha$ varies.

As mentioned previously, since the analysis in this section is based on studying the $\Theta_{\alpha}^*$ separately on the sequence $s^*$, it is not sufficient for any conclusions to be drawn either about global minima of $E_{\delta, S}(\Theta)$ or about how well the results might extend to other sequences. Instead, the analysis in this section should be considered a first step towards establishing a broader understanding of how the $E_{\delta, S}(\Theta)$ vary over the parameter spaces of distance-based problems, and as an introduction to \texttt{ad-eval}, including some useful ways to explore performance characteristics.

\subsection{The k value}
\FloatBarrier

\begin{figure}[h]
    \vspace{-10pt}
    \begin{center}
        \includegraphics[width=\textwidth]{resources/k_heat_map}
    \end{center}
    \vspace{-20pt}
    \caption{\small{Heat map showing $A_k$ for $k$ = $1, 2, \dots, 100$. Red and blue indicate high and low anomaly scores, respectively.}}
    \label{fig:k_heat_map}
\end{figure}

\begin{wrapfigure}{r}{0.5\textwidth}
    \vspace{-30pt}
    \begin{center}
        \includegraphics[width=0.5\textwidth]{resources/k_errors}
    \end{center}
    \vspace{-20pt}
    \caption{\small{Errors as a function of $k$ for the standard sequence.}}
    \label{fig:k_errors}
    \vspace{-10pt}
\end{wrapfigure}

It is important to study how the anomaly vectors vary with $k$; first, because the choice of $k$ is likely to have a large impact on the appearance of the anomaly vector, regardless of the data set; and second, because the kNN evaluator only operates on a single $k$ value at a time, it is in a sense the simplest distance-based evaluator, and thus an ideal tool for better understanding how the choice of $k$ impacts the analysis. This understanding is crucial in effectively designing other types of distance-based evaluators. 

In order to understand how the $k$ value affects the resulting anomaly vectors, the anomaly vectors $A_{k}$ for $k = 1,2,\dots,100$ were calculated. Figure \ref{fig:k_heat_map} shows the resulting anomaly vectors, displayed as a heat map in which all anomaly vectors have been individually normalized to lie in the unit interval. Corresponding values of the three error measures are shown in figure \ref{fig:k_errors}. Note that this plot shows only relative errors, as the three error graphs have been individually normalized to the unit interval.

\begin{wrapfigure}{l}{0.5\textwidth}
    \vspace{-20pt}
    \begin{center}
        \includegraphics[width=0.5\textwidth]{resources/k_execution_times}
    \end{center}
    \vspace{-20pt}
    \caption{\small{Evaluation times when varying $k$ on the standard sequence.}}
    \vspace{-20pt}
    \label{fig:k_execution_times}
\end{wrapfigure}

Finally, figure \ref{fig:k_execution_times} shows the computation times for calculating the anomaly vectors $A_k$. Since the implemented kNN method operates by brute force, the entire reference set must be evaluated regardless of $k$, so the constant evaluation time exhibited in this figure is expected. For any distance measure that is a metric, such as the Euclidean distance, more efficient methods exist.

The smoothness with which the $A_k$ vary with $k$ indicate that using several nearby $k$ in distance-based evaluators is not likely to significantly improve accuracy. Furthermore, at least in this case, $k=1$ minimizes all three error measures, and there is no indication that considering additional $k$ might help. While higher $k$ do lead to other regions being marked as anomalous, these regions do not correspond to relevant features. If this holds in general, there is no need to consider $k$ higher than $1$, and using linear combinations of several $k$ is not likely to lead to any significant increase in accuracy. However, a much more thorough evaluation is required before any conclusions can be drawn.

\subsection{The distance function}
\FloatBarrier

\begin{figure}[ht]
    \vspace{-20pt}
    \begin{center}
        \includegraphics[width=\textwidth]{resources/delta_heat_map}
    \end{center}
    \vspace{-20pt}
    \caption{\small{Heat maps showing $A_{k, \delta}$ for the Euclidean and DTW distances.}}
    \vspace{-10pt}
    \label{fig:delta_heat_map}
\end{figure}

For obvious reasons, the choice of the distance function $\delta$ can have a great impact on the anomaly vectors when using distance-based methods. The distance measures implemented in \texttt{ad-eval} are the Euclidean distance, the dynamic time warp (DTW) distance, and the compression-based dissimilarity measure (CDM). To investigate the relative performance of these, the anomaly vectors $A_{k, \delta}$ were examined. Note that since $A_{\delta}$ consists of only one value per distance measure, calculating only $A_\delta$ would have yielded insufficient data.

The \texttt{ad-eval} implementation of the CDM performed poorly. To begin with, it ran significantly slower than the other methods, rendering any comprehensive analysis impossible. Furthermore, it produced poor anomaly vectors. There are a few possible explanations for this. First, the z-normalization step of the SAX transformation (in which each extracted subsequence is given zero empirical mean and unit variance) leads to poor results on random data regardless of the distance measure. Secondly, the window width of $10$ used in the standard configuration means that the extracted sequences are short and can not be efficiently compressed, leading to a roughly constant distance value. While the CDM will likely perform better and with other parameters, it was decided that the CDM would not be investigated further due to its slowness.

\begin{wrapfigure}{r}{0.5\textwidth}
    \vspace{-30pt}
    \begin{center}
        \includegraphics[width=0.5\textwidth]{resources/delta_errors}
    \end{center}
    \vspace{-20pt}
    \caption{\small{Errors for $A_{k, \delta}$.}}
    \vspace{-10pt}
    \label{fig:delta_errors}
\end{wrapfigure}

Instead, the focus was placed on comparing the Euclidean and DTW distances. Heat maps of the resulting anomaly vectors are shown in figure \ref{fig:delta_heat_map} and a plot of the corresponding errors is shown in figure \ref{fig:delta_errors}. As is seen in the heat map, there is generally little difference between the outcomes of the two distance measures; the DTW distance gives slightly `cleaner' (i.e., with non-anomalous regions closer to $0$) anomaly vectors for very low values of $k$, while the Euclidean distance assigns a slightly lower score to the false anomalies encountered at high values of $k$. While there are some differences in the obtained errors---the DTW distance gives a better normalized Euclidean error, while the Euclidean distance generally gives better values of the other two errors---the evaluation is not sufficient to draw any conclusions about the relative merits of the two measures.

However, the fact that the DTW distance does not perform worse than the Euclidean distance in this evaluation is interesting. Since the DTW was designed to recognize long, shifted but relatively similar continuous sequences, it might be expected to perform poorly on other types of data, such as the short, random data used in this evaluation. The fact that this is not the case is a positive indication.

\clearpage

\subsection{Transformations}
\FloatBarrier

\begin{figure}[h]
    \vspace{-10pt}
    \begin{center}
        \includegraphics[width=\textwidth]{resources/dft_heat_map}
    \end{center}
    \vspace{-20pt}
    \caption{\small{Heat maps of the $A_{k, t} $ for $k = 1,2,\dots,100$ with and without the discrete Fourier transform.}}
    \vspace{-10pt}
    \label{fig:dft_heat_map}
\end{figure}

As discussed in chapter \ref{ch:transformations}, applying transformations to extracted subsequences prior to evaluation, such as to perform dimensionality reduction, might assist in discovering certain types of anomalies. While a large number of compressions and other transformations deserving investigation have been proposed, due to time constraints, only the discrete Fourier transform (DFT) was implemented in \texttt{ad-eval}.

The performance of the DFT was investigated by evaluating the standard sequence for $k = 1,2,\dots, n$ with and without the DFT. A heat map of the results is shown in figure \ref{fig:dft_heat_map}, and a plot of the corresponding errors is shown in figure \ref{fig:dft_errors}.

\begin{wrapfigure}{r}{0.5\textwidth}
    \vspace{-30pt}
    \begin{center}
        \includegraphics[width=0.5\textwidth]{resources/dft_errors}
    \end{center}
    \vspace{-20pt}
    \caption{\small{Errors of the $A_{k, t}$.}}
    \vspace{-10pt}
    \label{fig:dft_errors}
\end{wrapfigure}

While the DFT gave fairly accurate anomaly vectors for low values of $k$, it performed poorly overall, returning less accurate anomaly vectors and higher error values over all $k$. This is reasonable: the DFT is not expected to perform well on random data. A proper evaluation of the performance characteristics of kNN methods using the DFT would require a more diverse data set.

\clearpage

\subsection{The sliding window width}
\FloatBarrier
\label{sect:w}

\begin{figure}[h]
    \vspace{-15pt}
    \begin{center}
        \includegraphics[width=\textwidth]{resources/w_heat_map}
    \end{center}
    \vspace{-20pt}
    \caption{\small{Heat map of the $A_w$ for $w = 1, 2, \dots, 50$.}}
    \vspace{-10pt}
    \label{fig:w_heat_map}
\end{figure}

\begin{wrapfigure}{r}{0.5\textwidth}
    \vspace{-20pt}
    \begin{center}
        \includegraphics[width=0.5\textwidth]{resources/w_error}
    \end{center}
    \vspace{-20pt}
    \caption{\small{Errors for the anomaly vectors $A_w$.}}
    \label{fig:w_error}
    \vspace{-20pt}
\end{wrapfigure}

Since $w$, the sliding window width, determines the size of the elements used by the evaluator, it should have a significant impact on the size of detected features. To determine if this was the case, the anomaly vectors $A_w$ for $w = 1$, $2$, \dots, $50$ were computed and examined. The results are shown in figures \ref{fig:w_heat_map}, \ref{fig:w_error}, and \ref{fig:w_time}.

As seen in the figures, very low values of this parameter are associated with a very high error. This is expected, since as $w$ tends to $1$, the target anomaly type is reduced to point anomalies. Furthermore, all errors increase sharply as $w$ nears $20$, indicating that large values of $w$ lead to inaccurate results.

\begin{wrapfigure}{l}{0.5\textwidth}
    \vspace{-25pt}
    \begin{center}
        \includegraphics[width=0.5\textwidth]{resources/w_time}
    \end{center}
    \vspace{-20pt}
    \caption{\small{Evaluation times for the anomaly vectors $A_w$.}}
    \label{fig:w_time}
    \vspace{-20pt}
\end{wrapfigure}

Interestingly, the plot in figure \ref{fig:w_heat_map} shows that beyond $w \approx 3$, increasing $w$ essentially amounts to smoothing the resulting anomaly vectors. Since the anomaly in the standard sequence has a relatively small width of $40$, and since its surroundings have low anomaly values for low values of $w$, this could help explain why the anomaly is not detected after $w \approx 40$. 

It is further interesting to note that while the errors are at a minimum when $w \approx 5$, the anomaly vectors in this area contain three separate spikes in the vicinity of the anomaly, rather than a single smooth bump. Arguably, the anomaly vectors at $w \approx 10$ are preferable, since they more clearly mark the anomaly. This suggests that the error measures may need refinement.

Finally, while the evaluation time ought to be roughly independent of $w$ (or proportional to the evaluation time of the distance metric with vectors of length $w$), figure \ref{fig:w_time} shows a decrease in the evaluation time as $w$ grows. This is likely due to the fact that the relatively small width of the evaluation sequence means fewer elements are evaluated as $w$ grows. An evaluation performed on a long sequence, in which the evaluation filter operates on the middle of the sequence while the reference filter operates on the entire sequence, could be used to confirm this.

\subsection{The sliding window step}
\FloatBarrier
\label{sect:s}

\begin{figure}[h]
    \vspace{-20pt}
    \begin{center}
        \includegraphics[width=\textwidth]{resources/s_heat_map}
    \end{center}
    \vspace{-20pt}
    \caption{\small{Heat map of the anomaly vectors $A_s$ for $s = 1,2,\dots,10$. Note that no major false anomalies occur for $s < 8$.}}
    \vspace{-10pt}
    \label{fig:s_heat_map}
\end{figure}

\begin{wrapfigure}{l}{0.49\textwidth}
    \vspace{-30pt}
    \begin{center}
        \includegraphics[width=0.5\textwidth]{resources/s_error}
    \end{center}
    \vspace{-20pt}
    \caption{\small{Errors of the anomaly vectors $A_s$.}}
    \vspace{-20pt}
    \label{fig:s_error}
\end{wrapfigure}

The sliding window step, $s$, is interesting mainly for the large effect it has on the execution time. For a brute-force kNN evaluator with the trivial context and sliding window filters, the number of comparisons performed on a sequence of length $L$ is $\Theta((L/s)^2)$. It is therefor desirable to choose a value of $s$ that is as large as possible. However, it is likely that all three errors increase with $s$ for all sequences, and large $s$ values might lead to poor results.

To gain some insight into the performance of kNN methods for higher $s$, the anomaly vectors $A_s$ were computed for $s = 1$ to $10$ (the value of $w$ is $10$ in the default configuration). The results are shown in figures \ref{fig:s_heat_map}, \ref{fig:s_error}, and \ref{fig:s_time}.

\begin{wrapfigure}{r}{0.49\textwidth}
    \vspace{-40pt}
    \begin{center}
\includegraphics[width=0.5\textwidth]{resources/s_time}
    \end{center}
    \vspace{-20pt}
\caption{\small{Evaluation times for $A_s$. As expected, the graph indicates that the times are $O(1/s^2)$.}}
    \vspace{-10pt}
    \label{fig:s_time}
\end{wrapfigure}

As seen in figure \ref{fig:s_heat_map}, the anomaly vectors are fairly accurate for all $s$. No major false anomalies are exhibited for $s < 8$, and the actual anomaly is still clearly detected over all $s$. This is reflected in the errors in figure \ref{fig:s_time}: all errors are low until $s \geq 8$. Additionally, the evaluation time plot follows the expected $O(1/s^2)$ trend.

In light of these results, perhaps a multi-resolution scheme should be considered, in which a preliminary, `coarse' evaluation (corresponding to high $s$), and a `fine' evaluation (corresponding to low $s$) is performed only on those subsequences which are given the highest anomaly scores in the coarse evaluation. Depending on how the subsequences for the fine evaluation are selected, and on the context type, such an algorithm could achieve either lower computational complexity or an evaluation time reduction by a constant factor. If, as indicated in this evaluation, false positives but no false negatives are introduced as $s$ increases, fine evaluation would only rule out false anomalies, and there would be no loss of analytical power. 

\subsection{The context width}
\FloatBarrier
\label{sect:m}

\begin{figure}[H]
    \vspace{-20pt}
    \begin{center}
        \includegraphics[width=\textwidth]{resources/m_heat_map}
    \end{center}
    \vspace{-20pt}
    \caption{\small{Heat map of the $A_m$ for $m = 20, 21, \dots, 400$. Note the false anomaly present at the left end of the anomaly vectors until $m \approx 330$.}}
    \vspace{-10pt}
    \label{fig:m_heat_map}
\end{figure}

Which values of the context width $m$ are appropriate depends heavily on the application domain and on the types of anomalies present in the data. Ideally, the importance of the context width should be evaluated by considering several sequences with a natural context concept, such as the bottom series in figure \ref{fig:anomaly_types}. Constructing representative artificial data sets of such sequences is likely to be difficult, so real-world series should be used for such an evaluation.

\begin{wrapfigure}{r}{0.49\textwidth} \vspace{-20pt}
    \vspace{-10pt}
    \begin{center}
        \includegraphics[width=0.5\textwidth]{resources/m_error}
    \end{center}
    \vspace{-20pt}
    \caption{\small{Errors of the anomaly vectors $A_m$.}}
    \vspace{-20pt}
    \label{fig:m_error}
\end{wrapfigure}

While such data sets are not available, a simple evaluation on the available data can still prove illuminating. The standard sequence is highly homogeneous and has no natural contexts. Thus, all errors should be expected to decrease monotonically with increasing $m$. To confirm this, the anomaly vectors $A_m$ were computed for $m = 20$ to $400$. The results of this evaluation are shown in figures \ref{fig:m_heat_map}, \ref{fig:m_error}, and \ref{fig:m_time}. 

As these figures demonstrate, the anomaly vectors identify a false anomaly at the left end until $m \approx 330$, at which point the false anomaly disappears and the errors decline sharply. That this false anomaly appears for small context widths is understandable since, as seen in figure \ref{fig:reference_sequence}, the sequence includes values at its left end that are not seen again until the right end. As expected, the error is minimized when the trivial context (corresponding to $m > 390$) is incorporated.

\begin{wrapfigure}{r}{0.49\textwidth}
    \vspace{-30pt}
    \begin{center}
        \includegraphics[width=0.5\textwidth]{resources/m_time}
    \end{center}
    \vspace{-20pt}
    \caption{\small{Evaluation times of the anomaly vectors $A_m$.}}
    \vspace{-20pt}
    \label{fig:m_time}
\end{wrapfigure}

Finally, it should be noted that while the size of the reference set, and consequently the evaluation time, grows linearly with the size of the context, the average context size only grows linearly with $m$ when $m$ is much smaller than the sequence length. When $m$ is close to the sequence length, the context size for a large portion of the subsequences extracted by the evaluation filter will be limited by the sequence edges. This leads to the curve in figure \ref{fig:m_time}.

\clearpage

\subsection{The aggregator}
\FloatBarrier
\label{sect:A}

To get an idea of how the choice of aggregator affects the analysis, the anomaly vectors $A_{k, \mathcal{A}}$ were computed and analyzed for the minimum, maximum, median and mean aggregators, with $k = 1,2,\dots,100$. Heat map plots of the results are shown in figure \ref{fig:aggregator_heat_map}, and plots of the corresponding error measures are shown in figure \ref{fig:aggregator_error}. Single anomaly vectors for $k=1$ are shown in figure \ref{fig:aggregator_vectors}.

As seen in figures \ref{fig:aggregator_error} and \ref{fig:aggregator_vectors}, the min and max aggregators produce blockly, piecewice constant anomaly vectors, while the mean aggregator (and, to a lesser extent, the median aggregator) produces smooth, continuous anomaly vectors.

\begin{figure}[!ht]
    \vspace{-20pt}
    \begin{center}
        \includegraphics[trim = 5mm 2mm 10mm 2mm, clip, width=\textwidth]{resources/aggregator_heat_map}
    \end{center}
    \vspace{-10pt}
    \caption{\small{Heat maps showing $A_{k, \mathcal{A}}$ for the four aggregators.}}
    \label{fig:aggregator_heat_map}
    \vspace{-15pt}
\end{figure}

As could be expected, the minimum aggregator consistently led to the highest values of $\epsilon_{FS}$. It is likely to give a low score to a point if a single element containing that point has a low anomaly score, which effectively means that parts of anomalies will tend to be undervalued---something the full support error is sensitive to. In contrast, the maximum aggregator consistently led to the lowest support error values. This is also as expected, since max will assign high values to any point contained in an anomalous subsequence. The median and mean aggregators performed roughly equally well---while the mean performed better for higher $k$, this is not relevant; both aggregators were very far off for higher $k$.

\begin{figure}[!ht]
    \vspace{-15pt}
    \begin{center}
        \includegraphics[width=\textwidth]{resources/aggregator_error}
    \end{center}
    \vspace{-18pt}
    \caption{\small{Errors of the anomaly vectors $A_{k, \mathcal{A}}$.}}
    \label{fig:aggregator_error}
    \vspace{-10pt}
\end{figure}

Similar, but less clear, results were obtained for the equal support errors. The minimum aggregator consistently performed the worst with low $k$, while the maximum aggregator performed the best, on average, with $k$ up to $40$. Again, the mean and median aggregators performed too similarly for any conclusions to be drawn on their relative merits.

\begin{wrapfigure}{r}{0.49\textwidth}
    \vspace{-20pt}
    \begin{center}
        \includegraphics[width=0.5\textwidth]{resources/aggregator_vectors}
    \end{center}
    \vspace{-20pt}
    \caption{\small{Plot of the anomaly vectors $A_{k, \mathcal{A}}$ for $k=1$.}}
    \vspace{-0pt}
    \label{fig:aggregator_vectors}
\end{wrapfigure}

Finally, the normalized Euclidean error gives almost indentical values to the mean and median aggregators, but exhibits a clear preference for the minimum aggregator over the maximum aggregator. This is likely a consequence of the fact fact that the minimum aggregator tends to assign scores close to zero to all elements except for a few, while the maximum aggregator tends to assign scores close to zero to only a few elements. As discussed in section \ref{sect:error_measure_eval}, the normalized Euclidean error has a bias in favor of anomaly vectors where most elements are close to zero, unlike the type of anomaly vectors produced by the maximum aggregator. 

In conclusion, all aggregators performed roughly equally well on $s^*$ (arguably, the minimum aggregator performed slightly worse than the others). If this holds in general, then it appears that the choice of aggregator is mainly one of aesthetics.

\chapter{Discussion}
\label{ch:discussion}

We conclude this report with a short post-mortem and a discussion of future work.

\section{Post-mortem}

Overall, the project was successful. The new theory introduced in the form of the tasks and problems and component frameworks has made reasoning about and evaluating problems, as well as proposing novel problems and methods, significantly easier. Additionally, \texttt{ad-eval} has shown that the component framework can be effectively implemented and used to easily compare problems and methods. Furthermore, the evaluation utilities and evaluation scripts in \texttt{ad-eval} have shown that performing objective, reproducible method evaluations that can be reused with different data sets need not be difficult. Finally, as summarized in the next section, the project illuminated several new frontiers for future work.

However, there were some shortcomings. Since the initial focus on the implementation and evaluation of a few specific methods was not recognized as inappropriate until these methods had been partially implemented and large sections of the report had been written, and since the subsequent way forward was initially unclear, much of the work performed for the project was ultimately discarded.

Moreover, a proper evaluation of the methods implemented in \texttt{ad-eval} was beyond the scope of the project. While mitigated by the fact that the qualitative evaluation performed in chapter \ref{ch:evaluation} and the reusable evaluation scripts and utilities added to the repository will greatly facilitate such an evaluation once appropriate data sets are obtained, this rendered the project goal of finding the problem formulations most appropriate for the target domain only partially achievable.

\section{Future work}

As repeated throughout this paper, there remains much work to be done on task \ref{task:main} and related tasks, as well as on \texttt{ad-eval}. A few potential areas in which such work would be useful are now highlighted.

\subsection{Evaluation}

As mentioned several times in this paper, the evaluation performed as part of this project, alone, cannot conclusively answer which methods are appropriate for the target domain. As indicated in section \ref{sect:evaluation_data}, preferably labeled data from the target domain should be used in the evaluation. However, such data could not be obtained, so a qualitative evaluation of how parameter and component choices affect the output anomaly vectors was performed. Once appropriate evaluation data is obtained, several interesting questions could be answered.

First, the tests performed as part of the evaluation in this project should be re-run on a more diverse data set, to see if the conclusions made in chapter \ref{ch:evaluation} hold in general. Second, a larger portion of the parameter space should be evaluated. To mitigate the difficulties caused by the very long evaluation times required for such evaluations, a few modifications should be made to \texttt{ad-eval}. Methods for caching the results obtained in evaluations, along with tools for parallelizing and distributing evaluations, should be implemented. Tools for more effectively searching parameter spaces for minima should also be provided. The relative smoothness with which the output anomaly vectors seem to vary with parameter choices (at least for the kNN evaluator) could be exploited to avoid a search of the entire parameter space; methods such as discrete gradient descent combined with random restarts could significantly reduce the computation time as opposed to a brute-force search, likely without significant loss of accuracy. Finally, optimizations to the $\texttt{anomaly\_detection}$ module could potentially lead to large, constant-factor evaluation speedups.

\subsection{Performance}

Throughout the implementation of \texttt{ad-eval} and the evaluation, performance (in terms of computation time and memory usage) was deemphasized in favor of accuracy. This was done consciously, in order to limit the scope of the paper and to avoid the excessive focus on methods and optimizations found in much of the literature.

Several interesting questions regarding performance warrant investigation. Several types of both pure optimizations and approximations could be applied to the implemented problem formulations. For instance, multi-resolution algorithms, such as the one suggested in section \ref{sect:s}, could potentially lead to methods that are both fast and accurate.

Fortunately, the modular nature of \texttt{ad-eval} facilitates the evaluation of optimized methods. Just as a suite of unit, integration, and performance tests are often run after additions to commercial software projects, running performance and accuracy tests to evaluate how optimizations affect performance would be trivial in \texttt{ad-eval}.

\subsection{Components}

Several tasks and problems within the component framework would be interesting to study in more depth. As shown in chapter \ref{ch:methods}, there is large potential in suggesting and implementing several new methods within the component framework. For instance, there exist several anomaly measures and other components in the literature (including several model-based, artificial neural network-based, and statistical measures, as well as other distance-based methods and classifiers) that have not been implemented in \texttt{ad-eval}, and it is likely that some of these will perform better than the implemented components on certain datasets. Furthermore, several types of transformations should be evaluated (such as the discrete wavelet transform).

And of course, the parameter spaces of the implemented methods should also be studied in more depth. It would be especially interesting to examine how the error functions $E_{S, \delta}(\Theta)$ vary with the evaluation data $S$. If the minima were to be computed for a large number of sequences from the target domain, then which problem formulations best suit which kinds of data could be studied from several angles.

\subsection{Related tasks}

As indicated in section \ref{sect:related_tasks_framework}, the components framework could easily be adapted to cover a large number of related tasks, including the tasks suggested in section \ref{sect:suggested_tasks}. Evaluations like the one in this paper could then be performed on these related tasks. It would be especially interesting to compare how individual components fare on different tasks.

Of course, other tasks would require other types of data, which might be difficult to obtain. However, the amount of development time required to adapt \texttt{ad-eval} should be minimal.

\subsection{Deployment}

As indicated in section \ref{sect:design}, \texttt{ad-eval} was designed for eventual use in real-world applications. Due to its architecture, integration with software such as Splunk would be trivial. Then, the combination of \texttt{ad-eval} with a good user interface could prove an invaluable tool to monitoring and diagnosis through anomaly detection in a wide range of application domains. 

\subsection{Ensembles}

The components framework might also be used to find correlations between the accuracy of problem formulations and the underlying characteristics of the data being analyzed, as mentioned in section \ref{sect:framework}. If such correlations exist, \texttt{ad-eval} could prove instrumental in their discovery due to the ease with which multiple problem formulations can be applied to, and evaluated in relation to, data sets.

Ideally, this could lead to an ensemble-style approach, in which several methods are combined and weighed based on their relative suitability to the characteristics of the data.

\chapter{Conclusions}
\label{ch:conclusions}

We conclude this report with a short summary and a discussion of a few possible directions for future work.

\section{Summary}

Overall, the project was successful. The new theory introduced in the form of the tasks and problems and component frameworks has made reasoning about and evaluating problems, as well as proposing novel problems and methods, significantly easier. Additionally, \texttt{ad-eval} has shown that the component framework can be effectively implemented and used to easily compare problems and methods. Furthermore, the evaluation utilities and evaluation scripts in \texttt{ad-eval} have shown that performing objective, reproducible method evaluations that can be reused with different datasets need not be difficult. Finally, as summarized in the next section, the project illuminated several new frontiers for future work.

However, there were some shortcomings. Since the initial focus on the implementation and evaluation of a few specific methods was not recognized as inappropriate until these methods had been partially implemented and large sections of the report had been written, and since the subsequent way forward was initially unclear, much of the work performed for the project was ultimately discarded.

Moreover, a proper evaluation of the methods implemented in \texttt{ad-eval} was beyond the scope of the project. While mitigated by the fact that the qualitative evaluation performed in Chapter \ref{ch:evaluation} and the reusable evaluation scripts and utilities added to the repository will greatly facilitate such an evaluation once appropriate datasets are obtained, this rendered the project goal of finding the problem formulations most appropriate for the target domain only partially achievable.

\section{Future work}

As repeated throughout this paper, there remains much work to be done on Task \ref{task:main} and related tasks, as well as on \texttt{ad-eval}. A few potential areas in which such work would be useful are now highlighted.

\subsection{Evaluation}

As mentioned several times in this paper, the evaluation performed as part of this project, alone, cannot conclusively answer which methods are appropriate for the target domain. As indicated in Section \ref{sect:evaluation_data}, preferably labeled data from the target domain should be used in the evaluation. However, such data could not be obtained, so a qualitative evaluation of how parameter and component choices affect the output anomaly vectors was performed. Once appropriate evaluation data is obtained, several interesting questions could be answered.

First, the tests performed as part of the evaluation in this project should be re-run on a more diverse dataset, to see if the conclusions made in Chapter \ref{ch:evaluation} hold in general. Second, a larger portion of the parameter space should be evaluated. To mitigate the difficulties caused by the very long evaluation times required for such evaluations, a few modifications should be made to \texttt{ad-eval}. Methods for caching the results obtained in evaluations, along with tools for parallelizing and distributing evaluations, should be implemented. Tools for more effectively searching parameter spaces for minima should also be provided. The relative smoothness with which the output anomaly vectors seem to vary with parameter choices (at least for the kNN evaluator) could be exploited to avoid a search of the entire parameter space; methods such as discrete gradient descent combined with random restarts could significantly reduce the computation time as opposed to a brute-force search, likely without significant loss of accuracy. Finally, optimizations to the $\texttt{anomaly\_detection}$ module could potentially lead to large, constant-factor evaluation speedups.

\subsection{Performance}

Throughout the implementation of \texttt{ad-eval} and the evaluation, performance (in terms of computation time and memory usage) was deemphasized in favor of accuracy. This was done consciously, in order to limit the scope of the paper and to avoid the excessive focus on methods and optimizations found in much of the literature.

Several interesting questions regarding performance warrant investigation. Several types of both pure optimizations and approximations could be applied to the implemented problem formulations. For instance, multi-resolution algorithms, such as the one suggested in Section \ref{sect:s}, could potentially lead to methods that are both fast and accurate.

Fortunately, the modular nature of \texttt{ad-eval} facilitates the evaluation of optimized methods. Just as a suite of unit, integration, and performance tests are often run after additions to commercial software projects, running performance and accuracy tests to evaluate how optimizations affect performance would be trivial in \texttt{ad-eval}.

\subsection{Components}

Several tasks and problems within the component framework would be interesting to study in more depth. As shown in Chapter \ref{ch:methods}, there is large potential in suggesting and implementing several new methods within the component framework. For instance, there exist several anomaly measures and other components in the literature (including several model-based, artificial neural network-based, and statistical measures, as well as other distance-based methods and classifiers) that have not been implemented in \texttt{ad-eval}, and it is likely that some of these will perform better than the implemented components on certain datasets. Furthermore, several types of transformations should be evaluated (such as the discrete wavelet transform).

And of course, the parameter spaces of the implemented methods should also be studied in more depth. It would be especially interesting to examine how the error functions $E_{S, \delta}(\Theta)$ vary with the evaluation data $S$. If the minima were to be computed for a large number of sequences from the target domain, then which problem formulations best suit which kinds of data could be studied from several angles.

\subsection{Related tasks}

As indicated in Section \ref{sect:related_tasks_framework}, the components framework could easily be adapted to cover a large number of related tasks, including the tasks suggested in Section \ref{sect:suggested_tasks}. Evaluations like the one in this paper could then be performed on these related tasks. It would be especially interesting to compare how individual components fare on different tasks.

Of course, other tasks would require other types of data, which might be difficult to obtain. However, the amount of development time required to adapt \texttt{ad-eval} should be minimal.

\subsection{Deployment}

As indicated in Section \ref{sect:design}, \texttt{ad-eval} was designed for eventual use in real-world applications. Due to its architecture, integration with software such as Splunk would be trivial. Then, the combination of \texttt{ad-eval} with a good user interface could prove an invaluable tool to monitoring and diagnosis through anomaly detection in a wide range of application domains. 

\subsection{Ensembles}

The components framework might also be used to find correlations between the accuracy of problem formulations and the underlying characteristics of the data being analyzed, as mentioned in Section \ref{sect:framework}. If such correlations exist, \texttt{ad-eval} could prove instrumental in their discovery due to the ease with which multiple problem formulations can be applied to, and evaluated in relation to, datasets.

Ideally, this could lead to an ensemble-style approach, in which several methods are combined and weighed based on their relative suitability to the characteristics of the data.


%-----------------------------------------------------------
\addcontentsline{toc}{chapter}{\numberline{}Bibliography}
\include{bibliography}
\bibliography{mybib1}
\bibliographystyle{acm}

%-----------------------------------------------------------
%\appendix
%-----------------------------------------------------------
\end{document}
