\chapter{Introduction}

\begin{figure}[htb]
    \vspace{-10pt}
    \begin{center}
        \includegraphics[trim= 30mm 10mm 30mm 10mm, clip, width=\textwidth]{resources/citations}
    \end{center}
    \vspace{-20pt}
    \caption{\small Approximate number of papers (by year) published between $1980$ and $2011$ containing the terms ``anomaly detection'', ``outlier detection'' and ``novelty detection''. All three terms exhibit strong upward trends in recent years. Source: Google Scholar.}
    \vspace{-0pt}
\label{fig:citations}
\end{figure}

This report is the result of a master's thesis project at the KTH Royal Institute of Technology, performed partly in conjunction with an internship at Splunk Inc.\@, based in San Francisco, California, USA\@. % The goal of the project was to develop efficient and general methods of anomaly detection suitable for sequences (and especially real-valued continuous time series).

% Splunk is essentially a database and tool for storing and analyzing very large sets of machine-generated data. The term \emph{machine-generated data} refers to any data consisting of discrete events that have been created automatically from a computer process, application, or other machine without the intervention of a human. Common types of machine-generated data include computer, network, or other equipment logs; environmental or other types of sensor readings; or other miscellaneous data, such as location information~\cite{machine_data}. Splunk is designed for this type of data, especially datasets where each event has an associated time stamp.

Roughly defined as the automated detection within datasets of elements that are somehow abnormal, anomaly detection encompasses a broad set of techniques and problems. In recent years, anomaly detection has become increasingly important in a variety of domains in business, science and technology. In part due to the emergence of new application domains, and in part due to the evolving nature of many traditional domains, new applications of and approaches to anomaly detection and related subjects are being developed at an increasing rate, as indicated in Figure~\ref{fig:citations}.

% Since anomaly detection is an important and common problem in the domains in which Splunk is used, it can be expected that efficient and general anomaly detection tools could be of great benefit to Splunk. Furthermore, since real-valued time series are easy to form from machine-generated data with timestamps, and are relatively amenable to analysis, anomaly detection methods for real-valued time series can be expected to be especially useful.

Anomaly detection tasks are encountered in almost every domain of science, business and technology, and providing efficient methods for solving these tasks ahs potentially enormous benefits. Typically, however, finding appropriate anomaly detection methods for a given application is a laborious process, which requires expertise both both in the specific application and in anomaly detection methods. This affects the uptake of anomaly detection methods negatively. A key challenge in anomaly detection research is providing automated tools that can be used to streamline and simplify the research process.

With the above in mind, it was decided that the aim of this thesis project would be to investigate efficient automated methods for anomaly detection research. The main contributions of this thesis are:
\begin{enumerate}
    \item A optimisation problem formulation of the task of finding appropriate anomaly detection methods.
    \item A framework for reasoning about anomaly detection problems and guiding the optimisation.
    \item A software implementation of the optimisation problem and framework.
\end{enumerate}

In Chapter~\ref{ch:background}, some background information useful to the rest of the report is presented. Specifically, the subject of anomaly detection is discussed in more depth, along with a few basic concepts. Some of the major problems faced in anomaly detection research are also discussed. Finally, the optimisation problem approach is introduced.

As a means of overcoming these hurdles, in Chapter~\ref{ch:framework}, a framework for reasoning about anomaly detection problems is introduced. As part of the framework, a few novel concepts and generalisations of existing concepts are introduced.

Next, in Chapter~\ref{ch:time_series}, an application of the framework to anomaly detection in sequences is presented. How existing methods fit in with the framework is also dicussed.

In Chapter~\ref{ch:implementation} a software implementation, called ADRT of the optimisation problem and framework is presented.

Chapter~\ref{ch:results} consists of a preliminary investigation into how ADRT can be used to perform the optimisation and gain insights into how well different types of problems perform for a given application.

The report is concluded in Chapter~\ref{ch:conclusions} with a summary of the project and a few possible directions for future work.
